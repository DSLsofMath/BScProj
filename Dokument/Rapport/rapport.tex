\documentclass[]{article}

\usepackage[swedish]{babel}
\usepackage[utf8]{inputenc}
\usepackage[T1]{fontenc}
\usepackage{hyperref,url,bbm,graphicx,icomma,units,lmodern,amsmath,glossaries,minted}

\loadglsentries{ordlista.tex}

\date{\today}

%Ordlistan skapas här.
%Måste refereras till något med \gls{ref} för att kunna visas.
\makeglossaries

\begin{document}

\begin{titlepage} \newcommand{\HRule}{\rule{\linewidth}{0.3mm}}
  \center
  \textsc{\Large Chalmers tekniska högskola}\\[0.05cm]
  \normalsize \today

  \HRule \\[0.08cm]
  { \large Programmering som undervisningsverktyg för signaler och system
    \\
    \normalsize{Slutrapport}
  } \\[0.08cm] %Signalteori är för stort?
  \HRule \\[0.3cm]

  \vfill

  \begin{flushleft}
    \small
    \emph{Författare: \\
      \quad Filip Lindahl\\
      \quad Cecilia Rosvall\\
      \quad Peter Ngo\\
      \quad Jacob Jonsson\\
      \quad Joakim Olsson\\}
  \end{flushleft}
\end{titlepage}
\newpage

\renewcommand*\abstractname{Sammandrag}
\begin{abstract}
  %Ett Sammandrag
%Subject to Change!--------------------------------------------
% Huvudpoänger:
% * Vad handlar rapporten om?
%     Vad är det för projekt vi beskriver utvecklingen av?
%     (Läsare ska här kunna bedöma om rapporten är intressant att läsa för dem)
% * Jättekort om syfte, resultat etc
% Illustration: Borde inte behövas
Denna rapport beskriver utvecklingen av läromaterialet “TSS med DSL” med dess tillhörande programmeringskod och lösningar.
Läromaterialet riktar sig till studenter på det datatekniska programmet på Chalmers och har som syfte att vara ett kompletterande läromedel inom signallära där man utnyttjar studenternas kunskaper inom programmering. 
Materialet innehåller förklarande text och teori, samt programmeringsövningar med domänspecifika språk.

Bakgrunden till detta projekt är att kursen TSS på det aktuella programmet under många år har haft väldigt dåliga studieresultat på grund av studenternas ovana att hantera den typ av matematik som utgör grundstommen i kurserna. 
Detta har visats sig inte vara något som är unikt för datastudenterna på Chalmers utan det är tvärtom ett problem inom många discipliner.

Rapporten beskriver vidare de undersökningar och efterforskningar som gjorts för att ta kunna ta fram läromaterialet, samt de tester som gjorts på den färdiga produkten.

\end{abstract}

\renewcommand*\abstractname{Abstract}
\begin{abstract}
This report describes the development of the supplementary tutorial ''Transforms, Signals and Systems with the help of a DSL'' with provided code and solutions.
The tutorial is aimed at students studying Computer Science at Chalmers University of Technology and its aim is to be a supplementary teaching material for the course ''Transforms, Signals and Systems'' where you will be progamming in domain specific languages.

The background to this project is that the computer science students studying the course ''Transforms, Signals and Systems'' have gotten poor results on the tests for several years and this might be because of the CS students unfamiliarity with the kind of mathematics which is the backbone of that course. 

The report further describes the surveying and analysis done to be able to produce this tutorial along with the tests that have been done on the finished product.
\end{abstract}
\newpage



\newpage
\tableofcontents

\newpage

\printglossary[title=Ordlista,nonumberlist]

\newpage

\setlength{\parskip}{2mm}
\setlength{\parindent}{0pt}


\section{Inledning}
Vi kommer nu inledningsvis att gå igenom bakgrunden till projektet och hur det
uppstod, samt beskriva projektets mål och rapportens syfte.

\subsection{Bakgrund}
% Huvudpoänger:
% * Problem med kurserna på data
% * Var kommer vårt projekt ifrån? Spinoff till DSLOfMathprojektet.
% * Ev. ta upp i korthet att detta är en del av ett större problem.
%     Hänvisa till TFPIE-artikelns källor.
%
% Illustrationer: Borde inte behövas

På Chalmers Tekniska Högskola finns det en lång trend där
studenterna på utbildningen för Datateknik har haft svårigheter
med kurserna \textit{Transformer, Signaler och System}, hädanefter omnämnd
som ``\gls{TSS}'', och \textit{Reglerteknik}.
Dessa svårigheter tror kursernas examinatorer beror till stor del på att
studenterna inte är tillräckligt bekväma i det matematiska språket.
Detta syns bland annat i statistiken för hur många datastudenter som
klarar kurserna.

För att minska svårigheterna som studenterna har för dessa kurser påbörjades
ett pedagogiskt projekt, DSLsOfMath, som hittills resulterat i kursen
Matematikens domänspecifika språk, där man använder funktionell
programmering för att beskriva matematiska problem \cite{kursplan:dslsofmath}.
Resonemanget bakom detta var att funktionell programmering är ett verktyg
som datastudenterna har erfarenhet av och som använder en notation med
fokus på tydlighet som lämpar sig relativt väl för att lära ut matematiska
begrepp \cite{TFPIE15_DSLsofMath_IonescuJansson}.

Detta projekt uppstod som en avgrening från DSLsOfMath-projektet med fokus på
att utveckla material till specifika kurser snarare än en allmän matematisk
grund.

\subsection{Rapportens syfte}

% Huvudpoänger:
% *Vad vill vi säga med rapporten? Vad vill vi visa på?
% *Vad har vi försökt göra  i korthet?

% Illustration:
Syftet med denna rapport är att beskriva utvecklingen av handledningen
``TSS med DSL'', som är ett kompletterande läromaterial till kursen TSS
för studenter på datatekniska programmet på Chalmers,
samt beskriva hur det bakomliggande problemet kan kopplas till en mer
generell problematik.

\subsection{Projektets mål}
% Huvudpoänger:
% *Beskriv projektets syfte och mål, vad ska vi åstadkomma för produkt?
Syftet med projektet är att underlätta för studenter inom datateknik att
ta till sig signal- och systemteoretiska ämnen genom att utnyttja deras
kunskaper inom programmering och även göra det möjligt att betrakta ämnet ur
en programmerares perspektiv. Tanken bakom detta är att man ska göra gapet
mellan datateknik och signalteori mindre och minska problemen som nämns
i bakgrundsavsnittet.

Projektet är tänkt att resultera i ett läromaterial som kan fungera som ett
komplement till den kurs som ges inom signalteori på den datatekniska
grundutbildningen. Detta läromaterial ska innehålla förklaringar och
programmeringsövningar som är anpassade för studenter på den datatekniska
utbildningen på Chalmers tekniska högskola.

\section{Avgränsningar}
% Huvudpoänger:
% * Vad behandlar vi inte i rapporten och varför?
% * Vad tas inte med i vårt projekt och varför?
% Illustrationer: Borde inte behövas

I projektet har gruppen avgränsat sig till att endast ta fram läromaterial
till endast en kurs, TSS. Detta då projektet är tidsbegränsat.
Läromaterialet som tagits fram är ett kompletterande material och är inte
tänkt att kunna ersätta kursen eller dess material.

Gruppen har under utvecklingen av materialet inte heller lagt fokus på matematisk korrekthet, 
matematiska bevis, avancerade matematiska begrepp eller teorier.  Anledningen till detta är att 
materialet  i första hand är tänkt att främja förståelse och väcka intresse för ämnet. Detta beslut 
baserades även på \textit{Didaktik för ingenjörslärare} där författarna, under en diskussion om aktivt lärande, 
hävdar att: ”När man lär sig något kan man få det att fastna på olika sätt. En ytligt inlärd kunskap 
eller ett beteende som inte tränats tillräckligt glöms snart bort. Det är med andra ord viktigt att 
metoderna tillämpas så att djuplärande uppnås. Textens detaljer är då inte lika intressanta som textens 
budskap, förståelse, sammanhang och principer. Snarare än att lära sig ”det man måste” utantill för 
reproduktion, sker ett personligt tillägnande relaterat till tidigare kunskap.” I detta projekt är det 
alltså viktigt att ta fram ett interaktivt material som främjar förståelse och inte fastna i detaljer. 

\section{Problemanalys}
%Huvudpoänger:
% * Vad är det för problem projektet ska lösa?
% * Varför har vi har vi valt att lösa problemet så här?
% * Knyter vårt problem an till ett större problem?
% * Ev. hänvisa kort till relaterad forskning

%Illustrationer: Statisktik från kurserna, pajdiagram

%Problemanalysen skall leda fram till produktspecen
Som nämndes i bakgrunden är problemet vi tacklar i vårt projekt att
datastudenterna på Chalmers har svårt för två av sina kurser,
TSS och Reglerteknik.

Statistiken från Chalmers visar tydligt att kurserna TSS och Reglertiknik
har en ovanligt hög andel studenter som inte blir godkända på tentamen.
Under perioden 2010 till 2016 blev i genomsnitt endast 51
% av de som skrev tentamen i TSS godkända och under samma period blev 53%
% godkända i Reglerteknik \cite{tentastatistik}.
%Pajdiagram
Det finns även ett mörkertal i dessa siffror eftersom alla studenter
inte alltid skriver tentamen, exempelvis valde 48 av 122 registrerade
studenter att inte tentera Reglerteknik 2014
\cite{kursinformation:ere102:14-15}.

Kurserna i fråga, TSS och Reglerteknik, läses under tredje året på
datateknik och kräver bland annat förkunskaper från de kurser i matematik
studenterna läser under sitt första år. Reglerteknik kräver även förkunskaper
från TSS. Det kan därför anses vara ett rimligt antagande att bristande
kunskaper i TSS orsakar följdproblem och utgör en del av studenternas
problem med Reglerteknik.

Kursernas examinatorer tror att studenternas svårigheter med kurserna till
stor del beror på att studenterna inte är tillräckligt bekväma
i det matematiska språket. Studenterna på Datateknik har inte tillräcklig
vana att tolka innebörden av de matematiska beskrivningarna och är allmänt
ovana vid matematiskt hantverk. Därför lägger datastudenterna det mesta av
sin energi på att få ihop matematiken rent praktiskt,
istället för att se vad matematiken står för.

%Hänvisa till relaterad forskning och knyt an till bredare problem

\section{Produktspecifikation}
% Huvudpoänger:
% * Beskriv produktbeskrivningen vi fick från början
% * Bena ut i detalj hur produktbeskrivningen vi tagit fram ser ut och
%     kommentera vilka val vi har gjort.
% * Förklara hur produkten löser/bidrar till att lösa problemet

%Illustration: Ev bild på utdrag från tutorial för att tydliggöra upplägget

För att åtgärda det problem som beskrivs i förgående avsnitt har gruppen skapat
en unik produkt som är anpassad speciellt till TSS på Chalmers datatekniska
program. Den ursprungliga produktspecifikationen som gruppen tilldelades vid
arbetets start var relativt vag. Gruppens uppgift var att ``ta fram
DSLsofMath-inspirerat kompletterande material för andra närliggande kurser''.
Det önskade implementeringsspråket var Haskell.

Gruppen beslutade snabbt att fokusera på kurserna TSS och Reglerteknik av de
anledningar som nämns i problembeskrivningen. Dock konstaterade gruppen en bit
in i projektet att det endast fanns tid att utveckla material av önskvärd
kvalitet till en kurs. Gruppen valde då kursen TSS, med motiveringen att
TSS innehåller förkunskaper som är nödvändiga för Reglerteknik.

\subsection{Produktens struktur}
För att göra läromaterialet så lättillgängligt som möjligt beslutades att
all materiel skulle samlas på en hemsida. På detta sätt kan man enkelt
samla text, bilder och programmeringskod och presentera dem på ett
överskådligt sätt för studenterna i målgruppen.

Det beslöts att läromaterialet skulle innehålla ett antal delavsnitt som
tillsammans täckte in innehållet i kursen TSS i stora drag.
Avsikten med indelningen är att göra materialet mer överskådligt och
även göra det möjligt för studenterna att enkelt hitta aktuellt material
för ett specifikt område.

Varje delavsnitt ska innehålla förklaringar och exempel i löpande text.
Denna text ska skrivas på ett sådant sätt att den är lätt att ta till
sig för målgruppen och uppmuntrar till vidare studier av ämnet.

Delavsnitten ska vidare innehålla exempel på hur det aktuella ämnet kan
beskrivas med funktionell programmering och DSL, samt programmeringsövningar.

\subsection{Programmeringskod och övningar}

I själva impementationen av DSL så undviks funktioner som är
ovanliga eller kan skrivas explicit. t.ex så ska fouriertransformen av
$sinh(x)$ inte implementeras eftersom $sinh(x)$ kan utryckas via dess
definition $\frac{e^{x} - e^{-x}}{2}$ istället.
Dessutom undviks onödigt komplexa funktioner som inte
dyker upp naturligt som en signal, t.ex $sin(cos(x)+e^x)$.

\section{Teknisk bakgrund}
I detta kapitel så beskrivs det kortfattad de områden och programvara som
behövs för att bygga upp handledningen.
% Huvudpoänger:
% * Ta upp och introducera tekniken vi bygger vår tutorial på.

\subsection{Allmänt om domänspecifika språk}
% Allmänt om DSL
Ett domänspecifikt språk (\gls{DSL}) är ett programmeringsspråk som till
skillnad från generella programmeringsspråk är anpassat för en
specifik domän. Att skapa domänspecifika språk är ingenting nytt och det
finns flera exempel på DSL är HTML, MATLAB och Game Maker Language.
Dessa DSL har gemensamt att det är relativt enkelt att formulera lösningar för
problem i deras domäner, exempelvis är det enkelt att strukturera upp text och
annat innehåll i HTML, utföra matrisoperationer i MATLAB eller skapa datorspel
med Game Maker Language.

När man designar domänspecifika språk kan man välja att skapa ett fristående
språk som analyseras till någon typ av maskinkod eller att utveckla sitt
domänspecifika språk i ett redan existerande generellt programmeringsspråk,
där beståndsdelarna i språket är funktionsanrop och konstruktorer i värdspråket.

Det finns även olika nivåer av DSL, så kallade djupa DSL (deep embedding) och
ytliga DSL (shallow embedding). I introduktionen till sin artikel skriver
Svenningson och Axelsson \cite{Svenningsson2013} att man i ett djupt
DSL använder sig av abstrakta datatyper och att man med dessa bygger upp ett
abstrakt syntaxträd som sedan tolkas av olika funktioner.
Man skapar alltså egna typer och funktioner som bygger upp strukturer med dessa
typer och skiljer på så sätt den domänspecifika logiken från värdspråkets logik.
I ett ytligt DSL använder man sig däremot av värdspråkets typer och försöker
representera den domänspecifika logiken med korresponderande funktioner och
typer i värdspråket.
Båda sätten att beskriva DSL har sina fördelar och nackdelar då det exempelvis
i ett djupt DSL är förhållandevis lätt att ge fler tolkningar av det abstrakta
syntaxträdet och därmed kunna bygga vidare på domänspecifika logiken,
men svårare att lägga till konstruktorer eller på andra sätt ändra i typerna som
bygger upp syntaxträdet eftersom man då måste ändra i de funktioner som tolkar
dem. I ett ytligt DSL är det lätt att lägga till konstruktorer och typer
eftersom man bara måste kunna representera dem med värdspråket och alltså inte
behöver hantera ett syntaxträd, men å andra sidan är dess semantik fixerad.
För att lägga till nya sätt att tolka ett ytligt DSL måste vi implementera
om det på nytt.

\subsection{Om funktionell programmering}
% -Referera till “Why functional programming matters”

\subsection{Om transformer, signaler och system}
Området transformer, signaler och system inom signalbehandling beskriver olika typer av signaler,
hur de relateras till ett system och hur en ingenjör kan modellera denna relation mellan signaler och systemet 
i matematiska termer. Ett system kan till exempel vara en fjäder vars signal är en kraft som påverkar fjädern utifrån. 
Inom detta område så begränsas signalen till enbart en oberoende variabel som brukas väljas 
som tiden och att systemen är linjära och tidsinvarianta. Det vill säga, ett LTI-system.
För att kunna lösa problem inom dessa system så kan man utveckla signalen i en fourierserie eller använda 
sig av olika typer av transformer vilket låter en betrakta signalen i en annan domän. 
\begin{itemize}
\item Fouriertransform - En operator som transformerar en funktion från tidsdomän till en annan domän som brukar kallas för frekvensplanet. 
\item Laplacetransform - En transform i kontinuerlig tid som transformerar funktionen till s-planet, en generalisering av den kontinuerliga frekvensplanet. Anses vara lättare att hantera jämfört med Fouriertransform.  
\item Z-transform - En transform i diskret tid som transformerar funktionen till Z-planet, en generalisering av den diskreta frekvensplanet.
\end{itemize}
Dessutom så innehåller området andra begrepp och operator som komplexa tal, faltning, sampling m.m. 
Alla dessa grundkunskaper och transformer behövs för att modulera system och signaler.


\subsection{Didaktik}
Didaktik är vetenskapen om undervisning.
I \textit{Didaktik för ingenjörslärare} hävdar författarna att
didaktik kan ses som ett komplement till pedagogik.
De didaktiska frågorna handlar, enligt författarna,
inte om hur studenten lär sig utan snarare om hur läraren skapar en
situation lämplig för lärande.

Som nämndes i inledningen har projektet “Matematikens domänspecifika språk”
som syfte att utveckla läromaterial som kan fungera som ett komplement till
kursen \textit{TSS} och väcka intresse för ämnet.
Därför är didaktik en viktig del av projektet.

En stor del av didaktik handlar om hur man kan påverka studentens motivation.
%TODO: stavning: "ingegörslärare" := "ingenjörslärare"
I \textit{Didaktik för ingejörslärare} sammanfattas motivationens betydelse
enligt följande: ”Varje kurs behöver motiveras för studenterna.
Med motiverade studenter blir resultatet bättre.”.
%TODO: Upprepa inte titeln här. "Boken hävdar" eller liknande blir bättre.
%\textit{Didaktik för ingegörslärare} 
Boken hävdar även att långt ifrån alla studenter
som går en kurs utgår från att den är läsvärd.
Även i \textit{Motivational Design for Learning and Performance} påpekas det
orimliga i att anta att alla studenter är motiverade att lära sig ett ämne
redan när de först kommer i kontakt med det.
Det är alltså viktigt att ett läromedel är uppbyggt så att det motiverar
studenterna att lära sig.

Enligt \textit{Motivational Design for Learning and Performance} behöver
motivationsaktiviteter stödja inlärningsmålen för att vara effektiva.
Motivationsmedel kan vara roligt och underhållande men om de inte engagerar
studenten i läromålen och läromaterialet så hjälper det inte studenten att
lära sig. Roliga aktiviteter och dylikt kan användas som ett belöningssystem
men främjar i sig inte lärande. Många studenter kommer även motsätta sig
eller rentav avsky aktiviteter tänkta att motivera dem om aktiviteterna
inte är knytna till läromålen, detta gäller särskilt för vuxna studenter.
Motivationsdesign har alltså utmaningen att göra materialet tilltalande
utan att det blir ren underhånning.

Ett angreppssätt inom didaktik är \textit{ARCS-modellen},
som beskrivs i detalj i
\textit{Motivational Design for Learning and Performance}.
ARCS är ett akronym för \textit{Attention}, \textit{Relevance},
\textit{Confidence} och \textit{Satisfaction}, vilket är grundpelarna
i \textit{ARCS-modellen}.

Den första grundpelaren \textit{Attention}, uppmärksamhet på svenska,
syftar till att väcka studentens uppmärksamhet och nyfikenhet.
Detta åstadkoms med perceptuellt triggande, frågor och variation.
Perceptuellt triggande innebär att man på något sätt ändrar i
undervisningsmiljön för att väcka studentens uppmärksamhet.
Dessa miljöförändringar kan vara av varierande typ som t.ex. en ändring i
röstläge, temperatur i en undervisningssal eller ett skämt.
% "en tillfälligt" := ""en tillfällig"
Denna typ av miljöförändringar väcker dock endast en tillfällig uppmärksamhet
och måste därför följas av frågor och variation för att ge en varaktig effekt.
Därför är det viktigt att i detta skede omedelbart följa upp med frågor som väcker
studentens nyfikenhet. För att behålla hens nyfikenhet och
intresse är det viktigt med variation undervisningen.
Oavsett hur bra struktur man har i sin undervisning kommer studenterna att tappa
intresset om man inte varierar upplägget.

Den andra grundpelaren \textit{Relevance}, relevans på svenska,
syftar till att övertyga studenten om att inlärningen är personligt relevant.
För att en student ska vara intresserad av att lära sig måste hen känna
att instruktionen är relaterad till personliga mål eller motiv.
Detta uppnås genom målorientering och anknytning till bekant kunskap.
Studenter är mycket mer motiverade att lära sig saker om dessa kan hjälpa dem
uppnå ett mål i nutid eller framtid, som att klara ett prov, få en befordran,
eller liknande. Därför är det viktigt att studenterna vet hur det de lär sig är
relaterat till deras mål. Ämnet kommer även att kännas mer relevant för studenten
om det kan knytas till hens tidigare kunskap eller intresse. Även om studenter kan
bli nyfikna på helt nya saker är de som regel mest intresserade av saker som på
något sätt är knutna till deras tidigare erfarenheter.
Konkreta exempel från familjära områden gör det hela mer relevant för studenten,
detta gäller i synnerhet när man lär ut abstrakt material.

Den tredje grundpelaren \textit{Confidence}, självförtroende på svenska,
syftar både till studenters tro på att de kan ämnet och deras tro på att
de kan lära sig det. Även om en student är nyfiken på ett ämne och känner
att det är relevant är det möjligt att hen ändå inte är tillräckligt
motiverade på grund av för mycket eller för lite självförtroende.
Vissa studenter kan rentav vara rädda för ämnet. På grund av detta är det
viktigt att utveckla materialet så att studenten övertygas om att hen kan
lära sig ämnet. Därför behöver man med hjälp av relevanta uppgifter ge studenter
en upplevelse av framgång så fort som möjligt. Detta kan vara en viktigt
stimulans för fortsatt motivation, förutsatt att uppgiften kräver så pass
mycket ansträngning att det betyder något att lyckas med den, men inte kräver
så mycket att det uppgiften leder till allvarlig oro eller hot om misslyckande.

Studentens självförtroende kan byggas upp genom lärokrav,
möjlighet till framgång och personlig kontroll.
Det är viktigt att sätta upp tydliga läromål eftersom det är en stor
källa till oro och osäkerhet för studenten att inte veta vad som
förväntas av hen. Efter att ha skapat en förväntan av framgång är det
viktigt att ge studenten möjlighet att lyckas med krävande och
meningsfulla uppgifter. Dessa uppgifter bör se olika ut beroende på
vilken nivå studenten befinner sig på. Studenter som är nya inom ett område
reagerar generellt bäst på om att ha en relativt låg svårighetsnivå med frekvent
återkoppling som hjälper dem lyckas eller bekräftar deras framgång.
När studenter bemästrat grunderna är de redo för mer krävande uppgifter.
Slutgiltligen är det viktigt att studenten känner att hen har personlig
kontroll över lärandesituationen. Att uppleva att man själv kan
kontrollera utgången av en situation är avgörande för självförtroendet.
För att förstärka motivationen bör instruktören förse studenten med en
stabil lärandemiljö och sedan låta studenten ha personlig kontroll över sin
inlärning. Det är viktigt att skapa en miljö där det är acceptabelt och helt
i sin ordning att göra misstag och lära sig av dem.

Om man använder de första tre grundpelarna, Attention, Relevance och Confidence,
väl, kommer studenten vara motiverad att lära sig.
Det är för att upprätthålla denna motivation som den sista grundpelaren
\textit{Satisfaction}, tillfredsställelse på svenska, används.
För att studenten ska ha ett fortsatt intresse av att lära måste hen känna
tillfredsställelse med inlärningens process eller resultat.
Denna tillfredställelse kan uppnås genom naturliga konsekvenser och allmänna positiva
konsekvenser. Med naturliga konsekvenser menas den kunskap som blir
en direkt följd av undervisningen. Att kunna lösa en uppgift eller dylikt
som man inte kunde lösa innan ger en tillfredställelse i sig.
Det är ett starkt belöningsverktyg att låta studenten använda ny kunskap.
Allmänna positiva konsekvenser kan vara materiella belöningar,
som till exempel en löneförhöjning, eller symboliska belöningar som diplom
eller ett bara ett erkännande av studentens förmåga.

Hur dessa didaktiska modeller och metoder har använts i projektet
``Matematikens domänspecifika språk'' beskrivs i avsnittet
\textit{Metod och Genomförande}.

\subsection{Om Relaterad forskning}
% DSLOFMATH-projektet som ledde till kursen

\section{Metod och Genomförande}

% Huvudpoänger:
% * Förstudier, intervjuer mm
% * Enkätundersökning
%    * Hänvisa till böckerna “Enkäten i praktiken” och “Enkätboken”
% * Didaktik, hur vi skrivit vår tutorial och varför
%    * Hänvisa till “Didaktik för ingenjörer” och “Motivational Design
%        for Learning and Performance”
% * TSS, hur vi har tacklat ämnet
% * Funktionell programmering och dsl, hur vi valt att
%     implementera det hela och varför
% * Testning


I detta kapitel beskrivs den tänkta tillvägagången till hur denna
handledning tillverkast.

\subsection{Litteraturstudier och Förundersökning (Temporär)}

% Detta ska ha underrubrikerna Förstudier och typer av undersökning

Projektdeltagarna skulle först återuppfriska kunskaperna om Haskell, TSS
och Reglerteknik. Därefter genomfördes vidare studier om DSL och pedagogik för
att kunna implementera formlerna i Haskell.

Efter förstudien så intervjuade vi de kursansvariga över deras tankar om vad
studenterna tyckte var svårt i kursen. Det ska även påbörjas en
enkätundersökning som vi tänker skicka ut till studenterna som går i årskurs
2 och 3 inom Datateknik.

% Ska litteratur tas med?

\subsection{Uppställning av handledning}

Handledningen är uppdelad i sex kapitel %planerat
där den första kapitlet ska vara en introduktion till koncepted DSL.
Därefter ska det finnas ett kapitel som kort förklarar om komplexa tal
med Eulers formel samt olika typer av signaler och deras egenskaper.

%Därefter kommer kapitel om LTI-system, Fourierserier och Fouriertransform
%samt Laplace- och Z-transform.

I varje kapitel ställde vi upp de definitioner och formler som tycktes
vara allmänt viktiga med tillhörande förklarande text. Den tillhörande
texten ska vara inspirerad av boken ``Learn You a Haskell for Great
Good'' av Miran Lipovača \cite{learnyouahaskell} och texter från sidan
\url{http://betterexplained.com/}.

Det vill säga, när vi förklarar hur formlerna fungerar så utnyttjar vi
oss av exempel som inte nödvändigtvis är helt korrekta. Målet är
att den ovanstående bilden ger en god uppfattning på hur formlerna kan användas.
Sedan förklarar vi hur det kan appliceras inom TSS i en mer matematiskt korrekt
form. Därefter tänker vi visa hur dessa definitioner eller formler skulle
se ut i vårt DSL.

Dessutom ska det finnas exempeluppgifter till de områden studenterna finner
svåra och formulera lite uppgifter som ska kunna lösas via programmering.
% Lösningarna som finns med ska visa hur uppgiften kan lösas både med den
% vanliga analytiska metoden och via användandet av DSL.
% -Inte något vi har gjort.

\subsection{Implementation i DSL}
%Förklara hur komplexa tal och alla typer av transformer implementerades i DSL. 
Det DSL som använts i detta läromaterial har utvecklats i Haskell och har
designats för att kunna täcka de enligt gruppen viktigaste delarna som
behandlas i TSS.
Exempel på detta är komplexa tal som används flitigt när man arbetar med
signaler och system och därför implementerades också komplexa tal som första
momentet i vårt DSL. Komplexa tal är också ett förkunskapskrav till kursen TSS
och därför lämpade sig det också bra som ett första avsnitt för att
introducera läsarna till det DSL som använts och hur vi skrivit vårt
läromaterial.

Själva utvecklingen av det DSL som använts har skett genom att gruppen först
utvecklat all kod som behövts för ett avsnitt och sedan när koden är testad och
funkar tillfredsställande har den delats upp i kodpaket. I dessa kodpaket
har vissa funktioner tagits bort för att lämnas som övningar till läsarna att
implementera.

Exempelvis om man undersöker hur komplexa tal implementerades så börjar man med
hur ett komplext tal representeras, i vårt fall som två flyttalsvärden.
Ett värde för den reella delen av talet och ett för den imaginära.
Så ett komplext tal \(z = 1 + 1j = (1,1) \) representeras i vår DSL som:

\begin{minted}{haskell}
Complex 1 1
\end{minted}

Efter det så undersöktes vilka funktioner som borde implementeras för
att vår tolkning komplexa talen skulle bete sig som matematikens komplexa tal.

Addition kunde implementeras relativt enkelt då de reella delarna adderas med
varandra och de imaginära med de imaginära enligt formeln nedan:
\[(a, b) + (c, d) = (a + c, b + d)\] \cite{conway1978functions}
Vilket i Haskell implementerades så här:
\begin{minted}{haskell}
z + w = Complex (realPart z + realPart w) (imPart z + imPart w)
\end{minted}

I kodexemplen ovan används funktioner som plockar ut den reella
(\mintinline{haskell}{realPart}) respektive den imaginära delen
(\mintinline{haskell}{imPart}) ur ett komplext tal.

Gemensamt för hela det DSL som använts i läromaterialet är att så beskrivande
namn som möjligt är givna till funktioner och typer.
Detta gjordes för att funktionerna ska kunna läsas och förstås av inte bara
studenter med djupare intresse av Haskell utan även av dem som enbart besitter
grundläggande kunskaper för programmeringsspråket.

Med multiplikationen så var det lite mer komplicerat då formeln ser ut så här:
 \[(a, b) \cdot (c, d) = (ac - bd, ad + bc) \] \cite{conway1978functions}

\begin{minted}{haskell}
z * w = Complex (realZ*realW - imZ*imW) (realZ*imW + realW*imZ)
  where realZ = realPart z
        realW = realPart w
        imZ   = imPart z
        imW   = imPart w
\end{minted}

%Är detta nödvändigt? Vi borde liksom inte vara för förklarande.
Här skapades hjälpfunktionerna (som man finner efter \mintinline{haskell}{where}
i koden) för att undvika att man skrev samma funktion om och om igen,
samtidigt som de håller koden tydlig och lättläst.

Efter att dessa operationer var implementerade så var division nästa ``stora''
operation, och formeln för det ser ut som följer:
\[ z / w = (z \cdot w’) / (w \cdot w’) \]
%(sommarmatte.se)  Kan lämnas som reference.
där w’ är w:s konjugat.
%Man förlänger både täljare och nämnare med nämnarens konjugat vilket innebär
%att nämnaren blir helt reell.
För att implementera division var man tvungen att först göra så man kunde ta
konjugatet på ett komplext tal.
Det innebär att man byter tecknet på den imaginära delen av talet.
Vår funktion för det såg ut så här:
\begin{minted}{haskell}
conjugate z = Complex (realPart z) (negate (imPart z))
\end{minted}
Vår implementation av division blev då som det står här under:
\begin{minted}{haskell}
z / w = Complex (realPart zw’ / realPart ww’) (imPart zw’ / realPart ww’)
  where zw’ = z * (conjugate w)
        ww’ = w * (conjugate w)
\end{minted}

En viktig sak att implementera för att kunna koppla de komplexa talen till
kursen TSS är att man kan använda sig av Eulers formel och kan skapa komplexa
tal med hjälp av en vinkel. Eulers formel ser ut så här \cite{trott2004}:
\[e^{jz}=\cos z+ j \cdot \sin z \]
%(encyclopediaofmath) -kan lämnas som reference
Implementationen av den här representationen skedde i flera steg.
Funktionen \mintinline{haskell}{euler} skapar ett komplext tal utifrån en
given vinkel, representerad av ett flyttal.
\begin{minted}{haskell}
euler phi = Complex (cos phi) (sin phi)
\end{minted}
Enligt potenslagarna%(formelsamlingen) - lämnas till reference?
så är \(e^{a+jb} = e^{a} \cdot e^{jb}\) och enligt Eulers formel så är
\(e^{j b} = cos b + j\dot sin b\) så då är exponentialfunktionen för ett
komplext talimplementerad enligt nedanstående kod:
\begin{minted}{haskell}
exp z = scale (exp (realPart z)) euler (imPart z))
\end{minted}
Sedan implementerades även trigonometriska funktioner med hjälp av Eulers
formel. Så här representeras sinus som ett komplext tal:
\[ sin(x) = (e^{j x} - e^{-j x}) / 2 j \]
Vilket ser ut så här i koden:
\begin{minted}{haskell}
sin z = (exp (j*z) - exp (-(j*z))) / (scale 2 j )
  where j = Complex 0 1
\end{minted}

\subsection{Verktyg}%behövs denna?
% Kanske skriva om hur vi använder google drive?
% Jag skulle säga att det bara är värt att ta upp verktyg som påverkar
% resultatet, t.ex. att vi
% skrivit koden i haskell, men att vi t.ex. arbetat fram texten i google
% drive känns inte så relevant att ha med här.
Gruppen ska använda sig av Google Documents för att skriva den första utkasten
av handledningen. Efter faktan och strukturen för varje kapitel blev
färdigställd så överfördes det till \url{https://www.sharelatex.com/},
en webbsida som erbjuder en gemensam latex kompilator.
Tillslut så ska den överföras till en HTML format som är slutprodukten.

Matlab och \url{fooplot.com/} används till att skapa figurer på funktions
kurvor eller vektor värden ifall det behövs till handledningen.
För enkätundersökningen så används webbsidan \url{https://www.webbenkater.com/}.

Koden ska vara skriven i programmeringspråket Haskell och gruppen använde sig
av \url{https://c9.io/} för att alla ska kunna ha tillgång till koden och en
kompilator.
%TODO: "all någorlunda färdiga material" := "allt någorlunda färdigt material"
Pågående under arbetets gång så laddas allt någorlunda färdigt material upp i \url{https://github.com/}. Här kan gruppens handledare kolla igenom materialet och ge sina kommentarer. Dessutom så blir materialet tillgänglig för andra att kolla igenom och bygga vidare på. 
%TODO: korrekt, men bara en lite del av motiveringen för github: materialet är även tillgängligt för andra (studentmålgruppen exempelvis) både för läsning och för att bygga vidare. Dessutom är git, och github, verktyg som ingår i många projekt i arbetslivet, dvs. git(hub)-kompetens är en del av de generella ingenjörskompetenserna. etc.

%mer?

\subsection{Didaktik}
Läromaterialet ska vara uppbyggt så att studenten själv får arbeta med det utifrån sina tidigare 
kunskaper. Vi ville åstadkomma ett aktivt lärande som ger studenten långvarig förståelse och 
intresse för ämnet, snarare än kunskap studenten lär sig tillfälligt utantill. \textit{Didaktik för 
ingenjörslärare} beskriver aktivt lärande enligt följande: ”Idéen bakom aktivt lärande är att en 
lärprocess bygger på en konkret erfarenhet som följs av reflektion och observation.” Med bakgrund av 
detta har vi lagt en stor del av vårt fokus på att göra läromaterialet interaktivt. Vi har lagt in 
enklare kryssfrågor i texten för att låta studenten arbeta aktivt med ämnet när de läser teorin och 
vi har även lagt in programmeringsövningar där studenten får använda och utveckla sina kunskaper i 
ämnet.

Eftersom läromaterialet som tagits fram är ett komplement till en redan existerande kurs har gruppen 
bedömt det som extra viktigt att materialet är motiverande eftersom studenten läser det vid sidan av 
sina vanliga studier. Därför har gruppen valt att använda sig av ARCS-modellen som beskrivs i avsnitt 
Teknisk bakgrund - Didaktik. 

ARCS-modellens första grundpelare, \textit{Attention}, har vi praktiserat bland annat genom att använda 
skämtsamma formuleringar, sammanfattningar och exempel inom ämnet. Vi har strävat efter att hålla en 
lättsam och humoristisk ton för att hålla strudenternas uppmärksamhet. Vi har dock försökt undvika 
lustiga utvikningar, formuleringar och exempel som avviker från ämnet eller inte fyller sin funktion 
att främja intresse och förståelse för ämnet. För att sedan väcka ett mer varaktigt intresse hos 
studenten ställer vi frågor i löpande text och övningar. Varje avsnitt inleds med en fråga av typen 
“Vad är det här egentligen för något och vad används det till?”. För att behålla studentens intresse 
genom hela läromaterialet har vi valt att variera upplägget på avsnitten till viss del. Eftersom vi 
vill behålla en läsvänlig och överskådlig struktur i hela läromaterialet har vi valt att främst variera 
materialet med olika typer av exempel och uppgifter. I ett kapitel får studenterna fylla i ett relativt 
stort antal kryssfrågor och i ett annat lägger vi stort fokus på en viss typ av programmeringsövningar 
där studenten får fylla i saknad kod. Vi har alltså siktat på att göra studenternas eget arbete 
omväxlande för att att de inte ska bli uttråkade och ge upp.

Den andra grundpelaren, \textit{Relevance}, har vi praktiserat främst i inledningen av avsnitten och i 
introduktionsavsnittet. Introduktionsavsnittet förklarar att läromaterialet finns till för att vara ett 
komplement till, och hjälpa datastudenterna med, \textit{TSS}. Varje ämnesavsnitt inleds sedan med en kort 
beskrivning av ämnet och vad det kan användas till. På detta sätt förklarar vi inledningsvis varför 
läromaterialet är relevant för studentens studier. Sedan förklarar vi mer i detalj vad de olika 
ämnesområderna kan användas till alteftersom studenten stöter på dem i materialet.

Den tredje grundpelaren, \textit{Confidence}, har vi praktiserat genom att skapa övningar och uppgifter 
av varierande svårighetsgrad och ett upplägg där studenten själv kontrollerar sin inlärning. Vad gäller 
övningarna så inleder vi i det första avsnittet med enklare kryssfrågor och relativt enkla 
programmeringsövningar. Kryssfrågorna kan kryssas i direkt på hemsidan, varpå de rättas direkt för att 
ge studenten omedelbar bekräftelse. Det finns även lösningsförslag till programmeringsövningarna som 
studenten kan jämföra sin kod med eller titta på om hen kör fast. Eftersom vi vill att studenten själv 
ska kunna kontrollera sin inlärning och arbeta i en takt som fungerar för hen passar det utmärkt att 
lägga upp läromaterialet i form av en hemsida. På så vis har studenten tillgång till all information och 
alla uppgifter men kan arbeta med materialet som hen vill.

Den fjärde och sista grundpelaren, \textit{Satisfaction}, har vi praktiserat främst i slutet på 
avsnitten och i uppgifterna. I våra kryssfrågor får studenten omedelbar bekräftelse när hen har lärt sig 
något eller kan det sedan innan. Efter att studenten har läst teorin inom ett område följer vi upp med 
övningar där studenten får praktisera sin nyvunna kunskap. Varje avsnitt i läromaterialet avslutas sedan 
med en lista på vad studenten nu har lärt sig. Studenten får alltså både ett erkännande av sin kunskap 
och förmåga, samt möjlighet att praktisera detta. 

\subsection{Test av handledning}

Vi tog kontakt med studenter som kunde tänka sig att delta i testandet genom att skicka ut en enkät där vi frågade om vad studenterna tyckte var lätt och svårt med kurserna, vad de själva ansåg att de hade förstått och inte förstått. 
De som sedan ville deltaga i testandet av vår produkt kunde lämna en mailadress som vi sedan mailade ut de första delarna av vår handledning till, när vi ansåg att handledningen var redo för att testas. 
Detta resulterade i en testgrupp på ungefär 20 personer. 
Testningen fungerade så som att man som testperson fick ta del av materialet, och då läsa våra texter och göra de tillhörande uppgifterna. Testgruppen fick även tillgång till programkoden som var skriven till uppgifterna, och även facitkod.
När studenterna sedan har testat produkten och gått igenom vad de tycker är bra och vad de tyckte är dåligt så bearbetar vi deras feedback, jämför och ser ifall vi hittar några övergripande fel eller delar som vi tycker att vi borde fokusera på att förbättra och sedan implementerar vi förbättringarna så att handledningen på det stora hela blir mer pedagogisk. 



\section{Resultat}

% Huvudpoänger:
% *Vad blev det för produkt? Tutorial på hemsida med 6 delavsnitt?
% *Hänvisa till hemsidan
% *Vad tyckte testgruppen?
%   *Citat från testgruppen

%
% Illustrationer:
% * Utdrag ur tutorialen

\subsection{Hantering av TSS}

Gruppen började med att kolla igenom kursernas kursplan som ställde upp lärandemålen. 
Från lärandemålen och resultaten från intervjun och enkätundersökningarna blir det 
tänkt att sluta sig till de områden som handledningen är tänkt att täcka. %Vi kollade även igenom tentorna över de senaste åren för att se vilka begrepp och områden som brukade komma för att höja säkerheten.

Efter områdena har blivit bestämda påbörjades skrivandet av texter och koden. %Dessa ansvarsområden delades inom gruppen.
Vi försökte att gå direkt till poängen vid beskrivandet av definitioner till 
olika operator och deras applikation. Därefter följs det av en väldigt informell 
text vars mål är att måla en ungefärlig bild på hur operatorn fungerar. 
Allt avslutades med exempel på hur operatorn kan appliceras och uppgifter där 
det är tänkt att lösas med programmering 

Ett alternativ metod är att presentera definitionen till det man vill förklara strax 
följt av beviset till definitionen samt exemplar. Det vill säga, en formell ansats med tyngd 
på matematisk korrekthet. För somliga studenter inom matematik så ger beviset en djupare förståelse 
för operatorn och därmed förstår de varför den fungerar och hur den kan appliceras. Denna metod 
kommer dock använda sig helt av den matematiska domän specifika språket och inte av programmering 
vilket gör det i teorin svårläsligt till målgruppen handledningen riktar sig mot.

Eftersom handledningen är tänkt att vara lätt tillgänglig och uttryckt i ett språk 
som \emph{studenterna} förstår så valde gruppen den informella ansatsen till förklarandet av ämnet. 
Detta har även med hur gruppen tyckte att \cite{learnyouahaskell} och texter från sidan
\url{http://betterexplained.com/} var väldigt pedagogiska.  

\section{Resultat}

% Huvudpoänger:
% *Vad blev det för produkt? Tutorial på hemsida med 6 delavsnitt?
% *Hänvisa till hemsidan
% *Vad tyckte testgruppen?
%   *Citat från testgruppen

%
% Illustrationer:
% * Utdrag ur tutorialen


\section{Diskussion}

% Huvudpoänger:
% * Har vi löst problemet?
% * Går problemet att lösa?
% * Vad är i grund och botten orsaken till problemet?
% * Vilka problem har dykt upp?
% * Vilka problem kan man undvika om man gör ett liknande arbete?
% * Stämmer vårt resultat överens med tidigare forskning?
% * Hur kan man bygga vidare på detta?


%Problem med hur vi implementerar integraler och annat kontinuerliga fall

Från intervjun förklarade examinatorerna att den största svårigheten studenterna
fann i kursen var kopplingen mellan matematiken och den bakomliggande fysiken.
% Hur tacklade vi denna problem?

% Ett problem var själva storleken på uppgiften gruppen tog sig an.
% För att kunna skriva en bra handledning så krävs det goda
% förkunskaper om både ämnet och pedagogik.

% Ta även upp varför reglerteknik inte togs med i projektet.

\section{Slutsatser}
% Huvudpoänger:
% * Sammanfattar vad vi kommit fram till i diskussion och resultat

\section{Avslutning}
% Huvudpoänger:
% * Vad är gjort och vad återstår att göra?
%
% Illustration: Borde inte behövas

\newpage

\bibliographystyle{IEEEtran}
\bibliography{referenser}

\newpage

\section{Bilagor}
% * Utdrag ur tutorial?
% * Utdrag ur intervju med examinatorer?
% * Utdrag ur enkätsvar?

\end{document}
