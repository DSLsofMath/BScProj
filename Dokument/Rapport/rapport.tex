\documentclass[12pt,a4paper,twoside,openright]{article}

% CREATED BY DAVID FRISK, 2015

% BASIC SETTINGS
\usepackage{textcomp}       % Fonts, symbols etc.
\usepackage{lmodern}        % Latin modern font
\usepackage{helvet}         % Enables font switching
\usepackage[T1]{fontenc}    % Output settings
\usepackage[swedish,english]{babel} % Language settings
\usepackage[utf8]{inputenc} % Input settings
\usepackage{amsmath}        % Mathematical expressions (American mathematical society)
\usepackage{amssymb}        % Mathematical symbols (American mathematical society)
\usepackage{graphicx}       % Figures
\usepackage{minted}         % Enables source code listings
\usepackage[top=3cm, bottom=3cm, inner=3cm,
  outer=3cm]{geometry}      % Page margin lengths
\usepackage{eso-pic}        % Create cover page background
\newcommand{\backgroundpic}[3]{ \put(#1,#2){
    \parbox[b][\paperheight]{\paperwidth}{
      \centering
      \includegraphics[width=\paperwidth,height=\paperheight,keepaspectratio]{#3}}}}
\usepackage{float}          % Enables object position enforcement using [H]
\usepackage{parskip}        % Enables vertical spaces correctly
\usepackage{url}
\usepackage{glossaries}


% OPTIONAL SETTINGS (DELETE OR COMMENT TO SUPRESS)

% Disable automatic indentation (equal to using \noindent)
\setlength{\parindent}{0cm}


% Caption settings (aligned left with bold name)
\usepackage[labelfont=bf, textfont=normal,
  justification=justified, singlelinecheck=false]{caption}


% Activate clickable links in table of contents
\usepackage{hyperref} \hypersetup{colorlinks, citecolor=black,
  filecolor=black, linkcolor=black, urlcolor=black}


% Define the number of section levels to be included in the t.o.c. and
% numbered (3 is default)
\setcounter{tocdepth}{5}
\setcounter{secnumdepth}{5}


% Chapter title settings
\usepackage{titlesec}
\titleformat{\chapter}[display]
  {\Huge\bfseries\filcenter}
  {{\fontsize{50pt}{1em}\vspace{-4.2ex}\selectfont
      \textnormal{\thesection}}}{1ex}{}[]


% Header and footer settings (Select TWOSIDE or ONESIDE layout below)
\usepackage{fancyhdr}
\pagestyle{fancy}


% Select one-sided (1) or two-sided (2) page numbering
\def\layout{2}	% Choose 1 for one-sided or 2 for two-sided layout
% Conditional expression based on the layout choice
\ifnum\layout=2	% Two-sided
    \fancyhf{}
	\fancyhead[LE,RO]{\nouppercase{ \leftmark}}
	\fancyfoot[LE,RO]{\thepage}
	\fancypagestyle{plain}{			% Redefine the plain page style
	\fancyhf{}
	\renewcommand{\headrulewidth}{0pt}
	\fancyfoot[LE,RO]{\thepage}}
\else			% One-sided
  	\fancyhf{}
	\fancyhead[C]{\nouppercase{ \leftmark}}
	\fancyfoot[C]{\thepage}
\fi


% Enable To-do notes
\usepackage[textsize=tiny]{todonotes}   % Include the option "disable" to hide all notes
\setlength{\marginparwidth}{2.5cm}


% Supress warning from Texmaker about headheight
\setlength{\headheight}{15pt}


\loadglsentries{include/ordlista.tex}

\date{\today}

%Ordlistan skapas här.
%Måste refereras till något med \gls{ref} för att kunna visas.
\makeglossaries

\begin{document}
\pagenumbering{roman}

% CREATED BY DAVID FRISK, 2015

% COVER PAGE
\begin{titlepage}
\newgeometry{top=3cm, bottom=2.7cm,
  left=2.25 cm, right=2.25cm}	% Temporarily change margins

% Cover page background
\AddToShipoutPicture*{\backgroundpic{-4}{56.7}{figure/auxiliary/frontpage_swe.pdf}}
\addtolength{\voffset}{2cm}

% Cover picture (replace with your own or delete)
% \begin{figure}[H]
% \centering
% \vspace{2cm}	% Adjust vertical spacing here
% \includegraphics[width=0.9\linewidth]{figure/Wind.png}
% \end{figure}

% Cover text
\mbox{}
\vfill
\renewcommand{\familydefault}{\sfdefault} \normalfont % Set cover page font
\textbf{{\Huge Programmering som\\ undervisningsverktyg för Transformer,\\signaler och system\\[0.2cm]}}
{\Large Utvecklingen av läromaterialet \textit{TSS med DSL} }\\[0.5cm]
Kandidatarbete inom Data- och Informationsteknik \setlength{\parskip}{1cm}

\begin{flushleft}
  \Large
  Filip Lindahl\\
  Cecilia Rosvall\\
  Peter Ngo\\
  Jacob Jonsson\\
  Joakim Olsson\\
\end{flushleft}


%{\large Filip Lindah, Cecilia Rosvall, Peter Ngo, Jacob Jonsson, Joakim Olsson}
\setlength{\parskip}{2.9cm}

\textsc{Chalmers tekniska högskola} \\
\textsc{Göteborgs universitet} \\
Institutionen för Data- och Informationsteknik \\
Göteborg, Sverige, Juni 2016

\renewcommand{\familydefault}{\rmdefault} \normalfont % Reset standard font
\end{titlepage}

% IMPRINT PAGE (BACK OF TITLE PAGE)
\newpage
\thispagestyle{plain}

The Author grants to Chalmers University of Technology and University
of Gothenburg the nonexclusive right to publish the Work
electronically and in a non-commercial purpose make it accessible on
the Internet. \\
The Author warrants that he/she is the author to the
Work, and warrants that the Work does not contain text, pictures or
other material that violates copyright law.

The Author shall, when transferring the rights of the Work to a third
party (for example a publisher or a company), acknowledge the third
party about this agreement. If the Author has signed a copyright
agreement with a third party regarding the Work, the Author warrants
hereby that he/she has obtained any necessary permission from this
third party to let Chalmers University of Technology and University of
Gothenburg store the Work electronically and make it accessible on the
Internet.

\vspace{1.5cm}

\textbf{Programmering som undervisningsverktyg för Transformer, \\ signaler och system} \\
- Utvecklingen av läromaterialet \textit{TSS med DSL}


\vspace{1cm}

Filip Lindahl   \\
Cecilia Rosvall \\
Peter Ngo       \\
Jacob Jonsson   \\
Joakim Olsson

\copyright ~ Filip Lindahl, 2016  \\
\copyright ~ Cecilia Rosvall, 2016\\
\copyright ~ Peter Ngo, 2016      \\
\copyright ~ Jacob Jonsson, 2016  \\
\copyright ~ Joakim Olsson, 2016
\vspace{0.5cm}

Supervisor: Patrik Jansson, Department of Computer Science and Engineering \\
Examiner: Niklas Broberg, Department of Computer Science and Engineering
\vspace{0.5cm}

Chalmers University of Technology\\
University of Gothenburg \\
Department of Computer Science and Engineering \\
DATX02-16-05 \\
SE-412 96 Gothenburg\\
Telephone +46 31 772 1000 \setlength{\parskip}{0.5cm}

\vfill
% Caption for cover page figure if used, possibly with reference to further information in the report
Typeset in \LaTeX \\
Gothenburg, Sweden 2016


\setlength{\parskip}{2mm}
\setlength{\parindent}{0pt}

\section*{Förord}
Denna rapport behandlar kandidatarbetet ``Matematikens domänspecifika språk
(DSLsofMath) för andra kurser'' som genomfördes på Chalmers tekniska högskola
 under vårterminen 2016. Vi som har gjort detta kandidatarbete är fem studenter
från civilingenjörsprogrammen Datateknik och Teknisk Matematik på Chalmers.

Vi vill tacka Patrik Jansson, vår handledare, som har varit väldigt hjälpsam
och skickat oss i rätt riktning när vi varit vilsna. Vi vill tacka Adam Sandberg Eriksson
som har svarat på frågor vi haft och funnits tillgänglig när vi undrat saker. Vi
vill tacka Bo Egardt och Ants Silberberg som tagit sig tid att diskutera sina
kurser med oss. Och slutligen vill vi tacka vår testgrupp som gått igenom vårt
material och gett oss återkoppling för att förbättra vår produkt.

\newpage

\thispagestyle{plain}

\section*{Sammandrag}
  %Ett Sammandrag
%Subject to Change!--------------------------------------------
% Huvudpoänger:
% * Vad handlar rapporten om?
%     Vad är det för projekt vi beskriver utvecklingen av?
%     (Läsare ska här kunna bedöma om rapporten är intressant att läsa för dem)
% * Jättekort om syfte, resultat etc
% Illustration: Borde inte behövas
Denna rapport beskriver utvecklingen av läromaterialet ``TSS med DSL'' med dess
tillhörande programmeringskod och lösningar. Läromaterialet riktar sig främst
till studenter på det datatekniska programmet på Chalmers tekniska högskola och har som syfte att
vara ett kompletterande läromedel inom signallära där man utnyttjar studenternas
kunskaper inom programmering. Materialet innehåller förklarande text och teori
om bland annat komplexa tal, signaler, system och transformer. I materialet
ingår även programmeringsövningar som är skrivna med domänspecifika språk i
programmeringsspråket \gls{Haskell}.

Bakgrunden till detta projekt är att kursen inom signallära, TSS, på det
aktuella programmet under många år har haft väldigt dåliga studieresultat på
grund av studenternas ovana att hantera den typ av matematik som utgör
grundstommen i kursen. Detta har visats sig inte vara något som är unikt
för datastudenterna på Chalmers utan det är tvärtom ett problem inom många
discipliner.

Rapporten beskriver vidare de undersökningar och efterforskningar som gjorts
för att ta kunna ta fram läromaterialet, samt de tester som gjorts på den
färdiga produkten.

Materialet som utvecklats kan ses på \url{https://github.com/DSLsofMath/BScProj}.

\newpage

\thispagestyle{plain}

\section*{Abstract}
\selectlanguage{english}
This report describes the development of the tutorial ``TSS med DSL'' with
associated programming code and solutions. The tutorial is aimed at students
studying Computer Science at Chalmers University of Technology and the aim of
the tutorial is to be a supplementary learning material for the course
``Transforms, Signals and Systems''. The tutorial contains explanations and
theory concerning complex numbers, signals, systems and transforms. The
tutorial also includes programming exercises using domain specific languages
written in Haskell.

The background to this project is that the computer science students studying
the course ``Transforms, Signals and Systems'' have gotten poor results for
several years due to the students' unfamiliarity with the kind of mathematics
which is necessary for that course. As it turned out this is a general problem
and not unique to Computer science students at Chalmers.

The report further describes the surveys and analyses done in order to produce
this tutorial and the test results from the finished product

The developed material can be viewed at \url{https://github.com/DSLsofMath/BScProj}.

\newpage

\selectlanguage{swedish}
\tableofcontents

\newpage

\printglossary[title=Ordlista,nonumberlist]

\newpage

\setcounter{page}{1}
\pagenumbering{arabic}			% Arabic numbering starting from 1 (one)
\setlength{\parskip}{2mm plus2mm}

\section{Inledning}
Vi inleder med att gå igenom bakgrunden till projektet och hur det
uppstod, samt beskriva projektets mål och rapportens syfte.

\subsection{Bakgrund}
% Huvudpoänger:
% * Problem med kurserna på data
% * Var kommer vårt projekt ifrån? Spinoff till DSLsofMathprojektet.
% * Ev. ta upp i korthet att detta är en del av ett större problem.
%     Hänvisa till TFPIE-artikelns källor.
%
% Illustrationer: Borde inte behövas

Det finns flera som menar att det matematiska språket används på ett för
otydligt sätt i undervisningssammanhang och att man antar att studenterna
är betydligt mer vana vid och bevandrade i matematiken än de oftast är
\cite{sussman2002role} \cite{wells1995communicating}.
Detta gör att vissa grundläggande detaljer som då antas vara allmän kunskap
utelämnas och de studenter som inte är tillräckligt vana vid matematik får inte
hela bilden utav förklaringen.
Det huvudsakliga problemet är alltså inte språket i sig utan att det ofta
används på fel detaljnivå. Hur ska man då veta på vilken nivå man bör lägga
sina förklaringar? En lösning som Sussman och Wisdom föreslår är att
försöka förklara det för en dator, om datorn förstår koncepten så är det
tillräckligt detaljerat.

Ett praktiskt exempel på detta problem går att finna på Chalmers tekniska högskola.
Där finns det en lång trend där studenterna på utbildningen för Datateknik har
haft svårigheter med kurserna \textit{Transformer, Signaler och System}, hädanefter
omnämnd som ``\gls{TSS}'', och \textit{Reglerteknik}.
Svårigheterna med dessa kurser tror examinatorerna till stor del beror på att
studenterna inte är tillräckligt bekväma i det matematiska språket.
%Detta syns bland annat i statistiken av hur många datastudenter som klarar kurserna.

Inspirerat av bland andra Sussman, Wisdom och Wells artiklar påbörjades ett
pedagogiskt projekt på Chalmers, Domain Specific Languages of Mathematics (\gls{DSLsofMath}), vars syfte var att minska
svårigheterna som studenterna hade med kursena TSS och Reglerteknik.
Projektet har resulterat hittills i en kurs som heter
\textit{Matematikens domänspecifika språk} och har körts i ett år och nu
är i utvärderingsfasen. I kursen får studenterna angripa
matematiska problem och försöka lösa dem med ett verktyg de är
mer familjära med än matematik, nämligen det funktionella
programmerings\-språket Haskell. Förutom det faktum att studenterna tidigare
kommit i kontakt med Haskell har språket dessutom en notation med fokus på
tydlighet som lämpar sig väl för att lära ut matematiska begrepp
\cite{TFPIE15_DSLsofMath_IonescuJansson}.

Detta kandidatprojekt uppstod som en avgrening ur DSLsofMath-projektet med fokus
på att utveckla material till specifika kurser snarare än en allmän matematisk grund.

\subsection{Rapportens syfte}

% Huvudpoänger:
% *Vad vill vi säga med rapporten? Vad vill vi visa på?
% *Vad har vi försökt göra i korthet?

% Illustration:
Syftet med denna rapport är att beskriva utvecklingen av läromaterialet
``\gls{TSSmDSL}'', som är ett kompletterande läromaterial till kursen TSS
för studenter på datatekniska programmet på Chalmers, samt beskriva hur det
bakomliggande problemet kan kopplas till en mer generell problematik.

\subsection{Projektets mål}
% Huvudpoänger:
% *Beskriv projektets syfte och mål, vad ska vi åstadkomma för produkt?
Syftet med projektet är att underlätta för studenter inom Datateknik att
ta till sig signal- och systemteoretiska ämnen genom att utnyttja deras
kunskaper inom programmering och även underlätta för studenterna att betrakta
ämnet ur en programmerares perspektiv. Tanken bakom detta är att man ska göra
gapet mellan Datateknik och signalteori mindre och minska problemen som nämns
i bakgrundsavsnittet.

Projektet är tänkt att resultera i ett läromaterial som kan fungera som ett
komplement till den kurs som ges inom signalteori på den datatekniska
grund\-utbildningen. Detta läromaterial ska innehålla förklaringar och
programmeringsövningar som är anpassade för studenter på den datatekniska
utbildningen på Chalmers tekniska högskola.

\section{Avgränsningar}
% Huvudpoänger:
% * Vad behandlar vi inte i rapporten och varför?
% * Vad tas inte med i vårt projekt och varför?
% Illustrationer: Borde inte behövas

På grund av tidsbrist i projektet har gruppen avgränsat sig till att ta fram
läromaterial till endast en kurs, \textit{TSS}. Läromaterialet som tagits fram är ett
kompletterande material och är inte tänkt att kunna ersätta kursen eller
dess material.

%TODO: kolla upp avstavningarna - kanske någon LaTeX-inställning är fel? Det borde inte bli "matem-atisk", "tex-tens", etc.
Gruppen har under utvecklingen av materialet inte heller lagt fokus på
ett matematiskt korrekt språk, matematiska bevis, avancerade matematiska
begrepp eller teorier. Anledningen till detta är att materialet i
första hand är tänkt att främja förståelse och väcka intresse för
ämnet. Detta beslut baserades även på boken \textit{Didaktik för
 ingenjörslärare} där författarna, under en diskussion om aktivt
lärande, hävdar att: ”När man lär sig något kan man få det att fastna
på olika sätt \cite{didaktik_for_ingenjorslarare}. En ytligt inlärd kunskap eller ett beteende som inte
tränats tillräckligt glöms snart bort. Det är med andra ord viktigt
att metoderna tillämpas så att djuplärande uppnås. Textens detaljer är
då inte lika intressanta som textens budskap, förståelse, sammanhang
och principer. Snarare än att lära sig 'det man måste' utantill för
reproduktion, sker ett personligt tillägnande relaterat till tidigare
kunskap.” I detta projekt är det alltså viktigt att ta fram ett
interaktivt material som främjar förståelse och inte fastna i
detaljer.

\section{Problemanalys}
%Huvudpoänger:
% * Vad är det för problem projektet ska lösa?
% * Varför har vi har vi valt att lösa problemet så här?
% * Knyter vårt problem an till ett större problem?
% * Ev. hänvisa kort till relaterad forskning

%Illustrationer: Statisktik från kurserna, pajdiagram

%Problemanalysen skall leda fram till produktspecen
% TODO: Här hade det vart najs om vi kunde knyta an till bakgrunden om att
% anledningen till att vi har svårt för TSS och Reglerteknik kan vara att den
% matematiska notationen är lite otydlig och att med programmeringsperspektiv
% så blir det förhoppningsvis tydligare.
% TODO: Knyta an till Relaterad forskning.
Som nämndes i bakgrunden är problemet vi tacklar i vårt projekt att
datastudenterna på Chalmers har svårt för två av sina kurser, TSS och
Reglerteknik.

Statistiken från Chalmers visar tydligt att kurserna TSS och
Reglerteknik har en ovanligt hög andel studenter som blir
underkända på tentamen. Under perioden 2010 till 2016 blev i genomsnitt
endast 51\% av de som skrev tentamen i TSS godkända och under samma
period blev 53\% godkända i Reglerteknik \cite{tentastatistik}.
%Pajdiagram
Det finns även ett mörkertal i dessa siffror eftersom alla studenter
inte alltid skriver tentamen, exempelvis valde 48 av 122 registrerade
studenter att inte tentera Reglerteknik 2014
\cite{kursinformation:ere102:14-15}.

Kurserna i fråga, TSS och Reglerteknik, läses under tredje året på
%TODO: Undvik att bland "Datateknik" och "datateknik". Kanske ni helt enkelt kan introducera "D" som förkortning och sedan köra med "D" hela vägen?
Datateknik och kräver bland annat förkunskaper från de kurser i
matematik studenterna läser under sitt första år. Reglerteknik kräver
även förkunskaper från TSS. Det kan därför anses vara ett rimligt
antagande att bristande kunskaper i TSS orsakar följdproblem och utgör
en del av studenternas problem med Reglerteknik.

% TODO: Det här känns som en upprepning av bakgrunden (ish).
Kursernas examinatorer tror att studenternas svårigheter med kurserna
till stor del beror på att studenterna inte är tillräckligt bekväma i
det matematiska språket. Studenterna på Datateknik har inte
tillräcklig vana att tolka innebörden av de matematiska
beskrivningarna och är allmänt ovana vid matematiskt hantverk. Därför
lägger datastudenterna det mesta av sin energi på att få ihop
matematiken rent praktiskt, istället för att se vad matematiken står
för.

\section{Produktspecifikation}
% Huvudpoänger:
% * Beskriv produktbeskrivningen vi fick från början
% * Bena ut i detalj hur produktbeskrivningen vi tagit fram ser ut och
%     kommentera vilka val vi har gjort.
% * Förklara hur produkten löser/bidrar till att lösa problemet

%Illustration: Ev bild på utdrag från tutorial för att tydliggöra upplägget

% TODO: Vad är målet att vår produkt ska lösa? Alltså vi borde skriva ut
% att vi vill lära ut tss genom programmering.
För att åtgärda det problem som beskrivs i förgående avsnitt har
gruppen skapat en unik produkt som är anpassad speciellt till TSS på
programmet för Datateknik på Chalmers. Den ursprungliga
produktspecifikationen som gruppen tilldelades vid arbetets start var
relativt vag. Gruppens uppgift var att ``ta fram DSLsofMath-inspirerat
kompletterande material för andra närliggande kurser''. Det önskade
implementeringsspråket var Haskell.

Gruppen beslutade snabbt att fokusera på kurserna TSS och Reglerteknik
av de anledningar som nämns i problembeskrivningen. Dock konstaterade
gruppen en bit in i projektet att det endast fanns tid att utveckla
material av önskvärd kvalitet till en kurs. Gruppen valde då kursen
TSS, med motiveringen att TSS innehåller förkunskaper som är
nödvändiga för Reglerteknik.

\subsection{Produktens struktur}
\label{sec:prodSpec}
För att göra läromaterialet så lättillgängligt som möjligt beslutades att
%TODO: kolla användningen av ordet "materiel" (här) jmf. med "material" (som ni skriver i övrigt)
all materiel skulle samlas på en hemsida. På detta sätt kan man enkelt
samla text, bilder och programmeringskod och presentera dem på ett
överskådligt sätt för studenterna i målgruppen.

Det beslöts att läromaterialet skulle innehålla ett antal delavsnitt
som tillsammans täckte in innehållet i kursen TSS i stora drag.
Avsikten med indelningen är att göra materialet mer överskådligt och
även göra det möjligt för studenterna att enkelt hitta aktuellt
material för ett specifikt område.

Varje delavsnitt ska innehålla förklaringar och exempel i löpande text.
Denna text ska skrivas på ett sådant sätt att den är lätt att ta till
sig för målgruppen och uppmuntrar till vidare studier av ämnet.

Delavsnitten ska vidare innehålla exempel på hur det aktuella ämnet kan
beskrivas med funktionell programmering och DSL, samt programmeringsövningar.

\subsection{Programmeringskod och övningar}

Programmeringskoden och övningarna ska skrivas på ett sådant sätt att
det endast krävs grundläggande förkunskaper i funktionell
programmering och Haskell för att förstå dem. Studenten ska inte
behöva ha läst kursen DSLsofMath eller motsvarande för att förstå
övningarna, även om det självklart underlättar.

Koden ska i första hand vara lättläst och pedagogiskt skriven snarare
än att vara effektiv, eftersom koden endast är tänkt som ett läromedel
och inte ska användas för att till exempel fouriertransformera
signaler i andra sammanhang än i undervisningssyfte.

\section{Teknisk bakgrund}
I detta kapitel så beskriver vi kortfattat de områden och programvara som
behövs för att bygga upp läromaterialet.
% Huvudpoänger:
% * Ta upp och introducera tekniken vi bygger vår tutorial på.

\subsection{Git och GitHub}
Git är ett versionshanteringsprogram som gör det möjligt för flera
personer att arbeta med samma material samtidigt utan att de skriver
över varandras arbete. Så länge inte två versioner innehåller
ändringar som står i konflikt med varandra, exempelvis att man
samtidigt ändrat på samma rader, slås de ihop och bildar då en tredje
version vars innehåll kommer från båda de tidigare versionerna. Alla
tidigare versioner utav en fil sparas och kan återställas och man kan
också återställa enstaka förändringar utan att behöva återställa hela
filen. Skulle två versioner stå i konflikt med varandra krävs däremot
en del manuellt arbete.

För ytterligare information om Git se hemsidan \url{https://git-scm.com}.

GitHub är en hemsida som använder Git. GitHub gör det möjligt för
utvecklare av program med öppen källkod att utveckla kod och annat
material offentligt. Öppen källkod innebär att den kod som programmet
är uppbyggt utav är offentlig och tillgänglig för alla att läsa och
kontrollera.

GitHub förenklar kommunikationen mellan användare och utvecklare av program med
%TODO: jag föreslår att ni klipper bort "av program med öppen källkod" för att göra meningen mer läsbar
öppen källkod genom att tillhandahålla verktyg såsom möjligheter att kommentera
% TODO: "tillhandahålla verktyg såsom möjligheter att" är för indirekt
på enskilda versioner av källkoden, verktyg för uppföljning av ärenden
och problem samt dokumentation. GitHub har även verktyg för att
underlätta samarbete mellan utvecklare genom att man kan klona
varandras projekt och bidra med egna ändringar som huvudutvecklaren
kan godkänna. GitHub underlättar framtida vidareutveckling av kod,
eftersom koden lagras offentligt på hemsidan och därför är
lättåtkomlig för andra utvecklare.

För ytterligare information om GitHub se hemsidan \url{https://github.com}.
%TODO: kanske bra ställe att också länka till ert eget projekt på github som exempel.

\subsection{Funktionell programmering}
% -Referera till “Why functional programming matters”
Funktionell programmering är en programmeringsparadigm där de
huvudsakliga byggstenarna är funktioner. Som Hughes nämner i sin
välciterade \textit{Why Functional Programming Matters}
\cite{hughes1989functional}, handlar funktionell programmering mycket
om att göra det möjligt att skriva så modulär kod som möjligt.
Anledningen till detta är att det lättaste sättet att angripa ett
problem på är genom att bryta ner det i mindre delproblem, lösa
delproblemen och sen sammanfoga lösningarna till en helhetslösning.
Hur mycket man kan bryta ned ett problem beror då på hur enkelt man
sedan kan sammanfoga delproblemens lösningar igen när man är klar.

Vad som gör funktioner till grundläggande byggstenar i funktionella
programmeringsspråk är att de kan skickas som parametrar till andra
funktioner, så kallade högre ordningens funktioner. Detta gör att vi
kan bygga funktioner som exempelvis bara tar hand om datastrukturer
och skicka funktionen som löser delproblemet som en parameter.

Partiell funktionsapplikation är en annan egenskap som underlättar för
funktionella programmerare att skriva mer modulär kod. Partiell
funktionsapplikation innebär att funktioner som appliceras på färre
argument än de är definierade för skapar en ny funktion, se exemplet i
figur \ref{fig:hask_partfunapp} nedan.

\begin{figure}[H]
  \begin{minted}{haskell}
    ghci> :t (+)
    (+) :: Num a => a -> a -> a

    ghci> :t (+ 1)
    (+ 1) :: Num a => a -> a

    ghci> let plusEtt = (+1)

    ghci> plusEtt 2
    3
  \end{minted}
  \caption{Figuren visar hur Haskell hanterar partiellt applicerade
    funktioner. I bilden är \mintinline{haskell}{(+)} addition som är
    definierad på två numeriska värden och
    \mintinline{haskell}{plusEtt}, \mintinline{haskell}{(+ 1)}, är en
    funktion som givet ett annat numeriskt värde adderar 1 till det.}
  \label{fig:hask_partfunapp}
\end{figure}

Haskell är ett funktionellt programmeringsspråk med stöd för partiell
funktionsapplikation, högre ordningens funktioner, algebraiska
datatyper samt lat evaluering. Lat evaluering betyder att Haskell inte
evaluerar ett uttryck förrän det behövs och detta gör det möjligt för
Haskell att hantera oändliga datastrukturer utan att processen kräver
allt virtuellt minne och terminerar.

Algebraiska datatyper är typer som består av flera andra typer. Tack
vare Haskells mönstermatchning (\textit{pattern matching}) är det lätt
att definiera rekursiva funktioner som matchar på kontruerare i de
algebraiska datatyperna vilket visas i exemplet i figur
\ref{fig:hask_alg_type}.

\begin{figure}[H]
  \begin{minted}{haskell}
    data Tree a = Leaf | Node a (Tree a) (Tree a)

    mapTree :: (a -> b) -> Tree a -> Tree b
    mapTree _ Leaf = Leaf
    mapTree f (Node a l r) = Node (f a) (mapTree f l) (mapTree f r)
  \end{minted}
  \caption{Figuren visar den algebraiska datatypen
    \mintinline{haskell}{Tree a} som beskriver en trädstruktur och den
    generella funktionen \mintinline{haskell}{mapTree} som använder
    mönstermatchning för att manipulera trädstrukturen.}
  \label{fig:hask_alg_type}
\end{figure}

I Haskell är det enkelt att definiera egna datatyper vilket gör att
man exempelvis kan inkapsla redan existerande typer för att göra deras
syfte mer tydligt. Man kan också med algebraiska datatyper
implementera typer som kan fungera som ett abstrakt syntaxträd. Med
högre ordningens funktioner kan man dessutom skapa generella
funktioner som arbetar på dessa typers strukturer. Detta gör att
Haskell lämpar sig bra som värdspråk för inbäddade domänspecifika
språk.

\subsection{Domänspecifika språk}
% Allmänt om DSL
Ett domänspecifikt språk (\gls{DSL}) är ett programmeringsspråk som
till skillnad från generella programmeringsspråk är anpassat för en
specifik domän. Att skapa domänspecifika språk är ingenting nytt, det
finns tvärtom många DSL som används i vid utstäckning.
Exempel på några relativt välkända DSL är HTML, MATLAB och Game Maker
Language.

Det som är grundtanken med ett DSL är att det är relativt enkelt att
formulera lösningar för problem inom deras domäner, exempelvis är det
enkelt att strukturera upp text i HTML, utföra matrisoperationer i
MATLAB eller skapa datorspel med Game Maker Language. Däremot är DSL
ineffektiva inom andra områden än de som de är implementerade för.
Det är till exempel inte effektivt att försöka skapa datorspel i
MATLAB

När man designar domänspecifika språk kan man välja att skapa ett
fristående språk som kompileras till någon typ av maskinkod eller att
utveckla sitt domänspecifika språk i ett redan existerande generellt
programmeringsspråk, där beståndsdelarna i språket är funktionsanrop
och konstruerare i värdspråket. Ett DSL som är inbäddat i ett
värdspråk kallas för ``\gls{EDSL}''.

% TODO: Nämna att vi valde approachen EDSL
% TODO: Lägg till att det finns blandningar av djupa och ytliga EDSL.
% Det går alltså att kombinera dessa två nivåer.
Det finns även olika nivåer av EDSL; så kallade djupa EDSL (deep
embedding) och ytliga EDSL (shallow embedding). I introduktionen till
sin artikel skriver Svenningson och Axelsson \cite{Svenningsson2013}
att man i ett djupt EDSL använder sig av algebraiska datatyper och att
man med dessa bygger upp ett abstrakt syntaxträd som sedan tolkas av
olika funktioner. Man definierar alltså egna typer och funktioner som
bygger upp strukturer med dessa typer och skiljer på så sätt det
domänspecifika språket från värdspråket. I ett ytligt EDSL försöker
man däremot uttrycka sitt domänspecifika språk i termer av
värdspråkets funktioner och typer. Ett uttryck i ett ytligt EDSL
motsvaras alltså av ett uttryck i värdspråket och behöver inte tolkas
ytterligare. Båda sätten att implementera ett EDSL har sina fördelar
och nackdelar då det exempelvis i ett djupt EDSL är förhållandevis
lätt att ge fler tolkningar av det abstrakta syntaxträdet och därmed
kunna bygga vidare på det domänspecifika språket, men svårare att
lägga till konstruerare eller på andra sätt ändra i typerna som bygger
upp syntaxträdet eftersom man då måste ändra i de funktioner som
tolkar dem. I ett ytligt EDSL är det lätt att lägga till konstruerare
och typer eftersom man bara måste kunna representera dem i värdspråket
och alltså inte behöver hantera ett syntaxträd, men å andra sidan är
dess tolkning fixerad. För att lägga till nya sätt att tolka ett
ytligt EDSL måste vi implementera om det på nytt.

\subsection{Transformer, signaler och system}
Transformer, signaler och system, inom signalbehandling, beskriver
olika typer av signaler, hur de relateras till ett system och hur en
ingenjör kan modellera denna relation mellan signaler och systemet i
matematiska termer. Ett exempel på ett system kan till exempel vara en
fjäder vars insignal är en kraft som påverkar fjädern utifrån. I
kursen \textit{TSS}, och så även i vårt projekt, begränsas signalen
till enbart en oberoende variabel som beskriver tid och de system som
som betraktas är linjära och tidsinvarianta, så kallade LTI-system.

För att kunna lösa problem med denna typ av signaler och system så kan
man utveckla signalen i en Fourierserie eller använda sig av olika
typer av transformer, vilket gör det möjligt att betrakta signalen i
en annan domän. %De transformer som tas upp i kursen \textit{TSS} är
%\textit{fouriertransform}, \textit{laplacetransform} och
%\textit{Z-transformen}.
En av transformerna som tas upp i kursen är Fouriertransform.
%TODO: välj en stil: två av tre slutar på "form" och en på "formen"
%TODO: homogenisera: "laplacetransform" eller "Laplacetransform"


\textit{Fouriertransform} är en transform i diskret eller
kontinuerlig tid som transformerar en funktion från tidsdomänen till
frekvensdomänen. \textit{Fouriertransform} i kontinuerlig tid är
definierad enligt följande:
%TODO: visa eller förklara var variabeln \omega binds
\[\mathcal{F}[f] (\omega)
%\mathcal{F} f \omega
%\mathcal{F} [f] \omega
= \int_{-\infty}^{\infty} f(t) e^{-j \omega t} dt\]


%\textit{Laplacetransformen} är en transform i kontinuerlig tid som
%transformerar en funktion från tidsdomänen till s-domänen. s-domänen
%är en generalisering av den kontinuerliga
%frekvensdomänen. \textit{Laplacetransformen} är definierad som:
%TODO: visa eller förklara var variabeln s binds

%\[ \mathcal{L} f = \int_{0}^{\infty} f(t)e^{-st} dt \]

%\textit{Z-transformen} är en transform i diskret tid som transformerar
%en funktion från tidsdomänen till z-domänen. z-domänen är en
%generalisering av den diskreta frekvensdomänen. \textit{Z-transformen}
%är definierad som:
%TODO: visa eller förklara var variabeln z binds

%\[\mathcal{Z} x = \sum_{n=-\infty}^{\infty} x[n] \cdot z^{-n}\]

Området transformer, signaler och system innehåller även andra begrepp
och operatorer som komplexa tal, faltning, sampling med flera. Alla
dessa grundkunskaper behövs för att modellera system och signaler.

\subsection{Didaktik}
\label{sec:didaktik}
Didaktik är vetenskapen om undervisning. I \textit{Didaktik för
 ingenjörslärare} hävdar författarna att didaktik kan ses som ett
komplement till pedagogik \cite{didaktik_for_ingenjorslarare}. De didaktiska frågorna handlar, enligt
författarna, inte om hur studenten lär sig utan snarare om hur läraren
skapar en situation lämplig för lärande.

%TODO: "Matematikens domänspecifika språk" är namnet på det pedagogiska projektet och kursen. Ert kandidatprojekt (som jag gissar att ni hänvisar till) heter något annat: "Programmering som undervisningsverktyg för signaler och system" enligt titeln (eller "Matematikens domänspecifika språk (DSLsofMath) för andra kurser" i projektförslaget).
Som nämndes i inledningen har projektet \textit{DSLsofMath}
 som syfte att utveckla läromaterial som kan fungera som ett
komplement till kursen \textit{TSS} och väcka intresse för ämnet.
Därför är didaktik en viktig del av projektet.

En stor del av didaktik handlar om hur man kan påverka studentens
motivation. I \textit{Didaktik för ingenjörslärare} sammanfattas
motivationens betydelse enligt följande: ”Varje kurs behöver motiveras
för studenterna. Med motiverade studenter blir resultatet bättre.”.
Boken hävdar även att långt ifrån alla studenter som går en kurs utgår
från att den är läsvärd. Även i \textit{Motivational Design for
 Learning and Performance} påpekas det orimliga i att anta att alla
studenter är motiverade att lära sig ett ämne redan när de först
kommer i kontakt med det \cite{motivational_design}. Det är alltså viktigt att ett läromedel är
uppbyggt så att det motiverar studenterna att lära sig.

%TODO: Upprepa inte hela bokens titel flera gånger
Enligt \textit{Motivational Design for Learning and Performance}
behöver motivations\-aktiviteter stödja inlärningsmålen för att vara
effektiva. Motivationsmedel kan vara roligt och underhållande men om
de inte engagerar studenten i läromålen och läromaterialet så hjälper
de inte studenten att lära sig. Roliga aktiviteter och dylikt kan
användas som ett belöningssystem men främjar i sig inte lärande. Många
studenter kommer även motsätta sig eller rentav avsky aktiviteter
tänkta att motivera dem om aktiviteterna inte är knytna till
läromålen, detta gäller särskilt för vuxna studenter.
Motivationsdesign har alltså utmaningen att göra materialet
tilltalande utan att det blir ren underhållning.

Ett angreppssätt inom didaktik är \textit{ARCS-modellen}, som beskrivs
i detalj i \textit{Motivational Design for Learning and Performance}.
ARCS är en akronym för \textit{Attention}, \textit{Relevance},
\textit{Confidence} och \textit{Satisfaction}, vilket är grundpelarna
i \textit{ARCS-modellen}.

Den första grundpelaren \textit{Attention}, uppmärksamhet på svenska,
syftar till att väcka studentens uppmärksamhet och nyfikenhet.
%TODO: Försök hitta en svenskare term än "perceptuellt triggande"
Detta åstadkoms med perceptuellt triggande, frågor och variation.
Perceptuellt triggande innebär att man på något sätt ändrar i
undervisningsmiljön för att väcka studentens uppmärksamhet. Dessa
miljöförändringar kan vara av varierande typ som till exempel en
ändring i röstläge, temperatur i en undervisningssal eller ett skämt.
Denna typ av miljöförändringar väcker dock endast en tillfällig
uppmärksamhet och måste därför följas av frågor och variation för att
ge en varaktig effekt. Därför är det viktigt att i detta skede
omedelbart följa upp med frågor som väcker studentens nyfikenhet. För
att behålla hens nyfikenhet och intresse är det viktigt med variation
undervisningen. Oavsett hur bra struktur man har i sin undervisning
kommer studenterna att tappa intresset om man inte varierar upplägget.

Den andra grundpelaren \textit{Relevance}, relevans på svenska, syftar
till att övertyga studenten om att inlärningen är personligt relevant.
För att en student ska vara intresserad av att lära sig måste hen
känna att instruktionen är relaterad till personliga mål eller motiv.
Detta uppnås genom målorientering och anknytning till bekant kunskap.
Studenter är mycket mer motiverade att lära sig saker om dessa kan
hjälpa dem uppnå ett mål i nutid eller framtid, som att klara ett
prov, få en befordran, eller liknande. Därför är det viktigt att
studenterna vet hur det de lär sig är relaterat till deras mål. Ämnet
kommer även att kännas mer relevant för studenten om det kan knytas
till hens tidigare kunskap eller intresse. Även om studenter kan bli
nyfikna på helt nya saker är de som regel mest intresserade av saker
som på något sätt är knutna till deras tidigare erfarenheter.
Konkreta exempel från familjära områden gör det hela mer relevant för
studenten, detta gäller i synnerhet när man lär ut abstrakt material.

Den tredje grundpelaren \textit{Confidence}, självförtroende på
svenska, syftar både till studenters tro på att de kan ämnet och deras
tro på att de kan lära sig det. Även om en student är nyfiken på ett
ämne och känner att det är relevant är det möjligt att hen ändå inte
är tillräckligt motiverade på grund av för mycket eller för lite
självförtroende. Vissa studenter kan rentav vara rädda för ämnet. På
grund av detta är det viktigt att utveckla materialet så att studenten
övertygas om att hen kan lära sig ämnet. Därför behöver man med hjälp
av relevanta uppgifter ge studenter en upplevelse av framgång så fort
som möjligt. Detta kan vara en viktigt stimulans för fortsatt
motivation, förutsatt att uppgiften kräver så pass mycket ansträngning
att det betyder något att lyckas med den, men inte kräver så mycket
att det uppgiften leder till allvarlig oro eller hot om misslyckande.

Studentens självförtroende kan byggas upp genom lärokrav, möjlighet
till fram\-gång och personlig kontroll. Det är viktigt att sätta upp
tydliga läromål eftersom det är en stor källa till oro och osäkerhet
för studenten att inte veta vad som förväntas av hen. Efter att ha
skapat en förväntan av framgång är det viktigt att ge studenten
möjlighet att lyckas med krävande och meningsfulla uppgifter. Dessa
uppgifter bör se olika ut beroende på vilken nivå studenten befinner
sig på. Studenter som är nya inom ett område reagerar generellt bäst
på om att ha en relativt låg svårighetsnivå med frekvent återkoppling
som hjälper dem lyckas eller bekräftar deras framgång. När studenter
bemästrat grunderna är de redo för mer krävande uppgifter.
Slutligen är det viktigt att studenten känner att hen har personlig
kontroll över lärandesituationen. Att uppleva att man själv kan
kontrollera utgången av en situation är avgörande för självförtroendet.
För att förstärka motivationen bör instruktören förse studenten med en
stabil lärandemiljö och sedan låta studenten ha personlig kontroll över sin
inlärning. Det är viktigt att skapa en miljö där det är acceptabelt och helt
i sin ordning att göra misstag och lära sig av dem.

Om man använder de första tre grundpelarna, Attention, Relevance och
Confidence, väl, kommer studenten vara motiverad att lära sig. Det är
för att upprätthålla denna motivation som den sista grundpelaren
\textit{Satisfaction}, tillfredsställelse på svenska, används. För
att studenten ska ha ett fortsatt intresse av att lära måste hen känna
tillfredsställelse med inlärningens process eller resultat. Denna
tillfredsställelse kan uppnås genom naturliga konsekvenser och
allmänna positiva konsekvenser. Med naturliga konsekvenser menas den
kunskap som blir en direkt följd av undervisningen. Att kunna lösa en
uppgift eller dylikt som man inte kunde lösa innan ger en
tillfredsställelse i sig. Det är ett starkt belöningsverktyg att låta
studenten använda ny kunskap. Allmänna positiva konsekvenser kan vara
materiella belöningar, som till exempel en löneförhöjning, eller
symboliska belöningar som diplom eller ett bara ett erkännande av
studentens förmåga.

Hur dessa didaktiska modeller och metoder har använts i projektet
``Matematikens domänspecifika språk'' beskrivs i avsnittet
\textit{Utveckling av läromaterialet}.

\subsection{Relaterad forskning}
\label{sec:relForsk}
I detta avsnitt kommer vi i korthet beskriva några studier som
är nära relaterade till vårt projekt.

En studie som studerat problem som liknar vårt är \textit{The role of
programming in the formulation of ideas} \cite{sussman2002role} av
Gerald Sussman och Jack Wisdom.
Som Sussman och Wisdom uttrycker det är det svårt att lära sig fysik:
de resonerar att detta troligen kan bero på
att studenter tycker att det är svårt med det matematiska språket och
att det därför blir problematiskt att lära sig både ämnet och språket
som ämnet är uttryckt i. De jämför det med att försöka läsa
\textit{Les Misérables} samtidigt som man försöker lära sig
franska. Det huvudsakliga problemet med att uttrycka idéer och koncept
i det matematiska språket är dock inte språket i sig utan snarare att
man, som med naturliga språk, antar att den man talar med eller
förklarar för är van vid det matematiska språket och därför delar
samma kunskapsbas om matematik som en själv. Detta gör, enligt Sussman
och Wisdom, att man istället för att förklara
matematiken så tydligt som man bör snarare skissar upp en lösning,
vilket för den matematiskt ovane studenten blir i det närmaste
obegripligt.

En lösning, som föreslås i artikeln, på hur man hittar den
detaljnivån som krävs för att otvetydigt förklara något matematiskt är
att försöka förklara det för en dator. En dator försöker inte tänka
själv och accepterar inte halvfärdiga instruktioner där resten lämnas
till intuitionen.

En annan studie som är intressant för vårt projekt är
\textit{Communicating mathematics: Useful ideas from computer science}
av Charles Wells \cite{wells1995communicating}. I denna artikel
argumenterar Wells för att matematiker har mycket att vinna på att
uttrycka sig tydligare. Wells
drar paralleller till datavetenskapen där man skiljer noga på syntax
och semantik samt tänker explicit på typer och hur dessa
manipuleras. Detta skiljer sig från matematiken där man inte separerar
på semantik och syntax i samma grad och där typerna oftast är
implicita och lämnas till läsaren att själv fundera på.

Projektet DSLsofMath bygger som tidigare nämnts på idéerna från Wells,
Sussman och Wisdom och vårt projekt bygger i sin tur på DSLsofMath.
I kursen \textit{Matematikens domänspecifika
 språk} får studenterna med funktionell programmering angripa
matematiska problem genom att bygga domänspecifika språk för att kunna
hantera problemen programmeringsmässigt. För att kunna göra detta
måste studenterna fundera över vad de faktiskt ska modellera och vilka
typer de kan tänkas behöva för att uttrycka problemet
med~\cite{kursplan:dslsofmath}.

\newpage

\section{Utveckling av läromaterialet}
\label{sec:utveckling}

% Huvudpoänger:
% * Förstudier, intervjuer mm
% * Enkätundersökning
%    * Hänvisa till böckerna “Enkäten i praktiken” och “Enkätboken”
% * Didaktik, hur vi skrivit vår tutorial och varför
%    * Hänvisa till “Didaktik för ingenjörer” och “Motivational Design
%        for Learning and Performance”
% * TSS, hur vi har tacklat ämnet
% * Funktionell programmering och dsl, hur vi valt att
%     implementera det hela och varför
% * Testning


I detta avsnitt beskrivs hur projektgruppen gått till väga under
utvecklingen av läromaterialet, samt vilka verktyg och metoder som
använts och hur dessa tillämpats.

Enligt boken \textit{Motivational design for learning and performance}
kan man använda sig av en modell med 10 steg vid framtagningen av nytt
material till en kurs \cite{motivational_design}. Dessa 10 steg är som följer:

\begin{enumerate}
%TODO: Formulera om "Erhållning av". Kanske "Samla"?
\item Erhållning av kursinformation. Hur ser situationen ut nu? Vad är
 nuvarande kursbeskrivning, kursmål, etc?

\item Erhållning av information om studenterna. Vad har studenterna
 för relevant egenskaper i form av intressen och tidigare
 erfarenheter och förkunskaper?

\item Analys av studenterna. Vad har studenterna för motivation och
 syn på kursen?

\item Analys av existerande material. Finns det någon
 motivationstaktik i det nuvarande materialet och är den i så fall
 lämplig?

\item Lista mål och utvärderingar. Vad vill jag åstadkomma och hur
 bedömer jag om jag lyckas?

\item Lista möjliga taktiker. Vilka möjliga taktiker kan uppnå de
 satta målen?

\item Välj och designa taktik. Vilken/vilka taktiker passar bäst för
 den här publiken, ämnet etc.

\item Integrera med instruktion. Hur kan jag kombinera
 motivationskomponenterna med ämnesinformationen?

\item Välj och utveckla material. Hur hittar eller skapar jag material
 för dessa mål?

\item Utvärdera och iterera vid behov. Hur kan jag utvärdera de
 väntade och oväntade effekterna av resultatet?

\end{enumerate}

Dessa steg har legat till grund för, och även använts för att
komplettera, mycket av projektgruppens metodik när läromaterialet
utvecklats. Gruppen inledde i enlighet med de 10 stegen med att samla
in och analysera information om kursen och målgruppen, se avsnitt
\textit{\ref{sec:efterforskning} - \nameref{sec:efterforskning}},
arbetade fram en taktik enligt ARCS-modellen, se avsnitt \textit{\ref{sec:didaktik}
- \nameref{sec:didaktik}}, utvecklade och testade materialet, se avsnitt \textit{\ref{sec:test}
- \nameref{sec:test}}, utvärderat och kompletterat det.

Gruppen hade även två inspirationskällor i form av
%Borde betterexplained också ha en riktig referens?
\url{http://betterexplained.com/} och \textit{Learn You a Haskell for Great Good} \cite{learnyouahaskell}.
Detta är två läromaterial som projektgruppen upplevt är både
informativa och underhållande läsning.

\subsection{Inledande efterforskning och förundersökning}
\label{sec:efterforskning}

Projektgruppen inledde med litteraturstudier och efterforskningar inom
ämnet. Projektdeltagarna läste in sig på signallära, DSL, didaktik,
funktionell programmering och annan relaterad forskning.

Under projektets tidiga fas intervjuades två personer ur
personalstyrkan på Chalmers som undervisat och varit examinatorer i
kurserna \textit{TSS} och \textit{Reglerteknik}.
%TODO: homogenisera \ref till bilagorna: här "Bilaga \ref...", nedan "bilaga 1".
Se \textit{\ref{bil:exam_intervju} - \nameref{bil:exam_intervju}} för en sammanfattning av denna
intervju. Under intervjun redogjorde examinatorerna för hur de
upplevde studenternas problem med kurserna och vad de trodde var
orsaken.

Projektgruppen sammanställde sedan en enkät, riktad till de studenter
som läst kurserna \textit{TSS} och \textit{Reglerteknik}, eller skulle
läsa kurserna inom ett år. Denna enkät arbetades fram utifrån böckerna
\textit{Enkäten i praktiken}\cite{enkaten_i_praktiken} och \textit{Enkätboken}\cite{enkatboken}.

När enkäten var framtagen användes \textit{webbenkater.com} för att
enkelt kunna hantera och analysera studenternas svar anonymt.
%TODO: Skriv hellre "skickades" (se även "mailades" nedan)
Länken till enkäten skickades ut till alla
datastudenter från årgångarna 2012-2014 och gruppen erhöll totalt 77
svar. Se \textit{\ref{bil:1} - \nameref{bil:1}} för ett utdrag ur dessa svar.

Projektgruppen sammanställde sedan den insamlade informationen om
kursen, såsom kursplan, lärandemål samt de senaste årens
tentor. Utifrån kursinformationen, samt intervjun och resultatet från
enkätundersökningen, beslutades vilka områden som läromaterialet
skulle täcka.

\subsection{Läromaterialets struktur}
\label{sec:struktur}
Läromaterialet är uppdelat i fyra ämnesavsnitt samt en introduktion. I
introduktionen beskrivs läromaterialet övergripande samt vilka
förkunskaper som krävs. Där förklaras även konceptet DSL. Efter
introduktionen följer det första avsnittet, vilket kort beskriver
komplexa tal och Eulers formel samt introducerar studenten till vårt
DSL. Det andra avsnittet beskriver olika typer av signaler och deras
egenskaper, det tredje beskriver LTI-system och det fjärde och sista
beskriver Fourierserier och Fouriertransform. De ämnen avsnitten
beskriver valdes, som nämns i avsnittet \textit{\ref{sec:efterforskning} - \nameref{sec:efterforskning}}, utifrån information från
kursen, intervjun med examinatorerna samt studenternas enkätsvar.

Läromaterialets första ämnesavsnitt beskriver komplexa tal. Eftersom
förståelse av komplexa tal är en viktig förkunskap till \textit{TSS}
valde gruppen att inleda med repetition av detta ämne. En annan
anledning till att inleda med komplexa tal var att detta är ett område
datastudenter känner relativt väl till vid den tidpunkt i studierna då
de läser \textit{TSS}. Gruppen ville ge studenterna möjlighet att
bekanta sig med det DSL som används och hur man kan implementera
matematik med detta innan man började beskriva nytt material. Därför
ansåg gruppen att det var lämpligt att inleda med ett ämne som
datastudenterna kände sig relativt bekväma med för att de skulle kunna
fokusera på programmeringsaspekten och förstå upplägget med DSL.

Det andra avsnittet beskriver signaler. Signaler är ett viktigt
begrepp i kursen \textit{TSS} och kunskap om signaler är en nödvändig
grund för att förstå de följande avsnitten. Därför var det ett
självklart val för gruppen att låta andra avsnittet handla om detta. I
avsnittet förklaras, något förenklat, med text och figurer vad
signaler är och hur några vanliga typer av signaler, som till exempel
enhetssteg, ser ut och beter sig.

Det tredje avsnittet beskriver LTI-system. I princip alla system som
beskrivs i \textit{TSS} är LTI-system. Kunskap om LTI-system och dess
egenskaper är därför avgörande för att förstå och räkna på de system
som tas upp i \textit{TSS}. Avsnittet tar även upp faltning och hur
detta fungerar i LTI-system.

Det fjärde avsnittet beskriver Fourierserier och
Fouriertransform. Dessa är centrala begrepp i kursen och något som
många datastudenter upplever som svårt. Därför valde gruppen, då det
konstaterades att det inte fanns tid att beskriva alla transformer som
finns med i \textit{TSS}, att fokusera på Fourier. En anledning till
detta var att Fouriertransform är nära besläktad med såväl
Laplacetransform som Z-transform och gruppen bedömde att
förståelse för Fouriertransform sannolikt skulle bidra till
förståelse för de andra två transformerna.

Mer information om hur avsnitten utformats för att uppmuntra till
inlärning följer i nästa avsnitt, \textit{\ref{sec:matDidaktik} - \nameref{sec:matDidaktik}}.

\subsection{Didaktik i läromaterialet}
\label{sec:matDidaktik}
Läromaterialet ska vara uppbyggt så att studenten själv får arbeta med
det utifrån sina tidigare kunskaper. Vi ville åstadkomma ett aktivt
lärande som ger studenten långvarig förståelse och intresse för ämnet,
snarare än kunskap studenten lär sig tillfälligt
utantill. \textit{Didaktik för ingenjörslärare}\cite{didaktik_for_ingenjorslarare} beskriver aktivt
lärande enligt följande: ”Idéen bakom aktivt lärande är att en
lärprocess bygger på en konkret erfarenhet som följs av reflektion och
observation.” Med bakgrund av detta har vi lagt en stor del av vårt
fokus på att göra läromaterialet interaktivt. Vi har lagt in enklare
kryssfrågor i texten för att låta studenten arbeta aktivt med ämnet
när de läser teorin och vi har även lagt in programmeringsövningar där
studenten får använda och utveckla sina kunskaper i ämnet.

Eftersom läromaterialet som tagits fram är ett komplement till en
redan existerande kurs har gruppen bedömt det som extra viktigt att
materialet är motiverande eftersom studenten läser det vid sidan av
sina vanliga studier. Därför har gruppen valt att använda sig av
ARCS-modellen som beskrivs i avsnitt \textit{Teknisk bakgrund -
 Didaktik}.

ARCS-modellens första grundpelare, \textit{Attention}, har vi
praktiserat bland annat genom att använda skämtsamma formuleringar,
sammanfattningar och exempel inom ämnet. Vi har strävat efter att
hålla en lättsam och humoristisk ton för att hålla studenternas
uppmärksamhet. Vi har dock försökt undvika lustiga utvikningar,
formuleringar och exempel som avviker från ämnet eller inte fyller sin
funktion att främja intresse och förståelse för ämnet. För att sedan
väcka ett mer varaktigt intresse hos studenten ställer vi frågor i
löpande text och övningar. Varje avsnitt inleds med en fråga av typen
“Vad är det här egentligen för något och vad används det till?”. För
att behålla studentens intresse genom hela läromaterialet har vi valt
att variera upplägget på avsnitten till viss del. Eftersom vi vill
behålla en läsvänlig och överskådlig struktur i hela läromaterialet
har vi valt att främst variera materialet med olika typer av exempel
och uppgifter. I ett kapitel får studenterna fylla i ett jämförelsevis
stort antal kryssfrågor och i ett annat lägger vi istället mer fokus på en
viss typ av programmeringsövningar där studenten får fylla i saknad
kod. Vi har alltså siktat på att göra studenternas eget arbete
omväxlande för att att de inte ska bli uttråkade och ge upp.

Den andra grundpelaren, \textit{Relevance}, har vi praktiserat främst
i inledningen av avsnitten och i
introduktionsavsnittet. Introduktionsavsnittet förklarar att
läromaterialet finns till för att vara ett komplement till, och hjälpa
datastudenterna med, \textit{TSS}. Varje ämnesavsnitt inleds sedan med
en kort beskrivning av ämnet och vad det kan användas till. På detta
sätt förklarar vi inledningsvis varför läromaterialet är relevant för
studentens studier. Sedan förklarar vi mer i detalj vad de olika
ämnesområdena kan användas till alteftersom studenten stöter på dem i
materialet.

Den tredje grundpelaren, \textit{Confidence}, har vi praktiserat genom
att skapa övningar och uppgifter av varierande svårighetsgrad och ett
upplägg där studenten själv kontrollerar sin inlärning. Vad gäller
övningarna så inleder vi i det första avsnittet med enklare
kryssfrågor och relativt enkla programmeringsövningar. Kryssfrågorna
kan kryssas i direkt på hemsidan, varpå de rättas direkt för att ge
studenten omedelbar bekräftelse. Det finns även lösningsförslag till
programmeringsövningarna som studenten kan jämföra sin kod med eller
titta på om hen kör fast. Eftersom vi vill att studenten själv ska
kunna kontrollera sin inlärning och arbeta i en takt som fungerar för
hen passar det utmärkt att lägga upp läromaterialet i form av en
hemsida. På så vis har studenten tillgång till all information och
alla uppgifter men kan arbeta med materialet som hen vill.

Den fjärde och sista grundpelaren, \textit{Satisfaction}, har vi
praktiserat främst i slutet på avsnitten och i uppgifterna. I våra
kryssfrågor får studenten omedelbar bekräftelse när hen har lärt sig
något eller kan det sedan innan. Efter att studenten har läst teorin
inom ett område följer vi upp med övningar där studenten får
praktisera sin nyvunna kunskap. Varje avsnitt i läromaterialet
avslutas sedan med en lista på vad studenten nu har lärt
sig. Studenten får alltså både ett erkännande av sin kunskap och
förmåga, samt möjlighet att praktisera detta.

\subsection{Implementationen av vårt DSL}
\label{sec:implDSL}
%Förklara hur komplexa tal och alla typer av transformer implementerades i DSL.
Det DSL som används i detta läromaterial har utvecklats i Haskell av
projektgruppen. Det har designats för att täcka de valda delarna av
TSS som behandlas i läromaterialet. Se avsnitt
\textit{\ref{sec:efterforskning} - \nameref{sec:efterforskning}} för mer
information om hur dessa delar valts ut.

En av dessa delar är komplexa tal, vilket används flitigt när man
arbetar med signaler och system. Därför implementerades också komplexa
tal som första byggstenen i vårt DSL. Därefter implementerades
signaler, LTI-system och Fouriertransform.

Själva utvecklingen av det DSL som använts har skett genom att gruppen
först utvecklat all kod som behövts för ett avsnitt. När koden sedan
hade testats och fungerade tillfredsställande delades den upp i
kodpaket. I dessa kodpaket har vissa funktioner tagits bort för att
lämnas som övningar till läsarna att implementera. Resterande kod i
paketen kan användas av studenten så att hen kan köra sin egen kod
direkt och testa sina lösningar.

Samtliga funktionerna har givits så beskrivande namn som
möjligt. Projektgruppen har även försökt åstadkomma en så lättläst
syntax som möjligt där alla implementerade funktioner kan läsas och
förstås av någon som behärskar grundläggande Haskell. Detta var
viktigt då gruppen ville åstadkomma kod som kan förstås av hela
målruppen och inte bara av studenter som är särskilt intresserade av
Haskell.


För att visa hur projektgruppen har tänkt vid implementationen av sitt
DSL kommer delar av implementationen av komplexa tal nu att
beskrivas. Vi börjar med hur ett komplext tal representeras, i vårt
fall som två flyttalsvärden.
\begin{minted}{haskell}
  data Complex = Complex Double Double
    deriving Eq
\end{minted}
Ett värde för den reella delen av talet
och ett för den imaginära. Så ett komplext tal
\(z = 1 + 1j = (1,1) \) representeras i vår DSL som:

\begin{minted}{haskell}
  Complex 1 1
\end{minted}

%TODO: jag skulle(;-) föredra att gör texten mer aktiv: "ska bete sig", "behöver vi implementera", "kan implementeras", ...
De implementerade komplexa talen skulle bete sig som matematikens
komplexa tal och därför var det nödvändigt att implementera diverse
beräkningsoperatorer för dem. Addition kunde implementeras relativt
enkelt då de reella delarna adderas med varandra och de imaginära med
de imaginära enligt formeln nedan \cite{conway1978functions}.
\[(a, b) + (c, d) = (a + c, b + d)\]
% TODO: Ändra så att citeringen görs i löpande text och inte "Hänger
% lös" i början av en rad.
% TODO: punkt i slutet av förra meningen
Vilket i Haskell implementerades så här:
\begin{minted}{haskell}
 z + w = Complex (realPart z + realPart w) (imPart z + imPart w)
\end{minted}

I kodexemplen ovan används funktioner som plockar ut den reella
(\mintinline{haskell}{realPart}) respektive den imaginära delen
(\mintinline{haskell}{imPart}) ur ett komplext tal.

Multiplikationen var lite mer komplicerad än additionen, då den
matematiska formeln för multiplikation med komplexa tal ser ut så här:
 \[(a, b) \cdot (c, d) = (ac - bd, ad + bc) \] \cite{conway1978functions}

Detta implementerades enligt följande i vårt DSL:
% TODO: fixa sidbrytning (senare)
\begin{minted}{haskell}
 z * w = Complex (realZ*realW - imZ*imW) (realZ*imW + realW*imZ)
 where realZ = realPart z realW = realPart w imZ = imPart z imW =
 imPart w
\end{minted}

%Är detta nödvändigt? Vi borde liksom inte vara för förklarande.
Här skapades även hjälpdefinitioner, vilka är de funktioner man finner
efter \mintinline{haskell}{where} i koden. Detta var för att undvika
att skriva samma kod om och om igen, samtidigt som det koden mer
tydlig och lättläst för studenten.

Efter att dessa operationer var färdiga så var division nästa stora
operator att implementera. Den matematiska formeln för detta ser ut
som följer:
\[ z / w = (z \cdot w') / (w \cdot w') \]
%(sommarmatte.se) Kan lämnas som reference.
där $w'$ är $w$:s konjugat.
%Man förlänger både täljare och nämnare med nämnarens konjugat vilket innebär
%att nämnaren blir helt reell.
Man förlänger alltså både täljare och nämnare med nämnarens konjugat,
vilket innebär att nämnaren blir helt reell. För att implementera
division var gruppen därför tvungen att först implementera en funktion
som tar fram konjugatet av ett komplext tal. Detta innebär att man
byter tecknet på den imaginära delen av talet. Vår funktion för det
såg ut så här:

\begin{minted}{haskell}
conjugate z = Complex (realPart z) (negate (imPart z))
\end{minted}
Vår implementation av division blev då följande:
%TODO: för lång rad
\begin{minted}{haskell}
z / w = Complex (realPart zw' / realPart ww')
                (imPart zw' / realPart ww')
  where zw' = z * (conjugate w)
        ww' = w * (conjugate w)
\end{minted}

En viktig förutsättning för att kunna koppla de komplexa talen till
kursen TSS var att man först implementerade Eulers formel, som kan
användas för att skapa komplexa tal utifrån en vinkel \(\phi\). Eulers
formel ser ut så här:
%TODO: använd helst samma variabelnamn efter j här och nedan. ("z" här känns som ett komplext tal, men texten pratar om en "vinkel" dvs. ett reellt tal)
% \cite{trott2004}:
\[e^{j\phi}=\cos \phi+ j \cdot \sin \phi \]
%(encyclopediaofmath) -kan lämnas som reference
Implementationen av den här representationen skedde i flera steg.
Funktionen \mintinline{haskell}{euler} skapar ett komplext tal utifrån
en given vinkel, representerad av ett flyttal.
\begin{minted}{haskell}
euler z = Complex (cos z) (sin z)
\end{minted}
Enligt potenslagarna %(formelsamlingen) - lämnas till reference? -Borde inte behövas
så är \(e^{a+jb} = e^{a} \cdot e^{jb}\) och enligt Eulers formel så är
\(e^{j b} = \cos b + j\cdot \sin b\).
%
Exponentialfunktionen för ett komplext tal kan därför implementeras
enligt nedanstående kod:

\begin{minted}{haskell}
exp z = scale (exp (realPart z)) (euler (imPart z))
\end{minted}
Sedan implementerades även trigonometriska funktioner med hjälp av
Eulers formel. Så här representeras $sin(x)$ som ett komplext tal:
\[ sin(x) = (e^{j x} - e^{-j x}) / 2 j \]
Vilket ser ut så här i vårt DSL:
\begin{minted}{haskell}
sin z = (exp (j*z) - exp (-(j*z))) / (scale 2 j)
 where j = Complex 0 1
\end{minted}

Den fullständiga koden kan läsas på gruppens hemsida:
\url{https://cdn.rawgit.com/DSLsofMath/BScProj/master/Hemsida/}

Ett av de största problemen vid implementationen av signaler, och all
kod som bygger på signaler, var hur man skulle hantera kontinuerlig
tid. Kursen \textit{TSS} behandlar både diskreta och kontinuerliga
signaler och det var därför önskvärt att implementera båda dessa. En
dator kan dock under normala förhållanden inte hantera kontinuerliga
funktioner utan endast diskreta. %Osäker på hur man borde uttrycka detta för att det ska vara korrekt

Ett sätt att hantera detta är att simulera ingående data till en
kontinuerlig funktion med diskret data, där man låter avståndet mellan
de diskreta värden datan kan anta vara väldigt små. Ju mindre dessa
steg är desto bättre approximation av en kontinuerlig funktion får
man. Frågan som uppstår är hur lämplig en diskret approximation är för
att främja förståelsen av signaler mm i kontinuerlig tid. Se avsnittet
\textit{Diskussion} för vidare diskussion om detta.

Ett annat sätt att implementera kontinuerliga fall är med så kallad
\textit{pattern matching}. Det man då gör är att implementera olika
utfall av en funktion var för sig. Funktioner som implementeras på
detta sett ger svar som är mer matematiskt korrekta än en diskret
approximation. Detta ger dock inte samma möjlighet att se mönster och
att sammanfatta området i en funktion.

Projektgruppen valde till slut en kompromiss av dessa två angreppssätt
och använde både diskreta approximationer och \textit{pattern
 matching} för att beskriva kontinuerliga fall. Gruppen har växlat
mellan varianterna beroende på vilken av dem som föreföll ge den
tydligaste förklaringen av det aktuella ämnet.

\subsection{Presentation av läromaterialet}
%* PDF
%* Bootstrap
%* GitHub
Vi började med att skriva läromaterialet i LaTeX-format för att kunna
skicka ut en prototyp, en PDF-fil, till vår testgrupp. Det var även
lätt att få en bra struktur och att få ett bra utseende på vårt
läromaterial på detta sättet.

Allt arbete har även under projektets gång legat uppe på GitHub så det
har varit möjligt att följa projektet samt ge studenter och lärare
möjlighet att ge sina åsikter på ändringar med mera.

När de olika delarna i läromaterialet blivit färdiga så skrevs dem
över till HTML, med hjälp av Pandoc, för att kunna skapa en hemsida så
att det blir mer lätt\-åtkomligt. Pandoc är ett program som översätter
text mellan olika märkspråk och fungerar till den största delen ganska
smidigt, det blir bara några fel om man använder vissa speciella
tillägg. En annan fördel med att arbeta med HTML-sidor i stället för
PDF är att man kan använda sig av JavaScript, vilket ger möjligheten
att kunna göra uppgifterna interaktiva. Med hjälp av JavaScript blir
det bättre flyt i frågorna och man känner att man hänger med mer i vad
man lär sig så det inte känns som lika mycket att läsa.


Vi använde oss av jQuery, vilket är ett JavaScript-bibliotek, för att
göra det lättare att arbeta med JavaScript. jQuery anses av många vara
lättare att läsa och blir därmed lättare att felsöka problem i
kodningen.

Vi använde oss även av Bootstrap som stöd för hemsidan. Bootstrap gör
det lättare att få sidan att se snygg ut och gör även det möjligt att
gå in på sådan från en mobil eller surfplatta, vilket gör att det går
att lära sig på bussen eller liknande.

De figurer som gjorts till läromaterialet i form av funktionskurvor
och liknande är skapade i Matlab och \url{fooplot.com/}.

\subsection{Test av läromaterialet}
\label{sec:test}

För att få så givande testsvar som möjligt ville vi givetvis testa
vårt läromaterial på den aktuella målgruppen. Därför tog vi kontakt
med studenter som kunde tänka sig att delta i testandet när vi
skickade ut enkäten som beskrivs i avsnitt \textit{\ref{sec:efterforskning} -
\nameref{sec:efterforskning}}. De som
sedan ville deltaga i testandet av vår produkt lämnade en
e-postadress. Detta resulterade i en testgrupp på 13 personer.

När vi ansåg att läromaterialets tre första avsnitt var redo för att
testas så skickade vi ut dessa till testgruppen. Testningen fungerade så
som att man som testperson fick ta del av materialet, med texter,
exempel och tillhörande uppgifter. Testgruppen fick även tillgång till
programkoden som var skriven till uppgifterna, samt även facitkod och
lösningar till uppgifterna.

När studenterna sedan har testat produkten och gått igenom vad de
tyckte var bra och vad de tyckte var dåligt så bearbetade vi deras
återkoppling, utvärderade och såg ifall vi hittade några större fel
eller delar som vi tyckte att vi borde fokusera på att
förbättra. Sedan implementerade vi förbättringarna så att
läromaterialet på det stora hela blir mer lättläst och bättre anpassat
till målgruppen.

Det sista avsnittet skickades inte ut till testgruppen utan
utvärderades endast internt av gruppens medlemmar då detta avsnitt
inte var klart i tillräckligt god tid för att ge testgruppen en rimlig
chans att arbeta med materialet. Den mer generella feedback som gavs
på de tre första avsnitten tillämpades dock även på det sista
avsnittet.

\section{Resultat}

% Huvudpoänger:
% *Vad blev det för produkt? Tutorial på hemsida med 6 delavsnitt?
% *Hänvisa till hemsidan
% *Vad tyckte testgruppen?
% * *Citat från testgruppen

%
% Illustrationer:
% * Utdrag ur tutorialen
Projektgruppens uppgift var, vilket beskrivs i detalj i avsnittet
\textit{\ref{sec:prodSpec} - \nameref{sec:prodSpec}}, att ta fram ett kompletterande
läromaterial till \textit{TSS} inspirerat av \textit{DSLsofMath}.

%I detta avsnitt beskrivs det resulterande läromaterialet och
%resultaten från testerna.

\subsection{Läromaterialet}

Läromaterialet består av fyra avsnitt samt en introduktion. Avsnitten
innehåller förklarande text, bilder, kodexempel och uppgifter. Varje
avsnitt har även ett tillhörande kodpaket med förimplementerad kod
till övningarna och lösningar till dessa. De fyra avsnitten behandlar
i huvudsak komplexa tal, signaler, LTI-system och Fouriertransform.
Ytterligare avsnitt, som beskriver Laplacetransform och
Z-transform, var planerat men detta fick bortprioriteras på grund av
tidsbrist.

Läromaterialet finns samlat på en hemsida, där det även är möjligt att
ladda ner materialet i form av en PDF-fil. Följ länken nedan för att
se hela materialet:
\url{https://cdn.rawgit.com/DSLsofMath/BScProj/master/Hemsida/}


\textit{Bilaga 3} innehåller ett utdrag ur läromaterialet. Detta
utdrag är ett delavsnitt från avsnitt 2. Delavsnittet beskriver
\textit{Udda och Jämna signaler}. Det inleds med övergripande
definitioner och förklaringar av vad som menas med udda och jämna
signaler inom signallära, detta illustreras sedan med
figurer. Därefter följer kryssfrågor där studenten ska avgöra om den
givna signalen är jämn, udda eller ingetdera. Detta följs av ett
kodexempel där studenten får se hur man kan implementera en funktion
som avgör om en signal är udda. Studenten får sedan implementera
motsvarande funktion för jämna funktioner själv som en övning. Till
sist ges studenten två signaler att testa sina
programmeringsfunktioner på, både för detta och föregående delavsnitt.


\subsection{Testresultat}
\label{sec:testResultat}
Introduktionen och de tre första avsnitten i läromaterialet har
granskats av den testgrupp som beskrivs i avsnittet \textit{\ref{sec:test} - \nameref{sec:test}}. Det sista avsnittet har inte granskats av
testgruppen, utan endast av projektgruppen själva samt projektgruppens
handledare. Orsaken till detta beskrivs även det i avsnitt \textit{\ref{sec:test}
  - \nameref{sec:test}}.

Responsen från projektets testgrupp var genomgående
positiv. Testgruppen hävdade att de trodde att läromaterialet var ett
bra komplement till kursen.

Samtliga testsvar som kommenterade texten angav att studenterna
tilltalades av textens nivå och formuleringar. Här följer några citat
från studenter som testat materialet:
\begin{itemize}
\item ``Det lättsamma sättet som texten var skriven på gjorde att man
 inte tappade intresset.''
\item ``Texten var i allmänhet bra och kul att läsa, gillade den
 lättsamma tonen.''
\item ``Nivån är väldigt bra och att ni ibland använder monolog, där
 ni ställer en fråga och sen svarar som `en vän', gör att det känns
 mer begripligt och man får en känsla av att man pratar och
 diskuterar med en kompis om diverse ämnen. Detta ökar då förståelse
 enormt då det är annorlunda skrivet än hur till exempel föreläsare
 skriver sitt material.''
\end{itemize}

Testgruppen ansåg även att programmeringsövningarna var begripliga och
låg på en bra nivå. Några citat från testgruppen angående övningarna
följer här:
\begin{itemize}
\item ``Övningarna känns begripliga och att ni visar vilket `out-come'
 som efterfrågas hjälper en att förstå vad exakt som behöver göras.''
\item ``De övningarna jag gjorde kändes lagom svåra.''
\end{itemize}

Responsen från testgruppen innefattade även konkret kritik, exempelvis
önskade flera studenter ett annat typsnitt för övningarna än för den
övriga texten. Materialet har uppdaterats utifrån dessa önskemål och
kommentarer.

\section{Diskussion}

% Huvudpoänger:
% * Har vi löst problemet?
% * Går problemet att lösa?
% * Vad är i grund och botten orsaken till problemet?
% * Vilka problem har dykt upp?
% * Vilka problem kan man undvika om man gör ett liknande arbete?
% * Stämmer vårt resultat överens med tidigare forskning?
% * Hur kan man bygga vidare på detta?



I inledningen nämns det grundläggande problemet att datateknologer har
svårt för kursen \textit{TSS}, eftersom de har problem med
matematiken. Frågor man då kan ställa sig är vad som är orsaken till
detta problem, huruvida problemet går att lösa helt, och om det
verkligen är en bra metod att använda programmering för att lösa det?
%Svara på frågorna! Hänvisa till didaktik för ingenjörslärare etc

Något som möjligen är en del av orsaken till problemet är när kursen
\textit{TSS} ges. Det nuvarande upplägget på datateknologernas studier
är sådant att de läser ett antal kurser i matematik under sitt första
år, knappt någon matematik under sitt andra år och sedan läser de
\textit{TSS} i årskurs tre. Det är alltså nästan ett år mellan
matematikkurserna och \textit{TSS}. Denna del av problemet kan i så
fall lösas genom att till exempel ge TSS i årskurs två. Det är dock
osäkert om detta är en tillräckligt stor del av problemet för att en
förändring i kursupplägget skulle ge synbar effekt.

%TODO: använd \ref för att hänvisa till bilagan
Under intervjun med examinatorerna, se \textit{\ref{bil:exam_intervju} - \nameref{bil:exam_intervju}}, framgick det
att de tror att en stor del av problemet ligger i svårigheten att
koppla den formella matematiken till dess praktiska tolkning.

Detta skulle i så fall kunna orsaka problem när datateknologer kommer
i kontakt med matematiska läromaterial. Många matematiska texter är
uppbyggda så att de har ett inledande fokus på formell matematik, med
tyngd på matematisk korrekthet, och praktiska exempel och förklaringar
följer först senare, om alls. Ett exempel på hur en matematisk
lärotext kan vara upplagd att man inleder med en formell definition,
följd av ett matematiskt bevis, och till sist exempel. Denna typ av
upplägg kan mycket väl fungera bra för studenter som är vana vid
formell matematik men risken är att studenter som inte är det, till
exempel datateknologer, tappar bort sig redan vid definitionen.

I de följande delavsnitt kommer projektets metod, resultat och möjlig
framtida utveckling att diskuteras.

\subsection{Metoddiskussion}
\label{sec:metDisk}
För att utveckla ett väl fungerande läromaterial krävs goda kunskaper
inom såväl ämnet som didaktik. Projektgruppen består dock av fem studenter, vars erfarenhet
inom området signal och systemlära var att fyra av dem läst kursen
\textit{TSS}. Studenternas tidigare kunskap i didaktik var också
ytterst begränsad. På grund av detta lade projekt\-gruppen en stor del
av sin tid på nödvändig förberedande efterforskning. Detta lämnade
mindre tid till själva utvecklingen av läromaterialet.

Man kan ställa sig frågan om det överhuvudtaget är en bra idé att
låta studenter ta fram den här typen av läromaterial. En nackdel
är att studenterna som skriver materialet behöver lägga en stor
del av tiden på att lära sig ämnet och nödvändig didaktik
tillräckligt bra, precis som projektgruppen har behövt göra.
Därmot är följden av detta att studenterna
får en djupare förståelse för ämnet samt får nya
eller förbättrade kunskaper inom didaktik.
Därför går det att argumentera för utvecklandet av ett läromaterial
som ett läromedel.

En annan fördel med ett läromaterial som utvecklas av studenter
är att dessa känner till målgruppen studenter relativt väl,
eftersom de ingår i den. Studenterna har därför rimligtvis lättare
att sätta sig in i vad som är svårt att förstå för en student
än vad en expert inom ämnet har. Det är också möjligt att studenter
har lättare för uttrycka sig på ett sätt som andra studenter
kan förstå och ta till sig. Däremot kommer man inte ifrån
att experterna har betydligt djupare kunskaper inom sitt ämne
än studenterna och det är ytterst tveksamt att en grupp studenter
skulle kunna ta fram ett läromaterial som är på samma nivå med
lika goda insikter i ämnet som en expert. Det kan dock vara så
att den här typen av studentutvecklade läromaterial är ett bra
komplement till ämnesexperterna. En av kommentarerna från
testgruppen, se avsnitt \textit{\ref{sec:testResultat} - \nameref{sec:testResultat}} för hela sammanhanget,
påpekade just att: “att ni ibland använder monolog, där ni
ställer en fråga och sen svarar som "en vän", gör
att det känns mer begripligt och man får en känsla av att man
pratar och diskuterar med en kompis om diverse ämnen. Detta
ökar då förståelse enormt då det är annorlunda skrivet än hur
t.ex. föreläsare skriver sitt material.“ Denna och andra
kommentarer är starka argument för att denna typ av
läromaterial kan fungera bra som komplement till annan
undervisning.

%Vill vi ha med detta stycke? Känns lite malplacerat. Borde nog utvecklas mer elller tas bort.
%Eftersom projektet inledningsvis var relativt oklart definierat
%behövde projektgruppen även lägga mycket tid på att arbeta fram en
%produktspecifikation.

% På grund av projektgruppens bristande erfarenhet av denna typ av
% projektarbete uppstod problem med arbetets upplägg, rollfördelning
% och struktur. Detta var ett internt problem som vi identifierade och
% åtgärdade så fort vi kunde, men projektet hade saknat en tydlig
% projektledare i början av arbetet och på grund av att ämnet för vårt
% projekt var såpass oklart definierat så rörde sig projektet väldigt
% långsamt framåt i början. Det var ungefär efter att halva tiden för
% projektet hade passerat som vi lyckades identifiera var vår
% projektstruktur hade varit bristande och försökte åtgärda problemet
% med hjälp av en mer genomtänkt omstrukturering av arbetet och
% gruppen.

En stor del av utmaningen med detta projekt har varit att det inte
fanns någon uppenbar lösning på vad som skulle implementeras i
programmeringskoden. För varje område som läromaterialet tar upp har
projektgruppen behövt utvärdera vad som skulle ingå och vad som var
möjligt att implementera på en nivå som studenterna i målgruppen kan
förstå. Det har varit en ständig avvägning av att få med de mest
vitala delarna i \textit{TSS} och att implementera kod som är
begriplig. \textit{The Fastest Fourier Transform in the West}
\cite{fastestfourier} %Osäker på om denna referens är korrekt, kontrollera gärna
är ett exempel på hur Fouriertransform kan implementeras relativt
effektivt i Haskell, men problemet är att koden då hamnar på en nivå
som överstiger förmågan hos många studenter i målgruppen. Kod av detta
slag är alltså, oavsett hur effektiv den är, inte önskvärd i vårt
läromaterial eftersom vi vill använda koden för att förklara ett annat
ämne. Om koden är för svårläst riskerar vi att hamna i en situation
där studenten varken förstår programmeringen eller ämnet den
beskriver.

När man beslutat \textbf{vad} som ska implementeras uppstår frågan
\textbf{hur} det ska implementeras. Som nämndes i
% TODO: Byt ut alla dessa "handgjorda" referenser till riktiga
% LaTeX-referenser - kanske något i stil med mitt exempel nedan.
avsnitt \textit{\ref{sec:implDSL} - \nameref{sec:implDSL}}
finns det flera sätt att hantera detta. När man arbetar med signaler
finns det två grundtyper; diskreta signaler och kontinuerliga
signaler. Diskreta signaler kan implementeras relativt enkelt, medan
kontinuerliga signaler blir betydligt mer komplicerat. Projektgruppen
har under projektets gång granskat två olika metoder för att hanterar
kontinuerlig tid i programmeringskoden. En variant är att approximera
en kontinuerlig signal med en diskret. Använder man väldigt små steg
kan man få en relativt bra approximation på detta sätt, men det är
fortfarande endast en approximation och därför blir resultatet inte
fullständigt korrekt.
% TODO: "pattern matching" = mönstermatchning (ni kanske vill ange
% båda första gången ni använder ordet)
En annan variant är att använda \textit{pattern matching}, där
funktioner definieras exakt för ett antal givna insignaler som
definieras var för sig. Detta ger mer korrekta svar, men ger inte
samma möjlighet att

sammanfatta
%och abstrahera
området som en diskret approximation eftersom man definierar
olika fall var för sig. Denna
variant gör det å andra sidan möjligt att definiera typfall som
studenten kan känna igen i kursen \textit{TSS}.

Vad är då bäst i vårt fall, att använda en approximation som ger
större möjlighet att abstrahera
% och
sammanfatta kärnan i ämnet eller
en mer matematiskt korrekt variant som använder formler som studenten
kan känna igen? Detta är en fråga som projektgruppen inte har kunnat
ta fram ett generellt svar på. Projektgruppen har, efter mycket
diskussion, valt att använda båda varianter i läromaterialet. Vi ansåg
att det fanns stora för- och nackdelar med båda lösningarna och att de
argument som vägde tyngst berodde på vilket ämne som togs upp i det
aktuella avsnittet. Inom vissa områden avgjorde gruppen att det gav
större förståelse att sammanfatta och abstrahera, medan matematisk
exakthet och typfall gjorde detta bättre inom andra områden.

Som beskrivs i \textit{\ref{sec:didaktik} - \nameref{sec:didaktik}} är en av
utmaningarna när man utvecklar läromaterial att göra materialet
tilltalande utan att det blir ren underhållning. Detta innebär en
ständig balansgång när man utvecklar materialet. Å ena sidan vill man
att materialet ska vara så informativt och effektivt som möjligt. Å
andra sidan vill man fånga och behålla studentens intresse och
uppmuntra till vidare inlärning. Vi har i vårt material strävat efter
att beakta båda sidorna och försökt hålla en lättsam och lätt
humoristisk ton utan att överskugga det faktiska innehållet. Mer om
hur detta gått diskuteras i det kommande avsnittet \textit{\ref{sec:resDisk} - \nameref{sec:resDisk}}.

\subsection{Resultatdiskussion}
\label{sec:resDisk}
Den slutgiltiga produkten uppfyller alla delar av
produktspecifikationen med ett undantag. Ambitionen var att
läromaterialet skulle täcka in hela kursen \textit{TSS}. I nuläget
saknas dock avsnitt som beskriver Laplacetransform och
Z-transform. Som nämns i avsnittet \textit{Metod} prioriterades
detta bort på grund av tidsbrist i slutet av projektet. Projektgruppen
hamnade i läget att de kunde fokusera på antingen Fouriertransform
eller Laplacetransform och Z-transform. Anledningen till att
gruppen valde att fokusera på Fouriertransform var att den är nära
besläktad med Laplacetransform och Z-transform. Ur
programmeringssynpunkt var det även svårt att implementera
Laplacetransform och Z-transform utan att först implementera Fouriertransform. Det finns emellertid ett utkast
till avsnitt om Laplacetransform och Z-transform som kan utvecklas i framtiden.

%TODO: fixa \label ovan och \ref eller \nameref här
Som beskrivs i avsnittet \textit{\ref{sec:resDisk} - \nameref{sec:resDisk}} ställde sig testgrupppen
väldigt positiv till läromaterialet. Det är dock värt att påpeka att
testgruppen bestod av en relativt liten grupp, total 13 personer.
Även om testgruppens svar är en positiv indikation är det därför svårt
att dra några definitiva slutsatser om huruvida vårt läromaterial
uppnår sitt mål i nuläget.

Vidare är det så att flera personer i testgruppen går i samma årskurs
och på samma program som flera av medlemmarna i
projektgruppen. Återkopplingen kunde av praktiska skäl heller inte
vara helt anonym. Det är därför inte orimligt att anta att vissa
medlemmar i testgruppen kan ha varit partiska i sin bedömning och att
detta kan ha påverkat resultatet.

För en bättre utvärdering av läromaterialet krävs att fler studenter
inom målgruppen använder sig av det i samband med kursen
\textit{TSS}. Det kommer därför dröja innan man kan se det verkliga
resultatet. Detta kan dock vara svårt att utvärdera även då eftersom
det kan finnas många andra faktorer som kan påverka. Om en ökning i
antalet godkända studenter i kursen observeras i framtiden är det
svårt att säga med säkerhet i vilken utsträckning denna förändring
beror på vårt läromaterial.

%Stämmer vårt resultat överens med tidigare forskning? Har vi fått samma resultat som liknande projekt?

\subsection{Den sociala dimensionen}
%Social dimension - Bygga upp och bevara sociala institutioner och
%strukturer som är viktiga för mänskligt välbefinnande. T.ex så är det
%viktigt för universitet att upprätthålla kunskap och ideer som idag
%inte är så centrala men som kan vara viktigt i
%framtiden. Undervisningsorganisationer etc.

Ett läromaterial som lyckas få datastudenter att \emph{förstå}
innehållet i en kurs, som de annars skulle finna svår, kommer att leda
till positiva påverkan i den sociala dimensionen. %Kan någon hjälpa mig formulera om detta?

Det kortsiktiga resultatet från ett väl fungerande läromaterial är att
andelen godkända studenter i kursen höjs. Detta skulle i sin tur
medföra att välbefinnandet förbättras för studenterna som annars inte
skulle klara kursen.  Det skulle också möjligen medföra en minskad oro
hos de studenter som ännu inte läst kursen i fråga, eftersom den
statistiska chansen att klara kursen är större.

%---------------------OBS detta kanske är för dramatiskt....---------
Det långsiktiga målet för projektet DSLsofMath är dock att minska det
uppfattade gapet mellan matematik och programmering. %Kanske formuleras om?
Detta skulle kunna öppna upp för fler möjligheter att
tackla framtida utmaningar, eftersom det skulle utbildas fler
ingenjörer som inte är begränsade till enbart ett tankesätt.
%och väldigt optimistiskt....
Läromaterial är då i bästa fall ett \emph{social support} som hjälper
studenterna att få en bättre förståelse inom båda domänerna. Detta
skulle i så fall kunna både förbättra studenternas välbefinnande och
möjligen även de individer dessa framtida ingenjörer kan komma att
påverka i sitt arbete.

Det förefaller dock ytterst osannolikt att endast ett läromaterial
kan åstadkomma en tillräckligt stor effekt för att orsaka alla
dessa möjliga positiva konsekvenser. Dock förefaller det utifrån
våra testresultat och andra liknande studier som om ett
projekt som vårt kan vara ett bidrag till att minska
gapet mellan matematik och programmering.

\subsection{Framtida utveckling}
%* Hur kan man bygga vidare på detta?
%* 		Implementera fler transformer + Regler
%* 		Testa mer mot målgruppen
Vad återstår då att utveckla i framtiden? En möjlighet är givetvis att
vidareutveckla det befintliga läromaterialet. Till att börja med
skulle man kunna komplettera materialet med avsnitt om
Laplacetransform och Z-transform. Det finns, som tidigare nämnts,
påbörjade utkast för dessa och om de färdigställs täcker
läromaterialet in hela kursen \textit{TSS}.

En annat vidareutveckling av materialet skulle kunna vara utökad
implementation av funktioner i kontinuerlig tid. Som vi tidigare nämnt
i \textit{\ref{sec:metDisk} - \nameref{sec:metDisk}} är detta ett
relativt komplext problem som i vårt fall löstes med en
kompromiss. Detta är dock ett område där djupare studier säkerligen
skulle kunna ge bättre implementationer.

Om fler tester genomförs skulle materialet eventuellt kunna förbättras
ytterligare med dessa tester som underlag. En större testgrupp och
anonyma svar skulle kanske kunna ge fler och mer tillförlitliga
resultat.

Det skulle även kunna vara intressant att gå igenom materialet
tillsammans med experter inom \textit{TSS} samt Didaktik. Det är
mycket möjligt att experterna har en annan syn på materialet än
studenterna har och kan ge andra förslag på hur materialet kan
utvecklas.

Hela läromaterialet finns på \url{https://github.com/DSLsofMath/BScProj} och det är
fritt fram för vem som helst att arbeta vidare med det eller använda
det som inspiration för liknande projekt.

I vårt projekt valde vi att fokusera på kursen \textit{TSS} men som
beskrivs i bakgrunden är den bara en av problemkurserna på
Datatekniska programmet. En möjlighet skulle vara att i framtiden ta
fram ett liknande läromaterial även till kursen
\textit{Reglerteknik}. Ett sådant läromaterial skulle då kunna använda
sig av delar av den programmeringskod som tagits fram i vårt material,
eftersom \textit{Reglerteknik} till viss del bygger på innehållet i
\textit{TSS}.

Även andra kurser än de som tas upp i denna rapport skulle kanske
kunna gagnas av liknande läromaterial. Som vi nämner i
\textit{inledning} och \textit{\ref{sec:relForsk} - \nameref{sec:relForsk}}
finns liknande problem även inom andra områden.


\section{Slutsatser}
% Huvudpoänger:
% * Sammanfattar vad vi kommit fram till i diskussion och resultat
% * Vad är gjort och vad återstår att göra?

Projektets mål var att ta fram ett kompletterande läromaterial till
kursen \textit{TSS} inspirerat av \textit{DSLsofMath} som uppfyller
produktspecifikationen som beskrivs i avsnitt \textit{\ref{sec:prodSpec} - \nameref{sec:prodSpec}}.

Projektgruppen har utifrån detta tagit fram ett material som finns
samlat på en hemsida. Materialet innehåller förklarande text och exempel, bilder
samt programmeringsövningar. Det består av 4 delavsnitt
som i stora drag täcker in innehållet i kursen \textit{TSS}, med
undantag för Laplacetransform och Z-transform.

Enligt testgruppen är texten underhållande och informativ och
övningarna ligger på en lagom svår nivå. På grund av testgruppens
ringa storlek är det svårt att dra några definitiva slutsatser,
men testresultaten indikerar att materialet uppfyller de krav
som ställs i produktpecifikationen. Den faktiska kvalitén på
läromaterialet är även svår att bedöma eftersom det
ännu inte har använts i samband med kursen.

Sammanfattningsvis har projektgruppen alltså tagit
fram ett läromaterial som uppfyller kraven i produktspecifikationen,
med undantag av ett saknat delavsnitt. Testresultaten har varit
positiva men är inte nog för att dra definitiva slutsatser. Hur väl detta
material fyller sin funktion kommer förhppningsvis att
visa sig i framtiden när det används av studenter i målgruppen.

\newpage

%TODO: Lägg in länk till learn you a Haskell i bib-filen

\bibliographystyle{IEEEtran}
\bibliography{include/referenser}

\newpage

\section{Bilagor}
Bilaga 1 - Enkätundersökning

Bilaga 2 - Intervju med examiniatorer

Bilaga 3 - Utdrag ur läromaterialet \textit{TSS med DSL}
% * Utdrag ur tutorial?
% * Utdrag ur intervju med examinatorer?
% * Utdrag ur enkätsvar?
\subsection{Bilaga 1}
\label{bil:1}
Denna bilaga är ett utdrag ur den enkätundersökning som genomfördes av
kandidatgruppen under våren 2015 via \textit{webbenkater.com}. Bilagan
innehåller endast de delar av undersökningen som bedömts som relevanta
för rapporten och är tömd på all information som kan länkas till
enskilda individer.

Observera att antalet deltagare varierar markant. Detta beror på att
 fråga 1 och 9 riktade sig även till studenter som ska läsa kursen nästa
år, medans övriga frågor endast riktar sig till studenter som redan läst
kursen.



Samtliga bilder i bilagan kommer från
\textit{webbenkater.com}.
%TODO: saknas
%infoga bilaga 1
\newpage

\subsection{Bilaga 2}
\label{bil:exam_intervju}
Denna bilaga är en sammanfattning av den intervju projektgruppen höll
med Bo Egardt och Ants Silberberg. Bo och Ants har under flera år
varit både lärare och examinatorer i \textit{Reglerteknik} respektive
\textit{TSS} på Chalmers. Sammanfattningen är granskad och godkänd av
både Bo och Ants.
%TODO: saknas
%infoga bilaga 2
\newpage

\subsection{Bilaga 3}
\label{bil:3}
Denna bilaga är ett utdrag ur det läromaterial projektgruppen
utvecklat. Utdraget beskriver udda och jämna signaler och är hämtat
från avsnitt 2 i läromaterialet.
%TODO: saknas
%infoga bilaga 3

\end{document}
