\documentclass[]{article}

\usepackage[swedish]{babel}
\usepackage[utf8]{inputenc}
\usepackage[T1]{fontenc}
\usepackage[backend=biber]{biblatex}
\usepackage{hyperref,url,bbm,graphicx,icomma,units,lmodern,amsmath}

\addbibresource{referenser.bib}

\date{\today}

\begin{document}
\renewcommand*\abstractname{Sammandrag}

\begin{titlepage} \newcommand{\HRule}{\rule{\linewidth}{0.3mm}}
\center
\textsc{\Large Chalmers tekniska högskola}\\[0.05cm]
\normalsize \today

\HRule \\[0.08cm]
{\large  Programmering som undervisningsverktyg för signaler och system \\ \normalsize{Slutrapport}}\\[0.08cm] %Signalteori är för stort?
\HRule \\[0.3cm]

\vfill

\begin{flushleft} \small
    \emph{Författare: \\
    \quad Filip Lindahl\\
    \quad Cecilia Rosvall\\
    \quad Peter Ngo\\
    \quad Jacob Jonsson\\
    \quad Joakim Olsson\\}
\end{flushleft}
\end{titlepage}
\newpage

\begin{abstract}
%En Sammandrag
\end{abstract}
\newpage



\newpage
\tableofcontents

\newpage

\setlength{\parskip}{2mm}
\setlength{\parindent}{0pt}

%\section{Abstract}

\section{Sammanfattning}

% Huvudpoänger:
% * Vad är det här för projekt? (Så att läsaren kan bedöma om hen vill läsa rapporten)
% * Jättekort om syfte, resultat etc
% Illustration: Borde inte behövas
Denna rapport beskriver utvecklingen av tutorialen “TSS och Reglerteknik med DSL” med tillhörande programmeringskod och lösningar. Tutorialen riktar sig till studenter på det datatekniska programmet på Chalmers och har som syfte att vara ett kompletterande läromedel inom Signallära och Reglerteknik där man använder programmering med domänspecifika språk.

Bakgrunden till detta projekt är att kurserna TSS och Reglerteknik på det aktuella programmet under många år har haft väldigt dåliga studieresultat på grund av studenternas ovana att hantera den typ av matematik som utgör grundstommen i kurserna.

Rapporten beskriver vidare de undersökningar och efterforskningar som gjorts för att ta kunna ta fram tutorialen, samt de tester som gjorts på den färdiga produkten.

\section{Ordlista}
DSL - Domain Specific Language eller Domän specifik språk
GPL - General Purpose Language t.ex C, C++, Java, Haskell m.m.
TSS - Transformer, Signal och System
Regler - Reglerteknik

\section{Inledning}

\subsection{Bakgrund}

% Huvudpoänger:
% * Problem med kurserna på data
% * Var kommer vårt projekt ifrån? Spinoff till DSLOfMathprojektet.
%
% Illustrationer:
% * Statisktik från kurserna

På Chalmers Tekniska Högskola har det funnits en lång trend där
studenterna på utbildningen för Datateknik har haft svårigheter
med kurserna \textit{Transformer, Signaler och System}, hädanefter omnämnd
som ''TSS'', och \textit{Reglerteknik}.
Dessa svårigheter tror examinatorerna i kurserna beror till stor del på att
studenterna inte är tillräckligt bekväma i det matematiska språket.
Studenterna på Datateknik har inte tillräcklig vana att tolka innebörden av de matematiska beskrivningarna och är allmänt ovana vid matematiskt hantverk.
Därför lägger datastudenterna det mesta av sin energi på att få ihop matematiken
rent praktiskt, istället för att se vad matematiken står för.

Detta syns bland annat i statistiken för hur många datastudenter som
klarar kurserna, under perioden från 2010 till 2016 blev i genomsnitt 51\% av
alla som skrev tentamen godkända i TSS och under samma period blev 53\% godkända
i Reglerteknik \cite{tentastatistik}. Det finns även ett mörkertal i dessa siffror
eftersom inte alla studenter skriver tentan, exempelvis valde 48 studenter utav 122
registrerade att inte tentera Reglerteknik 2014 \cite{kursinformation:ere102:14-15}.

För att minska svårigheterna som studenterna har för dessa kurser påbörjades
ett pedagogiskt projekt, DSLsOfMath, som hittills resulterat i kursen
Matematikens domänspecifika språk, där man använder funktionell
programmering för att beskriva matematiska problem \cite{kursplan:dslsofmath}.
Resonemanget bakom detta var att funktionell programmering är ett verktyg
som datastudenterna har erfarenhet av och som använder en notation med
fokus på tydlighet som lämpar sig relativt väl för att lära ut matematiska
begrepp \cite{tfpie}.

Detta projekt uppstod som en avgrening av DSLsOfMath-projektet med fokus mer på signal- och systemteoretiska färdigheter snarare än en allmän matematisk grund.

\subsection{Syfte}

% Huvudpoänger:
%
% Illustration:

Syftet med projektet är att underlätta för studenter inom datateknik att
ta till sig signal- och systemteoretiska ämnen genom att utnyttja deras
kunskaper inom programmering och även göra det möjligt att betrakta ämnet ur
en programmerares perspektiv. Tanken bakom detta är att man ska göra gapet
mellan datateknik och signalteori mindre och minska problemen som nämns
i bakgrundsavsnittet.

Projektet är tänkt att resultera i en kompletterande undervisningshandledning som 
framöver i rapporten kommer kallas handledning. 
Denna handledning fungera som ett komplement till de kurser som ges inom
signalteori på den datatekniska grundutbildningen. Den ska innehålla
förklaringar och programmeringsövningar som är anpassade för studenter på den
datatekniska utbildningen på Chalmers tekniska högskola.

\section{Teori - Teknisk bakgrund om DSL mm}

% Allmänt om DSL
Ett DSL är ett programmeringsspråk som till skillnad från ett GPL, är anpassat
för endast en specifik domän. De har alltid funnits sedan programmering
skapades och exempel på DSL är HTML, MatLab och rpgmaker. Ett DSL kan
byggas upp från grunden eller vara utvecklat från ett redan existerande GPL
som Haskell. Fördelen med detta är att de erbjuder mer specialiserade egenskaper
inom detta specifika område men är inte avsedda för att kunna appliceras till andra
områden. Därför passar DSL utmärkt till detta kandidatarbete eftersom målet är
att lära ut TSS via programmering.

\section{Produkt/kravspec}

\section{Avgränsningar}

% Huvudpoänger:
% * Vad tas inte med i vårt projekt och varför?
%
% Illustrationer: Borde inte behövas

Det kommer endast fokuseras på de områden studenterna finner svårt inom
kurserna TSS och Reglerteknik. Dessutom tas inte bevis med i denna tutorial.

I själva impementationen av DSL så undviks funktioner som är
ovanliga eller kan skrivas explicit. t.ex så ska fouriertransformen av
$sinh(x)$ inte implementeras eftersom $sinh(x)$ kan utryckas via dess
definition $\frac{e^{x} - e^{-x}}{2}$ istället.
Dessutom undviks onödigt komplexa funktioner som inte
dyker upp naturligt som en signal, t.ex $sin(cos(x)+e^x)$.

Slutligen så visade det sig att det var för ont om tid för att hinna
skriva en handledning och tillhörande DSL för både områdena Reglerteknik och TSS. 
Därför fokuserades det endast på TSS.
%Kanske borde bara skriva om rapporten så att vi inte ens nämner reglerteknik?

\section{Metod och Genomförande}

% Det vi har gjort:
% * Förstudier
% * Enkätundersökning
%   * Hänvisa till böckerna ``Enkäten i praktiken'' och ``Enkätboken''
% * Skriva tutorial

I det här kapitlet beskrivs den tänkta tillvägagången till hur denna
handledning ska tillverkas.

\subsection{Litteraturstudier och Förundersökning (Temporär)}

% Detta ska ha underrubrikerna Förstudier och typer av undersökning

Projektdeltagarna skulle först återuppfriska kunskaperna om Haskell, TSS
och Reglerteknik. Därefter genomfördes vidare studier om DSL och pedagogik för
att kunna implementera formlerna i Haskell.

Efter förstudien så intervjuade vi de kursansvariga över deras tankar om vad
studenterna tyckte var svårt i kursen. Det ska även påbörjas en enkätundersökning
som vi tänker skicka ut till studenterna som går i årskurs 2 och 3 inom Datateknik.

% Ska litteratur tas med?

\subsection{Uppställning av tutorial}


Själva handledningen är uppdelad i fem kapitel. Det första kapitlet
ska introducera konceptet DSL samt förklara kortfattat om komplexa tal
med Eulers formel och lite om signalers egenskaper.

Därefter kommer kapitlarna om LTI-system, Fourierserier och Fouriertransform
samt Laplace- och Z-transform.

I varje kapitel ställde vi upp de definitioner och formler som tycktes
vara allmänt viktiga med tillhörande förklarande text. Den tillhörande
texten ska vara inspirerad av boken ``Learn You a Haskell for Great
Good'' av Miran Lipovača \cite{learnyouahaskell} och texter från sidan
\url{http://betterexplained.com/}.

Det vill säga, när vi förklarar hur formlerna fungerar så utnyttjar vi
oss av exempel som inte nödvändigtvis är helt korrekta. Målet är
att den ovanstående bilden ger en god uppfattning på hur formlerna kan användas.
Sedan förklarar vi hur det kan appliceras inom TSS i en mer matematiskt korrekt form.
Därefter tänker vi visa hur dessa definitioner eller formler skulle se ut i vårt DSL.

Dessutom ska det finnas exempeluppgifter till de områden studenterna finner
svåra och formulera lite uppgifter som ska kunna lösas via programmering.
Lösningarna som finns med ska visa hur uppgiften kan lösas både med den vanliga
analytiska metoden och via användandet av DSL.

\subsection{Implementation i DSL}

%Förklara hur komplexa tal och alla typer av transformer implementerades i DSL.
Vårt DSL ska vara uppbyggt från Haskell och ska täcka områdena TSS och Reglerteknik.
I båda kurserna så används komplexa tal vilket innebär att det först måste implementeras i vårt DSL.

\subsection{Verktyg}%behövs denna?

\subsection{Testning av handledningen}

Vi har ett tjugotal studenter som är villiga att testa vår produkt för
att se om den är lika hjälpsam som det är tänkt. Med responsen vi får
från dessa studenter kommer vi att göra förändringar för att få den mer
användbar och vara till mer hjälp. Vid testningen är tanken att
produkten nästan ska vara färdig så att vi får så viktig information som
möjligt.

\section{Resultat}

% Huvudpoänger:
% * Vad blev det för produkt? Tutorial på hemsida med 5 delavsnitt?
%
% Illustrationer:
% * Utdrag ur tutorialen

\section{Diskussion}

% Huvudpoänger:
% * Har vi löst problemet?
% * Går problemet att lösa?
% * Vad är i grund och botten orsaken till problemet?
% * Vilka problem har dykt upp?

\section{Slutsatser}

\section{Avslutning}

% Huvudpoänger:
% * Vad är gjort och vad återstår att göra?
%
% Illustration: Borde inte behövas

\newpage

\printbibliography

\end{document}
