\documentclass[]{article}

\usepackage[swedish]{babel}
\usepackage[utf8]{inputenc}
\usepackage[T1]{fontenc}
\usepackage[backend=biber]{biblatex}
\usepackage{hyperref,url,bbm,graphicx,icomma,units,lmodern,amsmath,glossaries}

\addbibresource{referenser.bib}
\loadglsentries{ordlista.tex}

\date{\today}

%Ordlistan skapas här.
%Måste refereras till något med \gls{ref} för att kunna visas.
\makeglosarries

\begin{document}

\begin{titlepage} \newcommand{\HRule}{\rule{\linewidth}{0.3mm}}
\center
\textsc{\Large Chalmers tekniska högskola}\\[0.05cm]
\normalsize \today

\HRule \\[0.08cm]
{\large  Programmering som undervisningsverktyg för signaler och system \\ \normalsize{Slutrapport}}\\[0.08cm] %Signalteori är för stort?
\HRule \\[0.3cm]

\vfill

\begin{flushleft} \small
    \emph{Författare: \\
    \quad Filip Lindahl\\
    \quad Cecilia Rosvall\\
    \quad Peter Ngo\\
    \quad Jacob Jonsson\\
    \quad Joakim Olsson\\}
\end{flushleft}
\end{titlepage}
\newpage
\renewcommand*\abstractname{Sammandrag}
\begin{abstract}
%Ett Sammandrag
%Subject to Change!--------------------------------------------
% Huvudpoänger:
% * Vad handlar rapporten om? Vad är det för projekt vi beskriver utvecklingen av? (Läsare ska här kunna bedöma om rapporten är intressant att läsa för dem)
% * Jättekort om syfte, resultat etc
% Illustration: Borde inte behövas
Denna rapport beskriver utvecklingen av läromaterialet “TSS med DSL” med dess tillhörande programmeringskod och lösningar. Läromaterialet riktar sig till studenter på det datatekniska programmet på Chalmers och har som syfte att vara ett kompletterande läromedel inom signallära där man utnyttjar studenternas kunskaper inom programmering. Materialet innehåller förklarande text och teori, samt programmeringsövningar med domänspecifika språk.

Bakgrunden till detta projekt är att kurserna TSS och Reglerteknik på det aktuella programmet under många år har haft väldigt dåliga studieresultat på grund av studenternas ovana att hantera den typ av matematik som utgör grundstommen i kurserna. Detta har visats sig inte vara något som är unikt för datastudenterna på Chalmers utan det är tvärtom ett problem inom många discipliner.

Rapporten beskriver vidare de undersökningar och efterforskningar som gjorts för att ta kunna ta fram läromaterialet, samt de tester som gjorts på den färdiga produkten.
\end{abstract}
\renewcommand*\abstractname{Abstract}
\begin{abstract}
This report describes the development of the supplementary tutorial ''Transforms, Signals and Systems and Control Theory with the help of a DSL'' with provided code and solutions. The Tutorial is aimed at students studying Computer Science at Chalmers University of Technology and its aim is to be a supplementary teaching material for the courses ''Transforms, Signals and Systems'' as well as ''Control Theory'' where you will be progamming in domain specific languages.

The background to this project is that the computer science students studying the courses ''Transforms, Signals and Systems'' and ''Control Theory'' have gotten poor results on the tests for several years and this might be because of the CS students unfamiliarity with the kind of mathematics which is the backbone(?) of those courses.

The report further describes the surveying and analysis done to be able to produce this tutorial along with the tests that have been done on the finished product.
\end{abstract}
\newpage



\newpage
\tableofcontents

\newpage

\printglossary[title=Ordlista,nonumberlist]

\newpage

\setlength{\parskip}{2mm}
\setlength{\parindent}{0pt}

%\section{Abstract}

%\section{Sammanfattning} Flyttade denna till sammandrag

%\section{Ordlista} Ska flyttas till en ordentlig ordlista
%DSL - Domain Specific Language eller Domänspecifikt språk
%GPL - General Purpose Language t.ex C, C++, Java, Haskell m.m.
%TSS - Transformer, Signal och System
%Regler - Reglerteknik

\section{Inledning}
Vi kommer nu inledningsvis att gå igenom bakgrunden till projektet och hur det uppstod, samt beskriva projektets mål och rapportens syfte.
\subsection{Bakgrund}

% Huvudpoänger:
% * Problem med kurserna på data
% * Var kommer vårt projekt ifrån? Spinoff till DSLOfMathprojektet.
% * Ev. ta upp i korthet att detta är en del av ett större problem. Hänvisa till TFPIE-artikelns källor.
%
% Illustrationer:
% * Statisktik från kurserna

På Chalmers Tekniska Högskola finns det en lång trend där
studenterna på utbildningen för Datateknik har haft svårigheter
med kurserna \textit{Transformer, Signaler och System}, hädanefter omnämnd
som ''\gls{TSS}'', och \textit{Reglerteknik}.
Dessa svårigheter tror kursernas examinatorer beror till stor del på att
studenterna inte är tillräckligt bekväma i det matematiska språket.
Studenterna på Datateknik har inte tillräcklig vana att tolka innebörden av de matematiska beskrivningarna och är allmänt ovana vid matematiskt hantverk.
Därför lägger datastudenterna det mesta av sin energi på att få ihop matematiken
rent praktiskt, istället för att se vad matematiken står för.

Detta syns bland annat i statistiken för hur många datastudenter som
klarar kurserna, under perioden från 2010 till 2016 blev i genomsnitt 51\% av
alla som skrev tentamen godkända i TSS och under samma period blev 53\% godkända
i Reglerteknik \cite{tentastatistik}. Det finns även ett mörkertal i dessa siffror
eftersom inte alla studenter skriver tentan, exempelvis valde 48 studenter utav 122
registrerade att inte tentera Reglerteknik år 2014 \cite{kursinformation:ere102:14-15}.

För att minska svårigheterna som studenterna har för dessa kurser påbörjades
ett pedagogiskt projekt, DSLsOfMath, som hittills resulterat i kursen
Matematikens domänspecifika språk, där man använder funktionell
programmering för att beskriva matematiska problem \cite{kursplan:dslsofmath}.
Resonemanget bakom detta var att funktionell programmering är ett verktyg
som datastudenterna har erfarenhet av och som använder en notation med
fokus på tydlighet som lämpar sig relativt väl för att lära ut matematiska
begrepp \cite{TFPIE15_DSLsofMath_IonescuJansson}.

Vårt projekt uppstod som en avgrening från DSLsOfMath-projektet med fokus på att utveckla material till specifika kurser snarare än en allmän matematisk grund.

\subsection{Rapportens syfte}

% Huvudpoänger:
% *Vad vill vi säga med rapporten? Vad vill vi visa på?
% *Vad har vi försökt göra  i korthet?

% Illustration:

Syftet med denna rapport är att beskriva utvecklingen av handledningen “TSS med DSL”, som är ett kompletterande läromaterial till kursen TSS för studenter på datatekniska programmet på Chalmers, samt beskriva hur det bakomliggande problemet kan kopplas till en mer generell problematik.

\subsection{Projektets mål}
% Huvudpoänger:
% *Beskriv projektets syfte och mål, vad ska vi åstadkomma för produkt?
Syftet med projektet är att underlätta för studenter inom datateknik att ta till sig signal- och systemteoretiska ämnen genom att utnyttja deras kunskaper inom programmering och även göra det möjligt att betrakta ämnet ur en programmerares perspektiv. Tanken bakom detta är att man ska göra gapet mellan datateknik och signalteori mindre och minska problemen som nämns i bakgrundsavsnittet.

Projektet är tänkt att resultera i ett läromaterial som kan fungera som ett komplement till den kurs som ges inom signalteori på den datatekniska grundutbildningen. Detta läromaterial ska innehålla förklaringar och programmeringsövningar som är anpassade för studenter på den datatekniska utbildningen på Chalmers tekniska högskola.

\section{Avgränsningar}

% Huvudpoänger:
% * Vad tas inte med i vårt projekt och varför?
%
% Illustrationer: Borde inte behövas

Det kommer endast fokuseras på de områden studenterna finner svårt inom
kurserna TSS och Reglerteknik. Dessutom tas inte bevis med i denna tutorial.

I själva impementationen av DSL så undviks funktioner som är
ovanliga eller kan skrivas explicit. t.ex så ska fouriertransformen av
$sinh(x)$ inte implementeras eftersom $sinh(x)$ kan utryckas via dess
definition $\frac{e^{x} - e^{-x}}{2}$ istället.
Dessutom undviks onödigt komplexa funktioner som inte
dyker upp naturligt som en signal, t.ex $sin(cos(x)+e^x)$.

Slutligen så visade det sig att det var för ont om tid för att hinna
skriva en handledning och tillhörande DSL för både områdena Reglerteknik och TSS.
Därför fokuserades det endast på TSS.
%Kanske borde bara skriva om rapporten så att vi inte ens nämner reglerteknik?

\section{Problemanalys}
%Huvudpoänger:
% *Vad är det för problem projektet ska lösa?
% *Ev. hänvisa kort till relaterad forskning

\section{Produktspecifikation}
%Huvudpoänger:
% *Beskriv produktbeskrivningen vi fick från början
% *Bena ut i detalj hur produktbeskrivningen vi tagit fram ser ut och kommentera vilka val vi har gjort.
%Illustration: Ev bild på utdrag från tutorial för att tydliggöra upplägget

\section{Teori - Teknisk bakgrund om DSL mm}

\subsection{DSL}
% Allmänt om DSL
Ett DSL är ett programmeringsspråk som till skillnad från ett \gls{GPL}, är anpassat
för endast en specifik domän. De har alltid funnits sedan programmering
skapades och exempel på DSL är HTML, MatLab och rpgmaker. Ett DSL kan
byggas upp från grunden eller vara utvecklat från ett redan existerande GPL
som Haskell. Fördelen med detta är att de erbjuder mer specialiserade egenskaper
inom detta specifika område men är inte avsedda för att kunna appliceras till andra
områden. Därför passar DSL utmärkt till detta kandidatarbete eftersom målet är
att lära ut TSS via programmering.

\subsection{Funktionell programmering}

\subsection{Signallära}

\subsection{Didaktik}

\subsection{Relaterad forskning}


\section{Metod och Genomförande}

% Huvudpoänger:
% *Förstudier, intervjuer mm
% *Enkätundersökning
% 	*Hänvisa till böckerna “Enkäten i praktiken” och “Enkätboken”
% *Didaktik, hur vi skrivit vår tutorial och varför
% 	*Hänvisa till “Didaktik för ingenjörer” och “Motivational Design for Learning and Performance”
% *TSS, hur vi har tacklat ämnet
% *Funktionell programmering och dsl, hur vi valt att implementera det hela och varför
% *Testning



I detta kapitlet beskrivs den tänkta tillvägagången till hur denna
handledning tillverkast.

\subsection{Litteraturstudier och Förundersökning (Temporär)}

% Detta ska ha underrubrikerna Förstudier och typer av undersökning

Projektdeltagarna skulle först återuppfriska kunskaperna om Haskell, TSS
och Reglerteknik. Därefter genomfördes vidare studier om DSL och pedagogik för
att kunna implementera formlerna i Haskell.

Efter förstudien så intervjuade vi de kursansvariga över deras tankar om vad
studenterna tyckte var svårt i kursen. Det ska även påbörjas en enkätundersökning
som vi tänker skicka ut till studenterna som går i årskurs 2 och 3 inom Datateknik.

% Ska litteratur tas med?

\subsection{Uppställning av handledning}

Handledningen är uppdelad i sex kapitel %planerat
där den första kapitlet ska vara en introduktion till koncepted DSL. Därefter ska det finnas
ett kapitel som kort förklarar om komplexa tal med Eulers formel samt olika typer av
signaler och deras egenskaper.

%Därefter kommer kapitel om LTI-system, Fourierserier och Fouriertransform
%samt Laplace- och Z-transform.

I varje kapitel ställde vi upp de definitioner och formler som tycktes
vara allmänt viktiga med tillhörande förklarande text. Den tillhörande
texten ska vara inspirerad av boken ``Learn You a Haskell for Great
Good'' av Miran Lipovača \cite{learnyouahaskell} och texter från sidan
\url{http://betterexplained.com/}.

Det vill säga, när vi förklarar hur formlerna fungerar så utnyttjar vi
oss av exempel som inte nödvändigtvis är helt korrekta. Målet är
att den ovanstående bilden ger en god uppfattning på hur formlerna kan användas.
Sedan förklarar vi hur det kan appliceras inom TSS i en mer matematiskt korrekt form.
Därefter tänker vi visa hur dessa definitioner eller formler skulle se ut i vårt DSL.

Dessutom ska det finnas exempeluppgifter till de områden studenterna finner
svåra och formulera lite uppgifter som ska kunna lösas via programmering.
%Lösningarna som finns med ska visa hur uppgiften kan lösas både med den vanliga
%analytiska metoden och via användandet av DSL. -Inte något vi har gjort.

\subsection{Implementation i DSL}

%Förklara hur komplexa tal och alla typer av transformer implementerades i DSL.
Vårt DSL ska vara uppbyggt från Haskell och ska täcka områdena TSS och Reglerteknik.
I båda kurserna så används komplexa tal vilket innebär att det först måste implementeras i vårt DSL.

\subsection{Verktyg}%behövs denna?
%Kanske skriva om hur vi använder google drive?
%Jag skulle säga att det bara är värt att ta upp verktyg som påverkar resultatet, t.ex. att vi skrivit koden i haskell, men att vi t.ex. arbetat fram texten i google drive känns inte så relevant att ha med här.
Gruppen ska använda sig av Google Documents för att skriva den första utkasten av handledningen.
Efter faktan och strukturen för varje kapitel blev färdigställd så överfördes det till
\url{https://www.sharelatex.com/}, en webbsida som erbjuder en gemensam latex kompilator.
Tillslut så ska den överföras till en HTML format som är slutprodukten.

Matlab och \url{fooplot.com/} används till att skapa figurer på funktions kurvor eller vektor
värden ifall det behövs till handledningen. För enkätundersökningen så används webbsidan
\url{https://www.webbenkater.com/}.

Koden ska vara skriven i programmeringspråket Haskell och gruppen använde sig av \url{https://c9.io/}
för att alla ska kunna ha tillgång till koden och en kompilator.
Pågående under arbetets gång så laddas all någorlunda färdiga material upp i Github.
Här kan gruppens handledare kolla igenom all produkt och ge sina kommentarer.

%mer?

\subsection{Test av handledning}

Vi har ett tjugotal studenter som är villiga att testa vår produkt för
att se om den är lika hjälpsam som det är tänkt. Med responsen vi får
från dessa studenter kommer vi att göra förändringar för att få den mer
användbar och vara till mer hjälp. Vid testningen är tanken att
produkten nästan ska vara färdig så att vi får så viktig information som
möjligt.

\section{Resultat}

% Huvudpoänger:
% *Vad blev det för produkt? Tutorial på hemsida med 6 delavsnitt?
% *Hänvisa till hemsidan
% *Vad tyckte testgruppen?
% 	*Citat från testgruppen

%
% Illustrationer:
% * Utdrag ur tutorialen

\section{Diskussion}

% Huvudpoänger:
% * Har vi löst problemet?
% * Går problemet att lösa?
% * Vad är i grund och botten orsaken till problemet?
% * Vilka problem har dykt upp?
% * Vilka problem kan man undvika om man gör ett liknande arbete?
% * Stämmer vårt resultat överens med tidigare forskning?
% * Hur kan man bygga vidare på detta?


%Problem med hur vi implementerar integraler och annat kontinuerliga fall

Från intervjun förklarade examinatorerna att den största svårigheten studenterna fann i kursen
var kopplingen mellan matematiken och den bakomliggande fysiken. %Hur tacklade vi denna problem?

%Ett problem var själva storleken på uppgiften gruppen tog sig an. För att kunna skriva en
%bra handledning så krävs det goda förkunskaper om både ämnet och pedagogik.

%Ta även upp varför reglerteknik inte togs med i projektet.

\section{Slutsatser}
% Huvudpoänger:
% * Sammanfattar vad vi kommit fram till i diskussion och resultat

\section{Avslutning}
% Huvudpoänger:
% * Vad är gjort och vad återstår att göra?
%
% Illustration: Borde inte behövas

\newpage

\printbibliography

\section{Bilagor}
% * Utdrag ur tutorial?
% * Utdrag ur intervju med examinatorer?
% * Utdrag ur enkätsvar?

\end{document}
