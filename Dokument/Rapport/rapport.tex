\documentclass[]{article}

\usepackage[swedish]{babel}
\usepackage[utf8]{inputenc}
\usepackage[T1]{fontenc}
\usepackage[backend=biber]{biblatex}
\usepackage{hyperref,url,bbm,graphicx,icomma,units,lmodern,amsmath,glossaries}

\addbibresource{referenser.bib}
\loadglsentries{ordlista.tex}

\date{\today}

%Ordlistan skapas här.
%Måste refereras till något med \gls{ref} för att kunna visas. 
\makeglosarries

\begin{document}

\begin{titlepage} \newcommand{\HRule}{\rule{\linewidth}{0.3mm}}
\center
\textsc{\Large Chalmers tekniska högskola}\\[0.05cm]
\normalsize \today

\HRule \\[0.08cm]
{\large  Programmering som undervisningsverktyg för signaler och system \\ \normalsize{Slutrapport}}\\[0.08cm] %Signalteori är för stort?
\HRule \\[0.3cm]

\vfill

\begin{flushleft} \small
    \emph{Författare: \\
    \quad Filip Lindahl\\
    \quad Cecilia Rosvall\\
    \quad Peter Ngo\\
    \quad Jacob Jonsson\\
    \quad Joakim Olsson\\}
\end{flushleft}
\end{titlepage}
\newpage

\renewcommand*\abstractname{Sammandrag}
\begin{abstract}
%Ett Sammandrag
%Subject to Change!--------------------------------------------
% Huvudpoänger:
% * Vad handlar rapporten om? Vad är det för projekt vi beskriver utvecklingen av? (Läsare ska här kunna bedöma om rapporten är intressant att läsa för dem)
% * Jättekort om syfte, resultat etc
% Illustration: Borde inte behövas
Denna rapport beskriver utvecklingen av läromaterialet “TSS med DSL” med dess tillhörande programmeringskod och lösningar. Läromaterialet riktar sig till studenter på det datatekniska programmet på Chalmers och har som syfte att vara ett kompletterande läromedel inom signallära där man utnyttjar studenternas kunskaper inom programmering. Materialet innehåller förklarande text och teori, samt programmeringsövningar med domänspecifika språk.

Bakgrunden till detta projekt är att kurserna TSS och Reglerteknik på det aktuella programmet under många år har haft väldigt dåliga studieresultat på grund av studenternas ovana att hantera den typ av matematik som utgör grundstommen i kurserna. Detta har visats sig inte vara något som är unikt för datastudenterna på Chalmers utan det är tvärtom ett problem inom många discipliner.

Rapporten beskriver vidare de undersökningar och efterforskningar som gjorts för att ta kunna ta fram läromaterialet, samt de tester som gjorts på den färdiga produkten.
\end{abstract}

\renewcommand*\abstractname{Abstract}
\begin{abstract}
This report describes the development of the supplementary tutorial ''Transforms, Signals and Systems and Control Theory with the help of a DSL'' with provided code and solutions. The Tutorial is aimed at students studying Computer Science at Chalmers University of Technology and its aim is to be a supplementary teaching material for the courses ''Transforms, Signals and Systems'' as well as ''Control Theory'' where you will be progamming in domain specific languages.

The background to this project is that the computer science students studying the courses ''Transforms, Signals and Systems'' and ''Control Theory'' have gotten poor results on the tests for several years and this might be because of the CS students unfamiliarity with the kind of mathematics which is the backbone(?) of those courses. 

The report further describes the surveying and analysis done to be able to produce this tutorial along with the tests that have been done on the finished product.
\end{abstract}
\newpage



\newpage
\tableofcontents

\newpage

\printglossary[title=Ordlista,nonumberlist]

\newpage

\setlength{\parskip}{2mm}
\setlength{\parindent}{0pt}


\section{Inledning}
Vi kommer nu inledningsvis att gå igenom bakgrunden till projektet och hur det uppstod, samt beskriva projektets mål och rapportens syfte.

\subsection{Bakgrund}
% Huvudpoänger:
% * Problem med kurserna på data
% * Var kommer vårt projekt ifrån? Spinoff till DSLOfMathprojektet.
% * Ev. ta upp i korthet att detta är en del av ett större problem. Hänvisa till TFPIE-artikelns källor.
%
% Illustrationer: Borde inte behövas

På Chalmers Tekniska Högskola finns det en lång trend där
studenterna på utbildningen för Datateknik har haft svårigheter
med kurserna \textit{Transformer, Signaler och System}, hädanefter omnämnd
som ''\gls{TSS}'', och \textit{Reglerteknik}.
Dessa svårigheter tror kursernas examinatorer beror till stor del på att
studenterna inte är tillräckligt bekväma i det matematiska språket.
Detta syns bland annat i statistiken för hur många datastudenter som
klarar kurserna. 

För att minska svårigheterna som studenterna har för dessa kurser påbörjades
ett pedagogiskt projekt, DSLsOfMath, som hittills resulterat i kursen
Matematikens domänspecifika språk, där man använder funktionell
programmering för att beskriva matematiska problem \cite{kursplan:dslsofmath}.
Resonemanget bakom detta var att funktionell programmering är ett verktyg
som datastudenterna har erfarenhet av och som använder en notation med
fokus på tydlighet som lämpar sig relativt väl för att lära ut matematiska
begrepp \cite{TFPIE15_DSLsofMath_IonescuJansson}.

Vårt projekt uppstod som en avgrening från DSLsOfMath-projektet med fokus på att utveckla material till specifika kurser snarare än en allmän matematisk grund.

\subsection{Rapportens syfte}

% Huvudpoänger:
% *Vad vill vi säga med rapporten? Vad vill vi visa på?
% *Vad har vi försökt göra  i korthet?

% Illustration:

Syftet med denna rapport är att beskriva utvecklingen av handledningen “TSS med DSL”, som är ett kompletterande läromaterial till kursen TSS för studenter på datatekniska programmet på Chalmers, samt beskriva hur det bakomliggande problemet kan kopplas till en mer generell problematik. 

\subsection{Projektets mål}
% Huvudpoänger: 
% *Beskriv projektets syfte och mål, vad ska vi åstadkomma för produkt?
Syftet med projektet är att underlätta för studenter inom datateknik att ta till sig signal- och systemteoretiska ämnen genom att utnyttja deras kunskaper inom programmering och även göra det möjligt att betrakta ämnet ur en programmerares perspektiv. Tanken bakom detta är att man ska göra gapet mellan datateknik och signalteori mindre och minska problemen som nämns i bakgrundsavsnittet.

Projektet är tänkt att resultera i ett läromaterial som kan fungera som ett komplement till den kurs som ges inom signalteori på den datatekniska grundutbildningen. Detta läromaterial ska innehålla förklaringar och programmeringsövningar som är anpassade för studenter på den datatekniska utbildningen på Chalmers tekniska högskola.

\section{Avgränsningar}
% Huvudpoänger:
% * Vad behandlar vi inte i rapporten och varför?
% * Vad tas inte med i vårt projekt och varför?
% Illustrationer: Borde inte behövas

I projektet har gruppen avgränsat sig till att endast ta fram läromaterial till endast en kurs, TSS. Detta då projektet är tidsbegränsat. Läromaterialet som tagits fram är ett kompletterande material och är inte tänkt att kunna ersätta kursen eller dess material.

Gruppen har under utvecklingen av materialet inte heller lagt fokus på matematisk korrekthet, matematiska bevis, avancerade matematiska begrepp eller teorier.  Anledningen till detta är att materialet  i första hand är tänkt att främja förståelse och väcka intresse för ämnet. 

\section{Problemanalys}
%Huvudpoänger: 
% * Vad är det för problem projektet ska lösa?
% * Varför har vi har vi valt att lösa problemet så här? 
% * Knyter vårt problem an till ett större problem?
% * Ev. hänvisa kort till relaterad forskning

%Illustrationer: Statisktik från kurserna, pajdiagram

%Problemanalysen skall leda fram till produktspecen
Som nämndes i bakgrunden är problemet vi tacklar i vårt projekt att datastudenterna på Chalmers har svårt för två av sina kurser, TSS och Reglerteknik. 

Statistiken från Chalmers visar tydligt att kurserna TSS och Reglertiknik har en ovanligt hög andel studenter som inte blir godkända på tentamen. Under perioden 2010 till 2016 blev i genomsnitt endast 51% av de som skrev tentamen i TSS godkända och under samma period blev 53% godkända i Reglerteknik \cite{tentastatistik}. 
%Pajdiagram
Det finns även ett mörkertal i dessa siffror eftersom alla studenter inte alltid skriver tentamen, exempelvis valde 48 av 122 registrerade studenter att inte tentera Reglerteknik 2014 \cite{kursinformation:ere102:14-15}.

Kurserna i fråga, TSS och Reglerteknik, läses under tredje året på datateknik och kräver bland annat förkunskaper från de kurser i matematik studenterna läser under sitt första år. Reglerteknik kräver även förkunskaper från TSS. Det kan därför anses vara ett rimligt antagande att bristande kunskaper i TSS orsakar följdproblem och utgör en del av studenternas problem med Reglerteknik.

Kursernas examinatorer tror att studenternas svårigheter med kurserna till stor del beror på att studenterna inte är tillräckligt bekväma i det matematiska språket. Studenterna på Datateknik har inte tillräcklig vana att tolka innebörden av de matematiska beskrivningarna och är allmänt ovana vid matematiskt hantverk. Därför lägger datastudenterna det mesta av sin energi på att få ihop matematiken rent praktiskt, istället för att se vad matematiken står för. 

%Hänvisa till relaterad forskning och knyt an till bredare problem


\section{Produktspecifikation}
%Huvudpoänger: 
% * Beskriv produktbeskrivningen vi fick från början
% * Bena ut i detalj hur produktbeskrivningen vi tagit fram ser ut och kommentera vilka val vi har gjort.
% * Förklara hur produkten löser/bidrar till att lösa problemet

%Illustration: Ev bild på utdrag från tutorial för att tydliggöra upplägget

För att åtgärda det problem som beskrivs i förgående avsnitt har gruppen skapat en unik produkt som är anpassad speciellt till TSS på Chalmers datatekniska program. 
Den ursprungliga produktspecifikationen som gruppen tilldelades vid arbetets start var relativt vag. Gruppens uppgift var att “ta fram DSLsofMath-inspirerat kompletterande material för andra närliggande kurser”. Det önskade implementeringsspråket var haskell.

Gruppen beslutade snabbt att fokusera på kurserna TSS och Reglerteknik av de anledningar som nämns i problembeskrivningen. Dock konstaterade gruppen en bit in i projektet att det endast fanns tid att utveckla material av önskvärd kvalitet till en kurs. Gruppen valde då kursen TSS, med motiveringen att TSS innehåller förkunskaper som är nödvändiga för Reglerteknik. 

\subsection{Produktens struktur}
För att göra läromaterialet så lättillgängligt som möjligt beslutades att all materiel skulle samlas på en hemsida. På detta sätt kan man enkelt samla text, bilder och programmeringskod och presentera dem på ett överskådligt sätt för studenterna i målgruppen.

Det beslöts att läromaterialet skulle innehålla ett antal delavsnitt som tillsammans täckte in innehållet i kursen TSS i stora drag. Avsikten med indelningen är att göra materialet mer överskådligt och även göra det möjligt för studenterna att enkelt hitta aktuellt material för ett specifikt område.

Varje delavsnitt ska innehålla förklaringar och exempel i löpande text. Denna text ska skrivas på ett sådant sätt att den är lätt att ta till sig för målgruppen och uppmuntrar till vidare studier av ämnet.

Delavsnitten ska vidare innehålla exempel på hur det aktuella ämnet kan beskrivas med funktionell programmering och DSL, samt programmeringsövningar.

\subsection{Programmeringskod och övningar}

I själva impementationen av DSL så undviks funktioner som är
ovanliga eller kan skrivas explicit. t.ex så ska fouriertransformen av
$sinh(x)$ inte implementeras eftersom $sinh(x)$ kan utryckas via dess
definition $\frac{e^{x} - e^{-x}}{2}$ istället.
Dessutom undviks onödigt komplexa funktioner som inte
dyker upp naturligt som en signal, t.ex $sin(cos(x)+e^x)$.

\section{Teknisk bakgrund}

\subsection{DSL}
% Allmänt om DSL
Ett DSL är ett programmeringsspråk som till skillnad från ett \gls{GPL}, är anpassat
för endast en specifik domän. De har alltid funnits sedan programmering
skapades och exempel på DSL är HTML, MatLab och rpgmaker. Ett DSL kan
byggas upp från grunden eller vara utvecklat från ett redan existerande GPL
som Haskell. Fördelen med detta är att de erbjuder mer specialiserade egenskaper
inom detta specifika område men är inte avsedda för att kunna appliceras till andra
områden. Därför passar DSL utmärkt till detta kandidatarbete eftersom målet är
att lära ut TSS via programmering.

\subsection{Funktionell programmering}

\subsection{Signallära}

\subsection{Didaktik}
Didaktik är vetenskapen om undervisning. I \textit{Didaktik för ingenjörslärare} hävdar författarna att 
didaktik kan ses som ett komplement till pedagogik. De didaktiska frågorna handlar, enligt författarna, 
inte om hur studenten lär sig utan snarare om hur läraren skapar en situation lämplig för lärande. 

Som nämndes i inledningen har projektet “Matematikens domänspecifika språk” som syfte att utveckla 
läromaterial som kan fungera som ett komplement till kursen \textit{TSS} och väcka intresse för ämnet. 
Därför är didaktik en viktig del av projektet.

En stor del av didaktik handlar om hur man kan påverka studentens motivation. I \textit{Didaktik för 
ingegörslärare} sammanfattar motivationens betydelse enligt följande: ”Varje kurs behöver motiveras 
för studenterna. Med motiverade studenter blir resultatet bättre.”. \textit{Didaktik för ingegörslärare} 
hävdar även att inte alla studenter som går en kurs utgår från att den är läsvärd, långt ifrån. Även i 
\textit{Motivational Design for Learning and Performance} påpekas det orimliga i att anta att alla studenter är 
motiverade att lära sig ett ämne redan när de först kommer i kontakt med läromedlet. Det är alltså 
viktigt att ett läromedel är uppbyggt så att det motiverar studenterna att lära sig.

Enligt \textit{Motivational Design for Learning and Performance} behöver motivationsaktiviteter stödja 
inlärningsmålen för att vara effektiva. Motivationsmedel kan vara roligt och underhållande men om de 
inte engagerar studenten i läromålen och läromaterialet så hjälper det inte studenten att lära sig. 
Roliga aktiviteter och dylikt kan användas som ett belöningssystem men främjar i sig inte lärande. Många 
studenter kommer även motsätta sig eller rentav avsky aktiviteter tänkta att motivera dem om 
aktiviteterna inte är knytna till läromålen, detta gäller särskilt för vuxna studenter. 
Motivationsdeseign har alltså utmaningen att göra materialet tilltalande utan att det blir ren 
underhånning.

Ett angreppssätt inom didaktik är \textit{ARCS-modellen}, som beskrivs i detalj i \textit{Motivational 
Design for Learning and Performance}. ARCS är ett akronym för \textit{Attention}, \textit{Relevance}, 
\textit{Confidence} och \textit{Satisfaction}, vilket är grundpelarna i \textit{ARCS-modellen}. 

Den första grundpelaren \textit{Attention}, eller uppmärksamhet på svenska, syftar till att väcka studentens 
uppmärksamhet och nyfikenhet. Detta åstadkoms med perceptuellt triggande, frågor och variation. 
Perceptuellt triggande innebär att man på något sätt ändrar i undervisningsmiljön för att väcka 
studentens uppmärksamhet. Dessa miljöförändringar kan vara av varierande typ som t.ex. en ändring i 
röstläge, temperatur i en undervisningssal eller ett skämt. Dessa miljöförändringar väcker dock endast 
en tillfälligt uppmärksamhet och måste därför följas av frågor och variation för att ge en varaktig 
effekt. Därför är det viktigt att i detta skede omedelbart följa upp med frågor som gör väcker dennes 
nyfikenhet. För att behålla denna nyfikenhet och intresse är det viktigt med variation undervisningen. 
Oavsett hur bra upplägg man har i sin undervisning kommer studenterna tappa intresset om man inte 
varierar upplägget.

Den andra grundpelaren \textit{Relevance}, relevans på svenska, syftar till att övertyga studenten om att 
inlärningen är personligt relevant. För att en student ska vara intresserad av att lära sig måste de 
känna att instruktionen är relaterad till personliga mål eller motiv. Detta uppnås genom målorientering 
och anknytning till bekant kunskap. Studenter är mycket mer motiverade att lära sig saker om de kan 
hjälpa dem uppnå ett mål i nutid eller framtid, klara ett prov, få en befordran etc. Därför är det 
viktigt att studenterna vet hur det de lär sig är relaterat till deras mål. Studenter kan bli nyfikna 
på helt nya saker men de är som regel mest intresserade av saker som på något sätt är knutna till deras 
tidigare erfarenheter och intressen. Konkreta exempel från familjära områden gör det hela mer relevant 
för studenten, detta gäller i synnerhet när man lär ut abstrakt material. 

Den tredje grundpelaren \textit{Confidence}, självförtroende på svenska, syftar både till studenters tro på att de 
kan ämnet och deras tro på att de kan lära sig det. Även om en student är nyfiken på ett ämne och känner 
att det är relevant är det möjligt att hen ändå inte är tillräckligt motiverade på grund av för mycket 
eller för lite självförtroende. Vissa studenter kan rentav vara rädda för ämnet. På grund av detta är det 
viktigt att utveckla materialet så att studenten övertygas om att hen kan lära sig ämnet. Det är viktigt 
att så fort som möjligt ge studenter en upplevelse av framgång. Detta kan vara en viktigt stimulans för 
fortsatt motivation, förutsatt att det krävs så pass mycket ansträngning att det betyder något att lyckas, 
men inte så mycket ansträngning att det uppgiften leder till allvarlig oro eller hot om misslyckande. 
Studentens självförtroende kan byggas upp genom lärokrav, möjlighet till framgång och personlig kontroll. 
Det är viktigt att sätta upp tydliga läromål eftersom det är en stor källa till oro och osäkerhet för 
studenten att inte vete vad som förväntas av hen. Efter att ha skapat en förväntan av framgång är det 
viktigt att ge studenten möjlighet att lyckas med krävande och meningsfulla uppgifter. Dessa uppgifter 
bär se olika ut beroende på vilken nivå studenten är på. Studenter som är nya inom ett område tycker 
generellt bäst om att ha en relativt låg svårighetsnivå med frekvent feedback som hjälper dem lyckas 
eller bekräftar deras framgång. När studenter bemästrat grunderna är de redo för mer krävande uppgifter. 
Slutgiltligen är det viktigt att studenten känner att hen har personlig kontroll över lärandesituationen. 
Att uppleva att man personligen kan kontrollera utgången av en situation är avgörande för självförtroende. 
För att förstärka motivationen bör instruktören förse studenten med en stabil lärandemiljö och låta 
studenten ha mycket personlig kontroll över inlärningen. Det är viktigt att skapa en miljö där det är ok 
att göra misstag och lära sig av dem.

Om man använder de första tre grundpelarna, Attention, Relevance och Confidence, väl kommer studenten vara 
motiverad att lära sig. Det är för att upprätthålla denna motivation som \textit{Satisfaction}, tillfredsställelse 
på svenska, används.För att studenten ska ha ett fortsatt intresse av att lära sig måste hen känna 
tillfredsställelse med inlärningens process eller resultat. Denna tillfredställelse kan uppnås genom 
naturliga konsekvenser och positiva konsekvenser. Med naturliga konsekvenser menas den kunskap som blir 
en direkt följd av undervisningen. Att kunna lösa en uppgift eller dylikt som man inte kunde lösa innan 
ger en tillfredställelse i sig. Det är ett starkt belöningsverktyg att låta studenten använda ny kunskap. 
Andra positiva konsekvenser kan materiella belöningar, som t.ex. en löneförhöjning,  eller symboliska 
belöningar som diplom eller ett bara ett erkännande av studentens förmåga.

Hur dessa didaktiska modeller och metoder har använts i projektet “Matematikens domänspecifika språk” 
följer i avsnittet metod och genomförande.


\subsection{Relaterad forskning}


\section{Metod och Genomförande}

% Huvudpoänger:
% *Förstudier, intervjuer mm
% *Enkätundersökning
% 	*Hänvisa till böckerna “Enkäten i praktiken” och “Enkätboken”
% *Didaktik, hur vi skrivit vår tutorial och varför 
% 	*Hänvisa till “Didaktik för ingenjörer” och “Motivational Design for Learning and Performance”
% *TSS, hur vi har tacklat ämnet
% *Funktionell programmering och dsl, hur vi valt att implementera det hela och varför
% *Testning


I detta kapitlet beskrivs den tänkta tillvägagången till hur denna
handledning tillverkast.

\subsection{Litteraturstudier och Förundersökning (Temporär)}

% Detta ska ha underrubrikerna Förstudier och typer av undersökning

Projektdeltagarna skulle först återuppfriska kunskaperna om Haskell, TSS
och Reglerteknik. Därefter genomfördes vidare studier om DSL och pedagogik för
att kunna implementera formlerna i Haskell.

Efter förstudien så intervjuade vi de kursansvariga över deras tankar om vad
studenterna tyckte var svårt i kursen. Det ska även påbörjas en enkätundersökning
som vi tänker skicka ut till studenterna som går i årskurs 2 och 3 inom Datateknik.

% Ska litteratur tas med?

\subsection{Uppställning av handledning}


Handledningen är uppdelad i sex kapitel %planerat
där den första kapitlet ska vara en introduktion till koncepted DSL. Därefter ska det finnas
ett kapitel som kort förklarar om komplexa tal med Eulers formel samt olika typer av
signaler och deras egenskaper.

%Därefter kommer kapitel om LTI-system, Fourierserier och Fouriertransform
%samt Laplace- och Z-transform.

I varje kapitel ställde vi upp de definitioner och formler som tycktes
vara allmänt viktiga med tillhörande förklarande text. Den tillhörande
texten ska vara inspirerad av boken ``Learn You a Haskell for Great
Good'' av Miran Lipovača \cite{learnyouahaskell} och texter från sidan
\url{http://betterexplained.com/}.

Det vill säga, när vi förklarar hur formlerna fungerar så utnyttjar vi
oss av exempel som inte nödvändigtvis är helt korrekta. Målet är
att den ovanstående bilden ger en god uppfattning på hur formlerna kan användas.
Sedan förklarar vi hur det kan appliceras inom TSS i en mer matematiskt korrekt form.
Därefter tänker vi visa hur dessa definitioner eller formler skulle se ut i vårt DSL.

Dessutom ska det finnas exempeluppgifter till de områden studenterna finner
svåra och formulera lite uppgifter som ska kunna lösas via programmering.
%Lösningarna som finns med ska visa hur uppgiften kan lösas både med den vanliga
%analytiska metoden och via användandet av DSL. -Inte något vi har gjort.

\subsection{Implementation i DSL}

%Förklara hur komplexa tal och alla typer av transformer implementerades i DSL.
Vårt DSL ska vara uppbyggt från Haskell och ska täcka områdena TSS och Reglerteknik.
I båda kurserna så används komplexa tal vilket innebär att det först måste implementeras i vårt DSL.

\subsection{Verktyg}%behövs denna?
%Kanske skriva om hur vi använder google drive?
%Jag skulle säga att det bara är värt att ta upp verktyg som påverkar resultatet, t.ex. att vi skrivit koden i haskell, men att vi t.ex. arbetat fram texten i google drive känns inte så relevant att ha med här.
Gruppen ska använda sig av Google Documents för att skriva den första utkasten av handledningen.
Efter faktan och strukturen för varje kapitel blev färdigställd så överfördes det till 
\url{https://www.sharelatex.com/}, en webbsida som erbjuder en gemensam latex kompilator. 
Tillslut så ska den överföras till en HTML format som är slutprodukten. 

Matlab och \url{fooplot.com/} används till att skapa figurer på funktions kurvor eller vektor 
värden ifall det behövs till handledningen. För enkätundersökningen så används webbsidan 
\url{https://www.webbenkater.com/}.

Koden ska vara skriven i programmeringspråket Haskell och gruppen använde sig av \url{https://c9.io/} 
för att alla ska kunna ha tillgång till koden och en kompilator. 
Pågående under arbetets gång så laddas all någorlunda färdiga material upp i Github. 
Här kan gruppens handledare kolla igenom all produkt och ge sina kommentarer. 

%mer?

\subsection{Test av handledning}

Vi har ett tjugotal studenter som är villiga att testa vår produkt för
att se om den är lika hjälpsam som det är tänkt. Med responsen vi får
från dessa studenter kommer vi att göra förändringar för att få den mer
användbar och vara till mer hjälp. Vid testningen är tanken att
produkten nästan ska vara färdig så att vi får så viktig information som
möjligt.

\section{Resultat}

% Huvudpoänger:
% *Vad blev det för produkt? Tutorial på hemsida med 6 delavsnitt?
% *Hänvisa till hemsidan
% *Vad tyckte testgruppen?
% 	*Citat från testgruppen

%
% Illustrationer:
% * Utdrag ur tutorialen

\section{Diskussion}

% Huvudpoänger:
% * Har vi löst problemet?
% * Går problemet att lösa?
% * Vad är i grund och botten orsaken till problemet?
% * Vilka problem har dykt upp?
% * Vilka problem kan man undvika om man gör ett liknande arbete?
% * Stämmer vårt resultat överens med tidigare forskning?
% * Hur kan man bygga vidare på detta?


%Problem med hur vi implementerar integraler och annat kontinuerliga fall

Från intervjun förklarade examinatorerna att den största svårigheten studenterna fann i kursen
var kopplingen mellan matematiken och den bakomliggande fysiken. %Hur tacklade vi denna problem?

%Ett problem var själva storleken på uppgiften gruppen tog sig an. För att kunna skriva en 
%bra handledning så krävs det goda förkunskaper om både ämnet och pedagogik.

%Ta även upp varför reglerteknik inte togs med i projektet. 

\section{Slutsatser}
% Huvudpoänger:
% * Sammanfattar vad vi kommit fram till i diskussion och resultat

\section{Avslutning}
% Huvudpoänger:
% * Vad är gjort och vad återstår att göra?
%
% Illustration: Borde inte behövas

\newpage

\printbibliography

\section{Bilagor}
% * Utdrag ur tutorial? 
% * Utdrag ur intervju med examinatorer?
% * Utdrag ur enkätsvar?

\end{document}
