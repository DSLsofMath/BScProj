\newglossaryentry{DSL}
{
    name=DSL,
    description={Domain specific language eller domänspecifikt språk}
}

\newglossaryentry{EDSL}
{
    name=EDSL,
    description={Embedded DSL, DSL som är inbäddat i ett värdspråk}
}

\newglossaryentry{DSLsofMath}
{
    name=DSLsofMath,
    description={Domain specific languages of Mathematics, ett pedagogiskt projekt som utvecklar kursen DAT325, Matematikens domänspecifika språk}
}

\newglossaryentry{TSSmDSL}
{
    name={TSS med DSL},
    description={Ett kompletterande läromaterial som förklarar transformer, signal och system med hjälp av DSL}
}

%\newglossaryentry{GPL}
%{
%    name=GPL,
%    description={General purpose language t.ex C, C++, Java, Haskell m.m}
%}

\newglossaryentry{TSS}
{
    name=TSS,
    description={Transformer, Signaler och System; en kurs på Chalmers med kurskod SSY080}
}

\newglossaryentry{Haskell}
{
	name=Haskell,
	description={Ett funktionellt programmeringspråk.}
}

\newglossaryentry{D}
{
    name=D,
    description={Datateknik, en grundutbildning inom ingenjörskap}
}

\newglossaryentry{Didaktik}
{
    name=Didaktik,
    description={Läran om undervisning}
}

\newglossaryentry{ARCS}
{
    name=ARCS,
    description={Attention, Relevance, Confidence, Satisfaction; en didaktisk modell för att främja motivation}
}

\newglossaryentry{Git}
{
    name=Git,
    description={Ett versionshanteringsprogram}
}

\newglossaryentry{GitHub}
{
    name=GitHub,
    description={En hemsida som bygger på Git}
}