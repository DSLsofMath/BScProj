\documentclass{article}
\usepackage[utf8]{inputenc}

\title{Kapitel: Laplace Transform}
\author{ }
\date{March 2016}

\begin{document}

\maketitle

\section{Laplace Transform}

%Är osäker på hur mycket detaljer jag ska gå in i.
Laplace transform kan ses som en special fall av Fourier Transform som bara är definired för $t>0$ 
Dess definition är: 
$$F(s) = \mathcal{L} f(s) = \hat{f}(-js) = \int_{0}^{\infty} f(t)e^{-st} dt $$

%där s är en komplex tal. 
där $s = \sigma + j \omega$ och båda är reella. Vi kan observera att om $\sigma = 0$ så har vi definitionen 
för en Fourier transform. Var nu inte rädda för att ni måste behandla komplexa tal, Laplace transformen är 
egentligen en mer "vänlig" transform om en jämfört med Fourier trasnformen. %Och förmodligen kommer 
en inte märka av att det är en komplex tal alls.


Invers Laplace transform 
$$f(t) = \mathcal{L}^{-1} F(t) = \frac{1}{2 \pi i} \int_{b-jr}^{b+jr} F(s) e^{st} ds $$
där f ska konvergera punktvis när $r\rightarrow \infty$. 

%Vi kanske borde också förklara dess användning inom signal teori och grejer
Den mest användbara egenskapen av Laplace funktion är dess förmåga att hantera differential ekvationer. 
$$\mathcal{L} f'(t) = s F(s) - f(0)$$
eller 
$$\mathcal{L} f^{(k)} (t) = s^k F(s) - \sum_{0}^{k-1} z^{k-1-j} f^{(j)} (0)$$

Den hanterar även en faltning som
$$\mathcal{L} f*g (t) = F G $$
%Sen antar jag att jag bara tar upp mer exemplar? Boken tar inte upp mer viktig fakta om detta utan bara
exemplar på hur de applicerar detta på olika områden?


\end{document}
