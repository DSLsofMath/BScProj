\documentclass{article}
\usepackage[utf8]{inputenc}

\title{Kapitel: Laplace Transform}
\author{ }
\date{March 2016}

\begin{document}

\maketitle

\section{Laplace Transform}

%TODO: general comment: spell checking

% TODO: general comment: use my comments to consider adding some
% explanation (here or earlier) to the readers about easy mistakes to
% make. (If you make mistakes, you can bet _many_ others will make the
% same, or similar, mistakes. So these mistakes can be seen as a
% resource when it comes to making a pedagogical end result.)

%Är osäker på hur mycket detaljer jag ska gå in i. This is a ROUGH draft. More like a template if anything else.
Laplace transform kan ses som ett specialfall av Fourier Transform som bara är definiered för $t>0$
Dess definition är:
$$F(s) = \mathcal{L} (f)(s) = \hat{f}(-js) = \int_{0}^{\infty} f(t)e^{-st} dt $$

%där s är en komplex tal.
där $s = \sigma + j \omega$ och båda är reella. Vi kan observera att om $\sigma = 0$ så har vi definitionen
för en Fourier transform. Var nu inte rädda för att ni måste behandla komplexa tal, Laplace transformen är
egentligen en mer "vänlig" transform om en jämfört med Fourier trasnformen. 
%Och förmodligen kommer en inte märka av att det är ett komplex tal alls.
Och precis som Fourier transformen så vill vi transformera tillbaka funktionerna. 
Den inversa Laplace transformen är given ur:

Invers Laplace transform
$$f(t) = \mathcal{L}^{-1} (F)(t) = \frac{1}{2 \pi i} \int_{b-jr}^{b+jr} F(s) e^{st} ds $$
där f ska konvergera punktvis när $r\rightarrow \infty$.

%Vi kanske borde också förklara dess användning inom signal teori och grejer
Den mest användbara egenskapen av Laplace funktion är dess förmåga att hantera differential ekvationer.
%TODO: LHS uses a variable "t" but RHS uses (unbound) "s"
$$\mathcal{L} (f'(t)) = s F(s) - f(0)$$
eller
%TODO: LHS uses a variable "t" but RHS uses (unbound) "s"
$$\mathcal{L} (f^{(k)}) (t) = s^k F(s) - \sum_{0}^{k-1} z^{k-1-j} f^{(j)} (0)$$

Den har även följande egenskap 
$$\mathcal{L} (t f'(t)) = -F'(s) $$
%Andra viktiga transformer
%Frekvens skift \mathcal{L} (f e^{ct}) = F(s-c)
%Tids skift \mathcal{L} (H(t-a) f(t-a)) = e^{-as) F(s)
%Tids skalning \mathcal{L} (f) (at) = \frac{1}{a} F(z/a)
%Dirac Delta \mathcal{L} (\delta)(t) = 1 
%Ska dessa tas upp som exemplar eller bara nämnas?

%TODO: Use "(" ")" to clarify what \mathcal{L} is applied to
%TODO: Fix the variable binding
Ett vanligt fel med Laplace transform är transformen av en produkt mellan två funktioner är produkten av varderas transform. 
Detta är oftast inte fallet! 
$$\mathcal{L} (f g) \neq L G $$
Observera att transformen av en faltning ger produkten av varderas transform!
$$\mathcal{L} (f*g) (t) = F G $$
\appendix
%Signals, System and Transforms från Charles L. Phillips, John M. Parr
%Fourier Analysis and Its Apllications by Gerald B. Folland


\end{document}
