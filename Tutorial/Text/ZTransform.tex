\documentclass{article}
\usepackage[utf8]{inputenc}

\usepackage[T1]{fontenc}
\usepackage[swedish]{babel}
\usepackage[makeroom]{cancel}
\usepackage{amssymb, amsmath, lmodern, units, icomma, color, graphicx, bbm, hyperref, pdfpages, csquotes}
\title{Kapitel: Z Transformer}
\author{ }
\date{April 2016}
%TODO: stavning: integers := heltal, "en exempel" := "ett exempel", Introduction, talföjlder, utmärk,
%TODO: Välj "z" eller "Z" konsistent: nu finns "Z Transformer", "z-planet", "z-transform", "Z-transform", ...

\title{Kapitel: Z Transformer}
%\author{ }
%\date{April 2016}

\begin{document}

\maketitle

\section{Introduction}

%Detta är bara en draft på vad som kan stå i Z-transformen. Angående användningsområden så vet
%jag inte själv vad som kan stå eftersom i min kurs så nämndes Z-transform bara lite och att
%den kunde användas på differensekvationer. Jag utesluter de flesta bevis och teorier.
%Likt Laplacetransform som är ett specialfall eller generalisering av det tidskontinuerliga
%frekvensplanet så är Z-transform ett specialfall av det tidsdiskreta frekvensplanet.
Om ni kommer ihåg att Laplacetransformen är ett specialfall eller generalisering av Fouriertransformen i ett
tidskontinuerlig fall. Z-transformen är då den tidsdiskreta generaliseringen av Fouriertransformen.
%Det vill säga ett special fall av DFT.
Likt den diskreta Fouriertransformen så används Z-transformen på kausala talföljder $x[n]$ eller
diskreta funktioner $f[n]$. %Där n innefattar vilken position värdet $x[n]$ eller $x_n$ har i talföljden.
%Jag nämner att dessa talföljder kommer ifrån sampling av signaler i Kapitel Fourier.

Definition av Z-transform

%TODO: please work through the notation carefully. currently it is unclear what [n] is doing on the LHS.
$$\mathcal{Z}(x) = X = \sum_{n=-\infty}^{\infty} x[n] \cdot z^{-n} $$

Det viktiga med denna definition är att en tidskift med en konstant $k \in \mathbb{Z}$ på samplingen. Om $x[n]$ betecknar värdet x ger på tidpunkten $n \in \mathbb{Z}$ och X(z) är dess motsvarighet i Z-planet. Då är $x[n-k]$ en tidskift av $x[n]$ och har följande formen $z^{-k} X(z)$ i  Z-planet. Det vill säga

%TODO: Again, the notation needs more explanation: Z should be applied to a series, like x or (\n-> x (n-k)) and the RHS should bind z

$$\mathcal{Z}(x[n-k]) = z^{-k} X(z) $$
%$$\mathcal{Z}(x_{n-k}) = z^{-k} X(z) $$
Denna egenskap gör Z-transformen utmärkt till att lösa differensekvationer.

%Ta upp ett exempel på en differensekvation som löses med Z-transform.
%Självfallet så vill vi tillslut omtransformera och den inversa Z-transformen är per definition följande:
%$$ x[n] = \frac{1}{2 \pi j} \oint_R X[z] z^{n-1} dz $$
%Den är dock inte så viktig eftersom studenterna kommer använda sig av tabeller istället för integralen.
%Det som behövs är att förklara poler och grejer.
%Ett kausalt LTI system är BIBO-stabilt (Bounded input, Bounded output) om polerna ligger inuti enhetscirkeln.
%Behovs även en förklaring på ROC och dess roll inom den inversa Z-Transformen. Visa helst detta med ett exempel.

\appendix
%http://www.cse.chalmers.se/~svenk/dig_sign.tl/web_version/web_version.htm
%Signals, System and Transforms från Charles L. Phillips
%John M. Parr


\end{document}
