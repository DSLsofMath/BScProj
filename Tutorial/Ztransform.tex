\documentclass{article}
\usepackage[utf8]{inputenc}

\title{Kapitel: Z Transformer}
\author{ }
\date{April 2016}

\begin{document}

\maketitle

\section{Introduction}

%Detta är bara en draft på vad som kan stå i Z-transformen. Angående användningsområden så vet
%jag inte själv vad som kan stå eftersom i min kurs så nämndes Z-transform bara lite och att
%den kunde användas på differensekvationer. Jag utesluter de flesta bevis och teorier.
%Likt Laplacetransform som är en specialfall eller generalisering av den tidskontinuerliga
%frekvensplanet så är z-transform en specialfall av den tidsdiskrete frekvensplanet.
Om ni kommer ihåg att Laplacetransformen är en specialfall eller generalisering av Fouriertransformen i ett
tidskontinuerlig fall. Z-transformen är då den tidsdiskreta generaliseringen av Fourier transformen.
%Det vill säga en special fall av DFT.

Definition av Z-transform

%TODO: please work through the notation carefully. currently it is unclear what [n] is doing on the LHS.
$$\mathcal{Z}[x[n]] = X(z) = \sum_{n=-\infty}^{\infty} x[n] \cdot z^{-n} $$

Där $z = \Sigma + j\Omega$ eller .
Det viktiga med denna definition är att en tidskift på samplingen, det vill säga $x[n-k]$ är i
z-planet bara en multiplikation med $z^{-k}$. Med andra ord:
%TODO: Again, the notation needs more explanation
$$\mathcal{Z}[x[n-k]] = z^{-k} X(z) $$ %kanske inte behövs  %TODO: I do think this part is needed
Denna egenskap gör Z-transformen utmärk till att lösa differensekvationer.

\appendix
%http://www.cse.chalmers.se/~svenk/dig_sign.tl/web_version/web_version.htm
%Signals, System and Transforms från Charles L. Phillips
%John M. Parr


\end{document}
